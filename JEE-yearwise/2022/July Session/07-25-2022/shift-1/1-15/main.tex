\iffalse
\title{Assignment}
\author{ee24btech11059}
\section{mcq-single}
\fi
    \item{
          	The total number of functions, $f : \cbrak{1, 2, 3, 4} \to \cbrak{1, 2, 3, 4, 5, 6}$ such that $f\brak{1} + f\brak{2} = f\brak{3}$, is equal to \text{  }\hfill
                {[July 2022]}
                \begin{multicols}{4}
					\begin{enumerate}
						\item 60
						\item 90
						\item 108
						\item 126
					\end{enumerate}
				\end{multicols}
            }
    %code by ysiddhanth 
    \item{
           	If $\alpha$, $\beta$, $\gamma$, $\delta$ are the roots of the equation $x^4 + x^3 + x^2 + x + 1 = 0$, then $\alpha^{2022} + \beta^{2022} + \gamma^{2022} + \delta^{2022}$ is equal to
           	\hfill
           	{[July 2022]}
                \begin{multicols}{4}
                	\begin{enumerate}
                		\item -4
                		\item -1
                		\item 1
                		\item 4
                	\end{enumerate}
                \end{multicols}
        }
\item{
        	
        	For $n \in \mathbb{N}$  $S_n=\cbrak{z\in C\colon |z-3+2i|=\frac{n}{4} } \text{and}~ T_n=\cbrak{ z\ \in\ C \colon \left|z-2+3i\right|=\frac{1}{n}}$. Then the number of elements in the set $ \cbrak{n\ \in\ N: S_n\cap T_n=\phi}$ is
        	\hfill
        	{[July 2022]}
        	\begin{multicols}{4}
        		\begin{enumerate}
        			\item 0
        			\item 2
        			\item 3
        			\item 4
        		\end{enumerate}
        	\end{multicols}
        	
        }
    \item{
     
            The number of $q \in (0, 4\pi)$ for which the system of linear equations
            
            \begin{align*}
            	3\brak{\sin 3\theta} x - y + z &= 2 \\
            	3\brak{\cos 2\theta} x + 4y + 3z &= 3 \\
            	6x + 7y + 7z &= 9
            \end{align*}
            
            has no solution, is \text{ }
            \hfill
            {[July 2022]}
            \begin{multicols}{4}
                \begin{enumerate}
                	\item 6
                	\item 7
                	\item 8
                	\item 9
                \end{enumerate}
            \end{multicols}
        
        }
    \item{
            If  $\lim_{n\rightarrow\infty}\brak{\sqrt{n^2-n-1}+n\alpha+\beta}=0 \text{ then } 8(\alpha + \beta) \text{ is equal to}$
           	\hfill
                {[July 2022]}
            
           \begin{multicols}{4}
            	\begin{enumerate}
            		\item 4
            		\item -8
            		\item -4
            		\item 8
            	\end{enumerate}
            \end{multicols}
        
        }
 	\item{
        	If the absolute maximum value of the function $f\brak{x} = \brak{x^2 – 2x + 7} e^{\brak{4x^3-12x^2-180x + 31}}$ in the interval $\sbrak{-3, 0}$ is $f\brak{\alpha}$, then
        	\hfill
        	{[July 2022]}
        	
        	\begin{multicols}{4}
        		\begin{enumerate}
        			\item $\alpha = 0$
        			\item $\alpha = -3$
        			\item $\alpha \in \brak{-1, 0}$
        			\item $\alpha \in \lbrak{}-3, -1\rsbrak{}$
        		\end{enumerate}
        	\end{multicols}
        	
        }
 	\item{
			The curve $y\brak{x} = ax^3 + bx^2 + cx + 5$ touches the x-axis at the point P(–2, 0) and cuts the y-axis at the point Q, where $y'$ is equal to 3. Then the local maximum value of $y\brak{x}$ is
			\text{ }
			\hfill
			{[July 2022]}
			
			\begin{multicols}{4}
				\begin{enumerate}
					\item $\frac{27}{4}$
					\item $\frac{29}{4}$
					\item $\frac{37}{4}$
					\item $\frac{9}{2}$
				\end{enumerate}
			\end{multicols}
			
		}
 	\item{
			The area of the region given by $A=\cbrak{\brak{x,y};x^2\leq y \leq \text{min}\cbrak{x+2, 4-3x}}$ is \\text{ } \hfill
			{[July 2022]}
			
			\begin{multicols}{2}
				\begin{enumerate}
					\item $\frac{31}{8}$
					\item $\frac{17}{6}$
					\item $\frac{19}{6}$
					\item $\frac{27}{8}$
				\end{enumerate}
			\end{multicols}
			
		}
    \item{
            For any real number $x$, let $\sbrak{x}$ denote the largest integer less than or equal to $x$. Let $f$ be a real valued function defined on the interval $\sbrak{-10, 10}$ by $$ f\brak{x}=\left\{\begin{matrix}x-\sbrak{x}, \text{ if } \sbrak{x} \text{ is odd} \\1+\sbrak{x}-x, \text{if }\sbrak{x}\text{ is even}\end{matrix}\right.$$ Then the value of $\frac{\pi^2}{10}\displaystyle\int\limits_{-10}^{10}f\brak{x}\cos\pi x\ dx$ is
             \text{ }
             \hfill
                {[July 2022]}
            \begin{multicols}{4}
                \begin{enumerate}
                	\item 4
                    \item 2
                    \item 1
                    \item 0
                \end{enumerate}
            \end{multicols}

        %code by ysiddhanth 
        
        }
    \item{
	           The slope of the tangent to a curve $C : y = y\brak{x}$ at any point $\brak{x, y}$ on it is $\frac{2e^{2x}-6e^{-x}+9}{2+9e^{-2x}}$. If $C$ passes through the points $\left(0, \frac{1}{2}+\frac{\pi}{2\sqrt{2}}\right)\text{ and }\left(\alpha,\frac{1}{2}e^{2\alpha}\right),$ then $ e^\alpha $ is equal to 
            
             \hfill
                {[July 2022]}
            \begin{multicols}{2}
                \begin{enumerate}
                	\item $\frac{3+\sqrt{2}}{3-\sqrt{2}}$
                	\item $\frac{3}{\sqrt{2}}\brak{\frac{3+\sqrt{2}}{3-\sqrt{2}}}$
                	\item $\frac{1}{\sqrt{2}}\brak{\frac{\sqrt{2}+1}{\sqrt{2}-1}}$
                	\item $\frac{\sqrt{2}+1}{\sqrt{2}-1}$
                \end{enumerate}
            \end{multicols}
        
        }
    \item{
    		The general solution of the differential equation $(x - y^2)dx + y(5x + y^2)dy = 0$ is :\\ \text{ }
             \hfill
                {[July 2022]}
			\begin{multicols}{2}
				\begin{enumerate}
					\item $\brak{y^2+x}^4=C\left|\brak{y^2+2x}^3\right|$
					\item $\brak{y^2+2x}^4=C\left|\brak{y^2+x}^3\right|$
					\item $\left|\brak{y^2+x}^3\right|=C\brak{2y^2+x}^4$
					\item $\left|\brak{y^2+2x}^3\right|=C\brak{2y^2+x}^4$
				\end{enumerate}
			\end{multicols}
        
        }
    \item{
        	A line, with the slope greater than one, passes through the point $A\brak{4, 3}$ and intersects the line $x - y - 2 = 0$ at the point B. If the length of the line segment AB is $\frac{\sqrt{29}}{3}$, then B also lies on the line\hfill
                {[July 2022]}
				\begin{multicols}{4}
	                \begin{enumerate}
	                	\item $2x + y = 9$
	                	\item $3x - 2y = 7$
	                	\item $x + 2y = 6$
	                	\item $2x - 3y = 3$
	                \end{enumerate}
				\end{multicols}
        
        }
 \item{
    	
	    	Let the locus of the centre $(\alpha, \beta)$, $\beta > 0$, of the circle which touches the circle $x^2 + \brak{y - 1}^2 = 1$ externally and also touches the x-axis be $L$. Then the area bounded by $L$ and the line $y = 4$ is:
	    	\text{   }\hfill
	    	{[July 2022]}
	    	\begin{multicols}{4}
	    		\begin{enumerate}
	    			\item $\frac{32\sqrt{2}}{3}$
	    			\item $\frac{40\sqrt{2}}{3}$
	    			\item $\frac{64}{3}$
	    			\item $\frac{32}{3}$
	    		\end{enumerate}
	    	\end{multicols}
	    	
	    }
    \item{
	
		    Let $P$ be the plane containing the straight line $\frac{x-3}{9}=\frac{y+4}{-1}=\frac{z-7}{-5}$ and perpendicular to the plane containing the straight lines $\frac{x}{2}=\frac{y}{3}=\frac{z}{5}$ and $\frac{x}{3}=\frac{y}{7}=\frac{z}{8}.$ If $d$ is the distance $P$ from the point $\brak{2, -5, 11}$, then $d^2$ is equal to:
			\text{   }\hfill
			{[July 2022]}
			\begin{multicols}{4}
				\begin{enumerate}
						\item $\frac{147}{2}$
						\item 96
						\item $\frac{32}{3}$
						\item 54
				\end{enumerate}
			\end{multicols}
			
		}

    \item{
        
            Let ABC be a triangle such that $ \overrightarrow{BC}=\overrightarrow{a},\overrightarrow{CA}=\overrightarrow{b},\overrightarrow{AB}=\overrightarrow{c},\left|\overrightarrow{a}\right|=6\sqrt{2},\left|\overrightarrow{b}\right|=2\sqrt{3}$ and $\overrightarrow{b}\cdot\overrightarrow{c}=1.$ 
            Consider the statements : $$ \left(S1\right):\left|\left(\overrightarrow{a}\times\overrightarrow{b}\right)+\left(\overrightarrow{c}\times\overrightarrow{b}\right)\right|-\left|\overrightarrow{c}\right|=6\left(2\sqrt{2}-1\right)$$ $$ \left(S2\right):\angle ACB=\cos^{-1}\left(\sqrt{\frac{2}{3}}\right) $$, then \text{ }
             \hfill
              {[July 2022]}
			\begin{multicols}{2}              
	              		\begin{enumerate}
	              			\item Both (S1) and (S2) are true
	              			\item Only (S1) is true
	              			\item Only (S2) is true
	              			\item Both (S1) and (S2) are false
	              	\end{enumerate}
  			\end{multicols}      
        }




