\iffalse
\title{July 2022, shift 2}
\author{EE24BTECH11062}	
\section{mcq-single}
\fi
%\begin{enumerate}
   
\item For $z\in C$ if the minimum value of $\brak{\abs{z-3\sqrt{2}}+\abs{z-p\sqrt{2}i}}$ is $5\sqrt{2}$, then a value of $p$ is \hfill{[July 2022]}
\begin{multicols}{4}
    a) 3\\
    b) $\frac{7}{2}$\\
    c) 4\\
    d) $\frac{9}{2}$
\end{multicols}
 \item The number of real values $\lambda$, such that the system of linear equations $2x-3y+5z=9\\ 
 x+3y-z=-18\\
 3x-y+\brak{\lambda^2 - \abs{\lambda}}z=16$ has no solution, is \hfill{[July 2022]}
 \begin{multicols}{4}
     a) 0\\
     b) 1\\
     c) 2\\
     d) 4
 \end{multicols}
 
 \item The number of bijective functions $f:{1,3,5,7,\dots 99}\mapsto {2,4,6,8,\dots 100}$, such that $f\brak{3}\geq f\brak{9}\geq f\brak{15}\geq f\brak{21}\geq \dots \geq f\brak{99}$, is \hfill{[July 2022]}
 \begin{multicols}{4}
    a) $^{50} P_{17}$\\
    b) $^{50} P_{33}$\\
    c) $33! \times 17!$\\
    d) $\frac{50!}{2}$
 \end{multicols}
 
\item The remainer when $\brak{11}^1011+\brak{1011}^11$ is divided by 9 is \hfill{[July 2022]}
\begin{multicols}{4}
    a) 1\\
    b) 4\\
    c) 6\\
    d) 8
\end{multicols}

\item The sum $\sum_{n=1}^{21} \frac{3}{\brak{4n-1}\brak{4n+3}}$ is equal to \hfill{[July 2022]}
\begin{multicols}{4}
    a) $\frac{7}{87}$\\
    b) $\frac{7}{29}$\\
    c) $\frac{14}{87}$\\
    d) $\frac{21}{29}$
\end{multicols}
\item $\lim_{x \to \frac{\pi}{4}} \frac{8\sqrt{2}-\brak{\cos x +\sin x}^7}{\sqrt{2}-\sqrt{2}\sin 2x}$ is equal to \hfill{[July 2022]}
\begin{multicols}{4}
    a) 14\\
    b) 7\\
    c) $14\sqrt{2}$\\
    d) $7\sqrt{2}$
\end{multicols}
\item $\lim_{x \to \frac{1}{2^n}} \brak{\frac{1}{\sqrt{1-\frac{1}{2^n}}}+\frac{1}{\sqrt{1-\frac{2}{2^n}}}+\frac{1}{\sqrt{1-\frac{3}{2^n}}}+\dots +\frac{1}{\sqrt{1-\frac{2^n -1}{2^n}}}}$ is equal to \hfill{[July 2022]}
\begin{multicols}{4}
    a) $\frac{1}{2}$\\
    b) 1\\
    c) 2\\
    d) -2
\end{multicols}
\item If $A$ and $B$ are two events such that $P\brak{A}=\frac{1}{3}, P\brak{B}=\frac{1}{5}$ and $P\brak{A\cup B}=\frac{1}{2}$ then $P\brak{A\mid B^{\prime}}+P\brak{B\mid A^{\prime}}$ is equal to \hfill{[July 2022]}
\begin{multicols}{4}
    a) $\frac{3}{4}$\\
    b) $\frac{5}{8}$\\
    c) $\frac{5}{4}$\\
    d) $\frac{7}{8}$
\end{multicols}
\item Let $\sbrak{t}$ denote the greatest integer less than or equal to $t$. Then the value of the integral $\int_{-3}^{101} \brak{\sbrak{\sin\brak{\pi x}}}+e^{\cos\brak{2\pi x}} \, dx$ is equal to \hfill{[July 2022]}
\begin{multicols}{4}
    a) $\frac{52\brak{1-e}}{e}$\\
    b) $\frac{52}{e}$\\
    c) $\frac{52\brak{2+e}}{e}$\\
    d) $\frac{104}{e}$
\end{multicols}
\item Let the point $\vec{P}\brak{\alpha,\beta}$ be at a unit distance from each of the two lines $L_1:3x-4y+12=0$ and $L_2:8x+6y+11=0$. If $\vec{P}$ lies below $L_1$ and above $L_2$, then $100\brak{\alpha+\beta}$ is equal to 
\begin{multicols}{4} \hfill{[July 2022]}
    a) -14\\
    b) 42\\
    c) -22\\
    d) 14
\end{multicols}
\item Let a smooth curve $y=f\brak{x}$ be such that the slope of the tangent at any point \brak{x,y} on it is directly proportional to $\frac{-y}{x}$. If the curve passes through the point \brak{1,2} and \brak{8,1}, then $\abs{y\brak{\frac{1}{8}}}$ is equal to \hfill{[July 2022]}
\begin{multicols}{4}
    a) $2\ln 2$\\
    b) 4\\
    c) 1\\
    d) $4\ln 2$
\end{multicols}
\item If the ellipse $\frac{x^2}{a^2}+\frac{y^2}{b^2}=1$ meets the line $\frac{x}{7}+\frac{y}{2\sqrt{6}}=1$ on the x-axis and the line $\frac{x}{7}-\frac{y}{2\sqrt{6}}=1$ on the y-axis, then the eccentricity of the ellipse is \hfill{[July 2022]}
\begin{multicols}{4}
    a) $\frac{5}{7}$\\
    b) $\frac{2\sqrt{6}}{7}$\\
    c) $\frac{3}{7}$\\
    d) $\frac{2\sqrt{5}}{7}$
\end{multicols}
\item The tangents at the point $\vec{A}\brak{1,3}$ and $\vec{B}\brak{1,-1}$ on the parabola $y^2-2x-2y=1$ meet at the point $\vec{P}$. Then the area of the triangle $PAB$ is \hfill{[July 2022]}
\begin{multicols}{4}
    a) 4\\
    b) 6\\
    c) 7\\
    d) 8
\end{multicols}
\item If the foci of the ellipse $\frac{x^2}{16}+\frac{y^2}{7}=1$ and the hyperbola $\frac{x^2}{144}-\frac{y^2}{\alpha}=\frac{1}{25}$ coincide. Then the length of the latus rectum of the hyperbola is \hfill{[July 2022]}
\begin{multicols}{4}
    a) $\frac{32}{9}$\\
    b) $\frac{18}{5}$\\
    c) $\frac{27}{4}$\\
    d) $\frac{27}{10}$
\end{multicols}
\item A plane $E$ is perpendicular to the two planes $2x-2y+z=0$ and $x-y+2z=4$, and passes through the point $\vec{P}\brak{1,-1,1}$. If the distance of the plane $E$ from the point $\vec{Q}\brak{a,a,2}$ is $3\sqrt{2}$. Then $\brak{PQ}^2$ is equal to \hfill{[July 2022]}
\begin{multicols}{4}
    a) 9\\
    b) 12\\
    c) 21\\
    d) 33
\end{multicols}
%\end{enumerate}
%\end{document}

