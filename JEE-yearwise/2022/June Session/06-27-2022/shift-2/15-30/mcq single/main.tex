\iffalse
\title{2022}
\author{ai24btech11028 - Ronit Ranjan}
\section{mcq-single}
\fi 

    \item The mean and variance of the data $4, 5, 6, 6, 7, 8, x, y$ where $x < y$ are $6$ and $\frac{9}{4}$ respectively. Then $x^4 + y^2$ is equal to \hfill{[June 2020]}
    \begin{multicols}{2}
    \begin{enumerate}
        \item $162$
        \item $320$
        \item $674$
        \item $420$
    \end{enumerate}
    \end{multicols}

    \item If a point $A\brak{x, y}$ lies in the region bounded by the y-axis, straight lines $2y + x = 6$ and $5x - 6y = 30$, then the probablity that $y<1$ is: \hfill{[Jun 2020]}
    \begin{multicols}{2}
    \begin{enumerate}
        \item $\frac{1}{6}$
        \item $\frac{5}{6}$
        \item $\frac{2}{3}$
        \item $\frac{6}{7}$
    \end{enumerate}
    \end{multicols}    

    \item The value of $\cot \brak{ \sum_{n=1}^{50} \tan^{-1} \brak{ \frac{1}{1+n+n^2} }}$ is  \hfill{[June 2020]}
    \begin{multicols}{2}
    \begin{enumerate}
        \item $\frac{26}{25}$
        \item $\frac{25}{26}$
        \item $\frac{50}{51}$
        \item $\frac{52}{51}$
    \end{enumerate}
    \end{multicols}    

    \item $\alpha = \sin 36\degree$ is a root of which of the following equation \hfill{[June 2020]}
    \begin{multicols}{2}
    \begin{enumerate}
        \item $10x^4 - 10x^2 -5 = 0$
        \item $16x^4 + 20x^2 -5 = 0$
        \item $16x^4 - 20x^2 +5 = 0$
        \item $16x^4 - 10x^2 +5 = 0$
    \end{enumerate}
    \end{multicols}

    \item Which of the following statement is a tautology?  \hfill{[June 2020]}
    \begin{multicols}{2}
    \begin{enumerate}
        \item $\brak{\brak{\sim q \cap p}\cap q} $
        \item $\brak{\brak{\sim q} \cap p} \cap \brak{p \cap \brak{\sim p}}$
        \item $\brak{\brak{\sim q} \cap p} \cup \brak{ \cup (\sim p))}$
        \item $((p\cap \hat q)\cup (\sim(p \cap q))$
    \end{enumerate}
    \end{multicols}
