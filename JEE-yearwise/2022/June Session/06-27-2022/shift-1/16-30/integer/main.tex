\iffalse
\title{Assignment-2}
\author{EE24BTECH11049}
\section{integer}
\fi

%\begin{document}
%\begin{enumerate}

    %1st Question
    \item 
    let $f: \mathbf{R} \mapsto \mathbf{R}$ be a function defined by 
    \begin{align*}
        f\brak{x} = \frac{2e^{2x}}{e^{2x} + e^{x}}
    \end{align*}
    then $f\brak{\frac{1}{100}} + f\brak{\frac{2}{100}} + f\brak{\frac{3}{100}} + \dots + f\brak{\frac{99}{100}}$ is equal to \rule{1cm}{0.1pt}

    \hfill{\sbrak{\text{Jun 2022}}}

    %2nd Question 
    \item
    If the sum of all the roots of the equation 
    \begin{align*}
        e^{2x} - 11e^x - 45e^{-x} +\frac{81}{2}=0
    \end{align*}
    is $\log_e p$, then $p$ is equal to \rule{1cm}{0.1pt}

    \hfill{\sbrak{\text{Jun 2022}}}

    %3rd Question 
    \item 
    The positive value of the determinant of the matrix $A$, whose
    \begin{align*}
        \text{adj}\brak{\text{adj}\brak{A}} = \myvec{14 & 28 & -14 \\ -14 & 14 & 28 \\ 28 & -14 & 14},
    \end{align*}
    is \rule{1cm}{0.1pt}

    \hfill{\sbrak{\text{Jun 2022}}}

    %4th Question 
    \item 
    The number of ways, $16$ identical cubes, of which $11$ are blue and rest are red, can be placed in a row so that between any two red cubes there should be at least $2$ blue cubes, is \rule{1cm}{0.1pt}

    \hfill{\sbrak{\text{Jun 2022}}}

    %5th Question 
    \item 
    If the coefficient of $x^{10}$ in the binomial expansion of 
    \begin{align*}
        \brak{\frac{\sqrt{x}}{5^{\frac{1}{4}}} + \frac{\sqrt{5}}{x^{\frac{1}{3}}}}^{60}
    \end{align*}
    is $5^kl$ where $l,k \in \mathbf{N}$ and $l$ is co-prime to 5, then $k$ is equal to \rule{1cm}{0.1pt}

    \hfill{\sbrak{\text{Jun 2022}}}

    %6th Question 
    \item 
    \begin{align*}
        A_1 = \cbrak{\brak{x,y} : \abs{x} \leq y^2, \abs{x} + 2y \leq 8} 
        \text{ and }
        A_2 = \cbrak{\brak{x,y} : \abs{x} + \abs{y} \leq k}.
    \end{align*}
    if $27\text{Area}\brak{A_1} = 5\text{Area}\brak{A_2}$, then $k$ is equal to: \rule{1cm}{0.1pt}

    \hfill{\sbrak{\text{Jun 2022}}}

    %7th Question 
    \item 
    If the sum of the first ten terms of the series 
    \begin{align*}
        \frac{1}{5} + \frac{2}{65} + \frac{3}{325} + \frac{4}{1025} + \frac{5}{2501} + \dots \text{ is } \frac{m}{n},
    \end{align*}
    where $m$ and $n$ are co-prime numbers, then $m + n$ is equal to \rule{1cm}{0.1pt}

    \hfill{\sbrak{\text{Jun 2022}}}

    %8th Question 
    \item 
    A rectangle $R$ with end points of one of its sides as \brak{1, 2} and \brak{3, 6} s inscribed in a circle. If the equation of a diameter of the circle is  $2x - y + 4 = 0$ , then the area of $R$ is \rule{1cm}{0.1pt}

    \hfill{\sbrak{\text{Jun 2022}}}

    %9th Question 
    \item 
    A circle of radius $2$ unit passes through the vertex and the focus of the parabola $y^2 =  2x$ and touches the parabola $y = \brak{x-\frac{1}{4}}^2 + \alpha$ where $\alpha > 0$. Then $\brak{4\alpha - 8}^2$ is equal to \rule{1cm}{0.1pt}

    \hfill{\sbrak{\text{Jun 2022}}}

    %10th Question 
    \item 
    Let the mirror image of the point $\brak{a, b, c}$ with respect to the plane $ 3x - 4y + 12z + 19 = 0 $ be $\brak{\alpha - 6, \beta, \gamma}$. if $a + b + c = 5$, then $7\beta - 9\gamma$ is equal to \rule{1cm}{0.1pt}
    
    \hfill{\sbrak{\text{Jun 2022}}}
    
%\end{enumerate}
%\end{document}
