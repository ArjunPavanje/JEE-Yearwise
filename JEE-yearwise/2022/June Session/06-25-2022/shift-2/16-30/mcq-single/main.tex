\iffalse
  \title{Assignment}
  \author{ee24btech11030}
  \section{mcq-single}
\fi

%   \begin{enumerate}
\item A biased die is marked with numbers 2, 4, 8, 16, 32, 32 on its faces and the probability of getting a face with mark n is $\frac{1}{n}$. If the die is thrown thrice, then the probability, that the sum of the numbers obtained is 48, is :  \\ \hfill{[JUN 2022]}
    \begin{multicols}{4}
    \begin{enumerate}
        \item $\frac{7}{2^{11}}$
        \item $\frac{7}{2^{12}}$
        \item $\frac{3}{2^{10}}$
        \item $\frac{13}{2^{12}}$
    \end{enumerate}
    \end{multicols}
    \bigskip
    \item The negation of the Boolean expression $((\sim q) \land p) \implies ((\sim p) \lor q)$ is logically equivalent to :  \\\hfill{[JUN 2022]}
    \begin{multicols}{4}
    \begin{enumerate}
        \item $p \implies q$
        \item $q \implies p$
        \item $\sim (p \implies q)$
        \item $\sim (q \implies p)$
    \end{enumerate} 
    \end{multicols}
    \bigskip
    \item If the line $y = 4 + kx$, k $>$ 0, is the tangent to the parabola $y = x - x^2$ at the point $\vec{P}$ and $\vec{V}$ is the vertex of the parabola, then the slope of the line through $\vec{P}$ and $\vec{V}$ is :  \\\hfill{[JUN 2022]}
    \begin{multicols}{4}
    \begin{enumerate}
        \item $\frac{3}{2}$
        \item $\frac{26}{9}$
        \item $\frac{5}{2}$
        \item $\frac{23}{6}$
    \end{enumerate}
    \end{multicols}
    \bigskip
    \item The value of $tan^{-1}{\left(\frac{\cos{\frac{15\pi}{4}} - 1}{\sin{\frac{\pi}{4}}}\right)}$ is equal to: \\\hfill{[JUN 2022]}
    \begin{multicols}{4}
    \begin{enumerate}
        \item $\frac{-\pi}{4}$
        \item $\frac{-\pi}{8}$
        \item $\frac{-5\pi}{12}$
        \item $\frac{-4\pi}{9}$
    \end{enumerate} 
    \end{multicols}
    \bigskip
    \item  The line $y = x + 1$ meets the ellipse $\frac{x^2}{4} + \frac{y^2}{2} = 1$ at two points $\vec{P}$ and $\vec{Q}$. If r is the radius of the circle with PQ as diameter then $(3r)^2$ is equal to : \\\hfill{[JUN 2022]}
    \begin{multicols}{4}
    \begin{enumerate}
        \item 20
        \item 12
        \item 11
        \item 8
    \end{enumerate} 
    \end{multicols}
%   \end{enumerate}
