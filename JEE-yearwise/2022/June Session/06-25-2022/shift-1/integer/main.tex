\iffalse
\title{Assignment-4}
\author{EE24BTECH11048-NITHIN.K}
\section{integer}
\fi
%\begin{enumerate}
%1
\item Let $C_r$ denote the binomial coefficient of $x^r$ in the expansion of $\brak{1 + x}^{10}$. If $\alpha, \beta \in R$. $C_1 + 3\cdot2C_2 + 5\cdot3C_3 + ...$ upto 10 terms = $\frac{\alpha\times2^{11}}{2^{\beta} - 1}\brak{C_0 +\frac{C_1}{2} + \frac{C_2}{3} + ...upto 10 terms}$ then the value of $\alpha + \beta$ is equal to
%2
\item The number of 3-digit odd numbers, whose sum of digits is a multiple of 7, is
%3
\item Let $\theta$ be the angle between the vectors $\vec{a}$ and $\vec{b}$, where $|\vec{a}|=4,|\vec{b}|=3$, $\theta \in \brak{\frac{\pi}{4},\frac{\pi}{3}}$. Then $|\brak{\vec{a}-\vec{b}} x \brak{\vec{a}+\vec{b}}|^2 + 4\brak{\vec{a}\cdot\vec{b}}^2$ is equal to
%4
\item Let the abscissae of the two points P and Q be the roots of $2x^2 - rx + p = 0$ and the ordinates of P and Q be the roots of $x^2-sx-q = 0$. If the equation of the circle described on PQ as diameter is $2\brak{x^2 + y^2}$ - 11x -14y - 22 =0, then 2r + s - 2q + p is equal to
%5
\item The number of values of x in the interval $\brak{\frac{\pi}{4},\frac{7\pi}{4}}$ for which $14\cosec^2x - 2\sin^2x = 21 - 4\cos^2x$ holds, is
%6
\item For a natural number n, let $a_n = 19^n - 12^n$. Then, the value of $\frac{31\alpha_9 -\alpha_10}{57\alpha_8}$ is
%7
\item Let f : $R \rightarrow R$ be a function defined by $f\brak{x} = \brak{2\brak{1-\frac{x^{25}}{2}}\brak{2+x^{25}}}^{\frac{1}{50}}$. If the function $g\brak{x} = f\brak{f\brak{f\brak{x}}} + f\brak{f\brak{x}}$, then the greatest integer less than or equal to g\brak{1} is
%8
\item Let the lines \\
	$L_1$ : $\vec{r} = \lambda\brak{\vec{i}+2\vec{j}+3\vec{k}}$, $\lambda \in R$ \\
	$L_2$ : $\vec{r} = \brak{\vec{i} + 3\vec{j} + \vec{k}} + \mu\brak{\vec{i} + \vec{j} +5\vec{k}}$ ; $\mu \in R$ \\
	intersect at the point S. If a plane $ax+by-z+d=0$ passes through S and is parallel to both the lines $L_1 andL_2$, then the value of a+b+d is equal to
%9
\item Let A be a 3 x 3 matrix having entries from the set $\cbrak{-1,0,1}$. The number of all such matrices A having sum of all entries equal to 5, is
%10
\item The greatest integer less than or equal to the sum of first 100 terms of the sequence $\frac{1}{3},\frac{5}{9},\frac{19}{27},\frac{65}{81}$,... is equal to
%\end{enumerate}
