\iffalse
    \title{2020}
    \author{EE24BTECH11021}
    \section{mcq-single}
\fi
    \item If $x^3dy+xy dx=x^2dy+2y dx;y\brak{2}=e$ and $x\textgreater 1$, then $y\brak{4}$ is equal to $\colon$
    \hfill{[Sep-2020]}
        \begin{enumerate}
            \item $\frac{3}{2}+\sqrt{e}$
            \item $\frac{3}{2}\sqrt{e}$
            \item $\frac{1}{2}+\sqrt{e}$
            \item $\frac{\sqrt{e}}{2}$
        \end{enumerate}
    \item Let $e_1$ and $e_2$ be eccentricities of the ellipse,$\frac{x^2}{25}+\frac{y^2}{b^2}=1\brak{b\textless 5}$ and the hyperbola, $\frac{x^2}{16}-\frac{y^2}{b^2}=1$ respectively satisfying $e_1e_2=1$. If $\alpha$ and $\beta$ are the distances between the foci of the ellipse and the foci of the hyperbola respectively, then the ordered pair $\brak{\alpha,\beta}$ is equal to $\colon$
    \hfill{[Sep-2020]}
        \begin{enumerate}
            \item $\brak{8,10}$
            \item $\brak{8,12}$
            \item $\brak{\frac{20}{3},12}$
            \item $\brak{\frac{24}{5},10}$
        \end{enumerate}
    \item The set of all real values of $\lambda$ for which the quadratic equations,\\ $\brak{\lambda^2+1}x^2-4\lambda x+2=0$ always has exactly one root in the interval $\brak{0,1}$ is $\colon$
    \hfill{[Sep-2020]}
        \begin{enumerate}
            \item $\brak{-3,-1}$
            \item $\left(1,3 \right]$
            \item $\brak{0,2}$
            \item $\left(2,4 \right]$
        \end{enumerate}
    \item If the term independent of $x$ in the expansion of $\brak{\frac{3}{2}x^2-\frac{1}{3x}}^9$ is $k$, then $18k$ is equal to $\colon$
    \hfill{[Sep-2020]}
        \begin{enumerate}
            \item $9$
            \item $11$
            \item $5$
            \item $7$
        \end{enumerate}
    \item Let $p,q,r$ be three statements such that the truth value of $\brak{p\wedge q}\rightarrow\brak{\sim p\vee r}$ is $F$. The truth values of $p,q,r$ are respectively$\colon$
    \hfill{[Sep-2020]}
        \begin{enumerate}
            \item $F,T,F$
            \item $T,F,T$
            \item $T,T,F$
            \item $T,T,T$
        \end{enumerate}
