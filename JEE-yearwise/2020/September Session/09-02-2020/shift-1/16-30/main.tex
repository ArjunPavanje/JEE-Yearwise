\iffalse
\title{02-sep-2020 shift-1}
\author{EE24BTECH11060}
\section{mcq-single}
\fi
%\begin{enumerate} [start=16]
    \item Let $\alpha$ \textgreater $0$, $\beta$\textgreater$0$ be such that $\alpha^3+\beta^3=4$.If the maximum value of the term independent of $x$ in the binomial expansion of \brak{\alpha x^\frac{1}{9}+\beta x^\frac{-1}{6}} is $10k$,then $k$ equals to:
    \hfill(sep-2020)
    \begin{enumerate}
        \item $176$
        \item $336$
        \item $352$
        \item $84$
    \end{enumerate}
    \item Let $S$ be the set of all $\lambda$ $\in$ $R$ for which the system of linear equations \\
    $2x-y+2z$=$2$\\
    $x-2y+\lambda z$=$-4$\\
    $x+\lambda y+z$=$4$ has no solution. Then the set $S$
    \hfill(sep-2020)
    \begin{enumerate}
        \item is an empty set
        \item is a singleton
        \item contains more than two elements.
        \item contains exactly two elements.
    \end{enumerate}
    \item Let $X$=$\cbrak{x\in N:1\leq x\leq 17}$ and $Y$=$\cbrak{ax+b:x \in X and a,b \in R,a \textgreater0}$.If mean and variance of elements of $Y$ are $17$ and $216$ respectively then $a+b$ is equal to:
    \hfill(sep-2020)
    \begin{enumerate}
        \item $27$
        \item $7$
        \item $-7$
        \item $9$
    \end{enumerate}
    \item Let $y$=$y\brak{x}$ be the solution of the differential equation, $\frac{2+\sin{x}}{\brak{y+1}\brak{\frac{dy}{dx}}}$=$-\cos{x}$, $y$\textgreater $0$,$y\brak{0}$=$1$.If $y\brak{\pi}$=$a$,and \brak{\frac{dy}{dx}}at $x$ = $\pi$ is $b$, then the ordered pair $\brak{a,b}$ is equal to:
    \hfill(sep-2020)
    \begin{enumerate}
        \item $\brak{2,\frac{2}{3}}$
        \item $\brak{1,1}$
        \item $\brak{2,1}$
        \item $\brak{1,-1}$
    \end{enumerate}
    \item The plane passing through the points $\brak{1,2,1}$, $\brak{2,1,2}$ and parallel to the line, $2x$ = $3y$, $z$=$1$ also passes through the point:
    \hfill(sep-2020)
    \begin{enumerate}
        \item $\brak{0,-6,2}$
        \item $\brak{0,6,-2}$
        \item $\brak{-2,0,1}$
        \item $\brak{2,0,-1}$
    \end{enumerate}
    
    
%\end{enumerate}


