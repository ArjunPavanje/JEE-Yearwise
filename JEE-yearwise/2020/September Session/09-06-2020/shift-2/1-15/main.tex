\iffalse
\title{Assignment}
\author{ee24btech11059}
\section{mcq-single}
\fi

    \item{
          	If the normal at an end of a latus rectum of an ellipse passes through an extremity of the minor axis, then the eccentricity e of the ellipse satisfies: \\ \text{  }\hfill
                {[Sep 2020]}
            \begin{multicols}{2}
				\begin{enumerate}
					\item $e^4+2e^2 - 1 = 0$
					\item $e^2+2e - 1 = 0$
					\item $e^4+e^2 - 1 = 0$
					\item $e^2+e - 1 = 0$
				\end{enumerate}
			\end{multicols}
            }
    %code by ysiddhanth 
    \item{
            The set of all real values of $\lambda$ for which the function $f\brak{x} = \brak{1- \cos^2 x}\brak{\lambda+\sin x} , x\in\brak{-\frac{\pi}{2}, \frac{\pi}{2}}$, has exactly one maxima and exactly one minima, is:\hfill
                {[Sep 2020]}
            \begin{multicols}{2}
                \begin{enumerate}
                    \item $\brak{-\frac{3}{2}, \frac{3}{2}}-\cbrak{0}$
                    \item $\brak{-\frac{1}{2}, \frac{1}{2}}-\cbrak{0}$
                    \item $\brak{-\frac{3}{2}, \frac{3}{2}}$
                    \item $\brak{-\frac{1}{2}, \frac{1}{2}}$
                \end{enumerate}
            \end{multicols}
        }
\item{
        	
        	The probabilities of three events $A$, $B$ and $C$ are given by $P\brak{A} = 0.6$, $P\brak{B} = 0.4$ and $P\brak{C} = 0.5$. If $P\brak{A \cup B} = 0.8$, $P\brak{A \cap C} = 0.3$, $P\brak{A \cap B \cap C} = 0.2$, $P\brak{B \cap C} = \beta$ and $P\brak{A \cup B \cup C} = \alpha$, where $0.85 \leq \alpha \leq 0.95$, then $\beta$ lies in the interval. \\
        	\hfill
        	{[Sep 2020]}
        	\begin{multicols}{4}
        		\begin{enumerate}
        			\item $\sbrak{0.36,0.40}$
        			\item $\sbrak{0.25,0.35}$
        			\item $\sbrak{0.35,0.36}$
        			\item $\sbrak{0.20,0.25}$
        		\end{enumerate}
        	\end{multicols}
        	
        }
    \item{
     
            The common difference of the $A.P.$ $b_1, b_2, \ldots, b_m$ is $2$ more than the common difference of $A.P.$ $a_1, a_2, \ldots, a_n$. If $a_{40} = -159$, $a_{100} = -399$ and $b_{100} = a_{70}$, then $b_1$ is equal to:\hfill
                {[Sep 2020]}
            \begin{multicols}{4}
                \begin{enumerate}
                    \item -127
                    \item 81
                    \item 127
                    \item -81
                \end{enumerate}
            \end{multicols}
        
        }
    \item{
            The integral $\int_{1}^{2}e^{x}.x^{x}\brak{2+log_{e}x}dx$ equals
           	\hfill
                {[Sep 2020]}
            
            \begin{multicols}{2}
				\begin{enumerate}
					\item $e\brak{4e-1}$
					\item $e\brak{4e+1}$
					\item $4e^2-1$
					\item $e\brak{2e-1}$
				\end{enumerate}
			\end{multicols}
        
        }
 	\item{
        	If the tangent to the curve, $y = f(x) = x\log_e x$, $\brak{x>0}$ at a point $\brak{c,f\brak{c}}$ is parallel to the line-segment joining the points $\brak{1,0}$ and $\brak{e,e}$, then $c$ is equal to: \\ \text{ }
        	\hfill
        	{[Sep 2020]}
        	
        	\begin{multicols}{4}
        		\begin{enumerate}
					\item \( e^{\brak{\frac{1}{1-e}}} \)
					\item \( \frac{\brak{e-1}}{e} \)
					\item \( \frac{1}{\brak{e-1}} \)
					\item \( e^{\brak{\frac{1}{e-1}}} \)
        		\end{enumerate}
        	\end{multicols}
        	
        }
 	\item{
			If $y = \brak{\frac{2}{\pi}x-1} \csc x$ is the solution of the differential equation, $\brak{\frac{dy}{dx}}+p\brak{x}y = \brak{\frac{2}{\pi}-1}\csc x$, $0<x<\frac{\pi}{2}$, then the function $p\brak{x}$ is equal to: 
			\hfill
			{[Sep 2020]}
			
			\begin{multicols}{4}
				\begin{enumerate}
					\item $\cosec x$
					\item $\cot x$
					\item $\tan x$
					\item $\sec x$
				\end{enumerate}
			\end{multicols}
			
		}
 	\item{
			If $\alpha$ and $\beta$ are the roots of the equation $2x\brak{2x+1} = 1$, then $\beta$ is equal to:\\ \text{ }
			\hfill
			{[Sep 2020]}
			
			\begin{multicols}{2}
				\begin{enumerate}
					\item $2\alpha\brak{\alpha-1}$
					\item $-2\alpha\brak{\alpha+1}$
					\item $2\alpha^2$
					\item $2\alpha\brak{\alpha+1}$
				\end{enumerate}
			\end{multicols}
			
		}
    \item{
            For all twice differentiable functions \( f: \mathbb{R} \to \mathbb{R} \), with \( f\brak{0} = f\brak{1} = f^{\prime}\brak{0} = 0 \), \\ \text{ }
             \hfill
                {[Sep 2020]}
            \begin{multicols}{2}
                \begin{enumerate}
                    \item $f^{\prime\prime}\brak{x} = 0$, at every point $x \in \brak{0,1}$
                    \item $f^{\prime\prime}\brak{x} \neq 0$, at every point $x \in \brak{0,1}$
                    \item $f^{\prime\prime}\brak{x} = 0$, for some $x \in \brak{0,1}$
                    \item $f^{\prime\prime}\brak{0} = 0$
                \end{enumerate}
            \end{multicols}

        %code by ysiddhanth 
        
        }
    \item{
            The area (in sq.units) of the region enclosed by the curves $y = x^2-1$ and $y = 1-x^2$ is equal to:
             \hfill
                {[Sep 2020]}
            \begin{multicols}{4}
                \begin{enumerate}
                    \item $\frac{4}{3}$
                   	\item $\frac{7}{2}$
                   	\item $\frac{16}{3}$
                    \item $\frac{8}{3}$
                \end{enumerate}
            \end{multicols}
        
        }
    \item{
            For a suitably chosen real constant $a$, let a function, $f: \mathbb{R} - \cbrak{-a} \rightarrow \mathbb{R}$ be defined by $f\brak{x} = \frac{a-x}{a+x}$. Further suppose that for any real number $x \neq -a$ and $f\brak{x} \neq -a$, $\brak{f \circ f}\brak{x} = x$. Then $f\brak{\frac{-1}{2}}$ is equal to:
             \hfill
                {[Sep 2020]}
			\begin{multicols}{4}
				\begin{enumerate}
					\item -3
					\item 3
					\item $\frac{1}{3}$
					\item $-\frac{1}{3}$
				\end{enumerate}
			\end{multicols}
        
        }
    \item{
        
            Let $\theta = \frac{\pi}{5}$ and $A = \myvec{ \cos\theta & -\sin\theta \\ \sin\theta & \cos\theta}$. If $B = A + A^4$, then $\det(B)$:
             \text{   }\hfill
                {[Sep 2020]}
            \begin{multicols}{2}
                \begin{enumerate}
                   	\item is one
                    \item lies in \brak{1,2}
                    \item lies in \brak{2,3}
                    \item is zero
                \end{enumerate}
            \end{multicols}
        
        }
    \item{
	
			The center of the circle passing through the point $\brak{0,1}$ and touching the parabola $y=x^2$ at the point $\brak{2,4}$ is :
			\text{   }\hfill
			{[Sep 2020]}
			\begin{multicols}{4}
				\begin{enumerate}
					\item $\brak{\frac{3}{10}, \frac{16}{5}}$
					\item $\brak{\frac{6}{5}, \frac{53}{10}}$
					\item $\brak{-\frac{16}{5}, \frac{53}{10}}$
					\item $\brak{-\frac{53}{10}, \frac{16}{5}}$
				\end{enumerate}
			\end{multicols}
			
		}
    \item{
	
			A plane $P$ meets the coordinate axes at $A$, $B$ and $C$ respectively. The centroid of a triangle $ABC$ is given to be $\brak{1,1,2}$. Then the equation of the line through this centroid and perpendicular to the plane $P$ is:
			\text{   }\hfill
			{[Sep 2020]}
			\begin{multicols}{2}
				\begin{enumerate}
					\item $\frac{x-1}{2} = \frac{y-1}{1} = \frac{z-2}{1}$
					
					\item $\frac{x-1}{2} = \frac{y-1}{2} = \frac{z-2}{1}$
					
					\item $\frac{x-1}{1} = \frac{y-1}{2} = \frac{z-2}{2}$
					
					\item $\frac{x-1}{1} = \frac{y-1}{1} = \frac{z-2}{2}$
				\end{enumerate}
			\end{multicols}
			
		}
    \item{
        
            Let $f : \mathbb{R}\rightarrow \mathbb{R}$ be a function defined by $f(x) = \max \cbrak{x,x^2}$. Let $S$ denote the set of all points in $\mathbb{R}$, where $f$ is not differentiable. Then
             \hfill
              {[Sep 2020]}
              \begin{multicols}{2}
              	\begin{enumerate}
              		\item $\cbrak{0,1}$
              		\item an empty set
              		\item $\cbrak{1}$
              		\item $\cbrak{0}$
              	\end{enumerate}
              \end{multicols}
        
        }


