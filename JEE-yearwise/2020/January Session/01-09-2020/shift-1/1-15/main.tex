\iffalse
\title{2020}   
\author{AI24Btech11024}
\section{mcq-single}
\fi
\item If C be the centroid of the triangle having vertices $\brak{3,-1},\brak{1,3}$ and $\brak{2,4}$. Let P be the point of intersection of the lines $x+3y-1=0$ and $3x-y+1=0$,then the line passing through the point$\colon$ \hfill{\brak{\text{Jan 2020}}}
    \begin{enumerate}
        \item \brak{-9,-7}
        \item \brak{-9,-6}
        \item \brak{7,6}
        \item \brak{9,7}
    \end{enumerate}
\item The product $2^{\frac{1}{4}}\times4^{\frac{1}{16}}\times8^{\frac{1}{48}}\times16^{\frac{1}{128}}\cdots\infty$ to is equal to \hfill{\brak{\text{Jan 2020}}}
    \begin{enumerate}
        \item $2^{\frac{1}{4}}$
        \item 2
        \item $2^{\frac{1}{2}}$
        \item 1
    \end{enumerate}
\item A spherical iron ball of 10 cm radius is coated with a layer of ice of uniform thickness that melts at the rate of $50 cm^{3}/min$.When the thickness of ice is 5cm, then the rate$\brak{cm/min.}$ at which the thickness of ice decreases, is$\colon$ \hfill{\brak{\text{Jan 2020}}}
    \begin{enumerate}
        \item $\frac{5}{6\pi}$
        \item $\frac{1}{54\pi}$
        \item $\frac{1}{36\pi}$
        \item $\frac{1}{18\pi}$
    \end{enumerate}
\item Let f be any function continuous on $\sbrak{a,b}$ and twice differentiable on $\brak{a,b}$. If for all $x\in\brak{a,b},f^{\prime}>0$ and $f^{\prime\prime}<0$, then for any $c\in\brak{a,b}$, $\brak{f\brak{c}-f\brak{a}}-\brak{f\brak{b}-f\brak{c}}$ is greater than$\colon$ \hfill{\brak{\text{Jan 2020}}}
    \begin{enumerate}
        \item $\brak{b-c}\slash\brak{c-a}$
        \item 1
        \item $\brak{c-a}\slash\brak{b-c}$
        \item $\brak{b+a}\slash\brak{b-a}$
    \end{enumerate} 
\item The value of $ \cos^3\left(\frac{\pi}{8}\right) \cos\left(\frac{3\pi}{8}\right) + \sin^3\left(\frac{\pi}{8}\right) \sin\left(\frac{3\pi}{8}\right) $ is: \hfill{\brak{\text{Jan 2020}}}
   \begin{enumerate}
       \item \( \frac{1}{4} \)
      \item \( \frac{1}{2\sqrt{2}} \)
      \item \( \frac{1}{2} \)
      \item \( \frac{1}{\sqrt{2}} \)
   \end{enumerate}
\item The number of real roots of the equation, $e^{4x}+e^{3x}-4e^{2x}+e^{x}+1=0$ is \hfill{\brak{\text{Jan 2020}}}
   \begin{enumerate}
       \item 3
       \item 4
       \item 1
       \item 2
   \end{enumerate}
\item The value of  $\int_0^{2\pi} \frac{x \sin^8 {x}}{\sin^8 {x} + \cos^8 {x}} \, dx
   $ is equal to \hfill{\brak{\text{Jan 2020}}}
   \begin{enumerate}
       \item 2
       \item 4
       \item $2^{2}$
       \item $\pi^{2}$
   \end{enumerate}
\item If for some $\alpha$ and $\beta$ in $R$,the intersection of the following three planes\\$x+4y-2z=1$\\$x+7y-5z=\beta$\\$x+5y+\alpha z=5$\\is a line in $R^{3}$, then $\alpha+\beta$ is equal to$\colon$ \hfill{\brak{\text{Jan 2020}}}
   \begin{enumerate}
       \item 0
       \item 10
       \item -10
       \item 2
   \end{enumerate}
\item If $e_{1}$ and $e_{2}$ are the eccentricities of the ellipse, $\brak{\frac{x^{2}}{18}}+\brak{\frac{y^{2}}{4}}=1$ and the hyperbola, $\brak{\frac{x^{2}}{9}}-\brak{\frac{y^{2}}{4}}=1$ respectively and $\brak{e_{1},e_{2}}$
  is a point on the ellipse,$15x^{2}+3y^{2}=k$.Then k is equal to$\colon$ \hfill{\brak{\text{Jan 2020}}}
   \begin{enumerate}
       \item 14
       \item 15
       \item 17
       \item 18
   \end{enumerate}
\item If f(x) =
$\begin{cases}
\frac{\sin(a+2)x + \sin x}{x}&,x < 0 \\
b&,x = 0 \\
\frac{(x + 3x^{\frac{2}{3}} - x^{\frac{1}{3}})}{\frac{4}{x^3}}&,x > 0
\end{cases}$\\is continuous at $x=0$ then $a+2b$ is equal to$\colon$ \hfill{\brak{\text{Jan 2020}}}
\begin{enumerate}
    \item -2
    \item 1
    \item 0
    \item 1
\end{enumerate}
\item If the matrices\\
$A=\begin{bmatrix}
1&1&2\\1&3&4\\1&-1&3\end{bmatrix},$ then $B=adj\,A$ and $C=3A$, then\\$\frac{\lvert adj\,B \rvert}{\lvert C \rvert}$ is equal to \hfill{\brak{\text{Jan 2020}}}
\begin{enumerate}
    \item 16
    \item 2
    \item 8
    \item 72
\end{enumerate}
\item A circle touches the y-axis at the point $\brak{0,4}$ and passes through the point $\brak{2,0}$. Which of the following lines is not a tangent to the circle? \hfill{\brak{\text{Jan 2020}}}
\begin{enumerate}
    \item $4x-3y+17=0$
    \item $3x+4y-6=0$
    \item $4x+3y-8=0$
    \item $3x-4y-24=0$
\end{enumerate}
\item Let $Z$ be a complex number such that $\left\lvert\frac{z-i}{z+2i}\right\rvert=1$ and $\lvert z \rvert=\frac{5}{2}$.Then the value of $\lvert z+3i\rvert$ is$\colon$ \hfill{\brak{\text{Jan 2020}}}
\begin{enumerate}
    \item $\sqrt{10}$
    \item $\frac{7}{2}$
    \item $\frac{15}{4}$
    \item $2\sqrt{3}$
\end{enumerate}
\item If $f^{\prime}\brak{x}=\tan^{-1}\brak{\sec x + \tan x},\frac{-\pi}{2}<x<\frac{\pi}{2},$ and $f\brak{0}=0$, then $f\brak{1}$ is equal to$\colon$ \hfill{\brak{\text{Jan 2020}}}
\begin{enumerate}
    \item $\frac{\pi+1}{4}$
    \item $\frac{\pi+2}{4}$
    \item $\frac{1}{4}$
    \item $\frac{\pi-1}{4}$
\end{enumerate}
\item Negation of the statement$\colon$ $\sqrt{5}$ is an integer or 5 is irrational is$\colon$ \hfill{\brak{\text{Jan 2020}}}
\begin{enumerate}
    \item $\sqrt{5}$ is irrational or 5 is an integer.
    \item $\sqrt{5}$ is not an integer or 5 is not irrational
    \item $\sqrt{5}$ is an integer and 5 is irrational
    \item $\sqrt{5}$ is not an integer and 5 is not irrational
\end{enumerate}
