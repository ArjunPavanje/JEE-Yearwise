\iffalse
\title{2023}
\author{EE24BTECH11008}
\section{mcq-single}
\fi
%\begin{enumerate}
    \item If $\frac{1}{n+1}^nC_n+\frac{1}{n}^nC_{n-1}+\cdot+\frac{1}{2}^nC_1+^nC_0=\frac{1023}{10}$ then $n$ is equal to $:$
	    \hfill{\brak{2023-Apr}}\\
    \begin{enumerate}
        \item $6$
        \item $9$
        \item $8$
        \item $7$
    \end{enumerate}
    \item Let $\textbf{C}$ be the circle in the complex plane with centre $z_0=\frac{1}{2}\brak{1+3i}$ and radius $r=1.$ Let $z_1=1+i$ and the complex number $z_2$ be outside the circle $\textbf{C}$ such that $\abs{z_1-z_0}\abs{z_2-z_0}=1.$ If $z_0,z_1$ and $z_2$ are collinear, then the smaller value of $\abs{z_2}^2$ is equal to $:$
	   \hfill{\brak{2023-Apr}} \\
    \begin{enumerate}
        \item $\frac{13}{2}$
        \item $\frac{5}{2}$
        \item $\frac{3}{2}$
        \item $\frac{7}{2}$
    \end{enumerate}
    \item If the point $\brak{\alpha,\frac{7\sqrt{3}}{3}}$ lies on the curve traced by the mid-points of the line segments of the lines $x\cos\theta+y\sin\theta=\gamma,$ $\theta\in \brak{0,\frac{\pi}{2}}$ between the co-ordinates axes, then $\alpha$ is equal to $:$
	   \hfill{\brak{2023-Apr}} \\
    \begin{enumerate}
        \item $7$
        \item $-7$
        \item $-7\sqrt{3}$
        \item $7\sqrt{3}$
    \end{enumerate}
    \item Two dice $A$ and $B$ are rolled. Let the numbers obtained on $A$ and $B$ be $\alpha$ and $\beta$ respectively. If the varience of $\alpha-\beta$ is $\frac{p}{q},$ where $p$ and $q$ are co-prime, then the sum of the positive divisors of $p$ is equal to $:$
	    \hfill{\brak{2023-Apr}}\\
    \begin{enumerate}
        \item $36$
        \item $48$
        \item $31$
        \item $72$
    \end{enumerate}
    \item In a triangle $ABC$ if $\cos A+2\cos B+\cos C=2$ and the lengths of the sides opposite to the angles $A$ and $C$ are $3$ and $7$ respectively, then $\cos A-\cos C$ is equal to $:$
	    \hfill{\brak{2023-Apr}}\\
    \begin{enumerate}
        \item $\frac{3}{7}$
        \item $\frac{9}{7}$
        \item $\frac{10}{7}$
        \item $\frac{5}{7}$
    \end{enumerate}
   %\end{enumerate}
