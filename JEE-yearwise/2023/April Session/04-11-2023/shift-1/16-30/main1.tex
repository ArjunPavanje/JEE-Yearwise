\iffalse
  \title{August:2021}
  \author{AI24BTECH11016}
  \section{fitb}
\fi
\item 
	The number of integral terms in the expansion of $\brak{3^{\frac{1}{2}} + 5^{\frac{1}{4}}}$ is:
	\hfill [Apr 2023]

	\item 
	The number of ordered triplets of the truth values of $p$, $q$, and $r$ such that the truth value of the statement $\brak{p \lor q} \land \brak{p \lor r} \implies \brak{q \lor r}$ is true, is equal to:
	\hfill [Apr 2023]

	\item
	Let $A = \myvec{0 & 1 & 2 \\ a & 0 & 3 \\ 1 & c & 0}$, where $a, c \in \mathbb{R}$. If $A^{3} = A$ and the positive value of $a$ belongs to the interval $(n-1,n]$ where $n \in \mathbb{N}$, then $n$ is equal to:
	\hfill [Apr 2023]
	
	\item 
	For $m, n > 0$, let $\alpha\brak{m, n} = \int_{0}^{2} t^{m}\brak{1+3t}^{n} dt$. If $11\alpha\brak{10, 6} + 18\alpha\brak{11, 5} = p 14^{6}$ , then $p$ is equal to:
	
	\hfill [Apr 2023]
	
	\item 
	Let $S = 109 + \frac{108}{5} + \frac{107}{5^2} + \dots + \frac{2}{5^{107}} + \frac{1}{5^{108}}$. Then the value of $16S - \brak{25^{-54}}$ is equal to:
	
	\hfill [Apr 2023]
	
	\item 
	Let $H_n:\frac{x^2}{1+n} - \frac{y^2}{3+n} = 1$, $n \in \mathbb{N}$. Let $k$ be the smallest even value of $n$ such that the eccentricity of $H_k$ is a rational number. If $l$ is the length of the latus rectum of $H_k$, then $21l$ is equal to:
	
	\hfill [Apr 2023]
	
	\item
	The mean of the coefficients of $x$, $x^2$,..., $x^7$ in the binomial expansion of $\brak{2 + x}^{9}$ is:
	
	\hfill [Apr 2023]
	
	\item
	If $a$ and $b$ are the roots of the equation $x^{2} - 7x - 1 = 0$, then the value of $\frac{a^{21} + b^{21} + a^{17} + b^{17}}{a^{19} + b^{19}}$ is equal to:
	
	\hfill [Apr 2023]
	
	\item
	In an examination, 5 students have been allotted their seats as per their roll numbers. The number of ways in which none of the students sits on the allotted seat is:
	
	\hfill [Apr 2023]
	
	\item
Let a line $l$ pass through the origin and be perpendicular to the lines \\
	$l_1 : \vec{r} = \hat{i} - 11\hat{j} - 7\hat{k} + \lambda\brak{\hat{i} +2\hat{j} + 3\hat{k}}, \lambda \in \mathbb{R}$ and \\
	$l_2 : \vec{r} = -\hat{i} + \hat{k} + \mu\brak{2\hat{i} +2\hat{j} + \hat{k}}, \mu \in \mathbb{R}$
	If $\vec{P}$ is the point of intersection of $l$ and $l_1$, and $\vec{Q} \myvec{\alpha & \beta & \gamma}$ is the foot of perpendicular from $\vec{P}$ on $l_2$, then $9\brak{\alpha+\beta+\gamma}$ is equal to
	\hfill [Apr 2023]
