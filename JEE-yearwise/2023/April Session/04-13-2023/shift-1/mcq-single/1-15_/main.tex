\iffalse
\title{2023}
\author{AI24BTECH11013}
\section{mcq-single}
\fi
\item $\int_0^{\infty} \frac{6}{e^{3x} + 6e^{2x} + 11e^{x} + 6} \, dx$
\hfill{\brak{April 2023}}
\begin{enumerate}
    \item $\log_e(\frac{32}{27})$
    \item $\log_e(\frac{256}{81})$
    \item $\log_e(\frac{512}{81})$
    \item $\log_e(\frac{64}{27})$
\end{enumerate}
\item Among\\
$ \brak{S1} : \lim_{n \to \infty} \frac{1}{n^2} \brak{2+4+6+....+2n} = 1$\\
$\brak{S2} : \lim_{n \to \infty} \frac{1}{n^16} \brak{1^15 +2^15 +3^15 +.....+ n^15} = \frac{1}{16}$
\hfill{\brak{April 2023}}
\begin{enumerate}
    \item Only $\brak{S1}$ is true
    \item Both $\brak{S1}$ and $\brak{S2}$ are true 
    \item Both $\brak{S1}$ and $\brak{S2}$ are false
    \item Only $\brak{S2}$ is true
\end{enumerate}
\item The number of symmetric matrices of order 3, with all the entries from the set {0, 1, 2, 3, 4, 5, 6, 7, 8, 9}, is
\begin{enumerate}
    \item $10^9$
    \item $10^6$
    \item $9^{10}$
    \item $6^{10}$
\end{enumerate}
\item Let $\bar{a} = \hat{i} + 4\hat{j} +2\hat{k}$, $\bar{b} = 3\hat{i}-2\hat{j}+7\hat{k}$ and $\bar{c} = 2\hat{i}-\hat{j}+4\hat{k}$. If a vector $\bar{d}$ satisfies $\bar{d}*\bar{b} = \bar{c}*\bar{b}$ and $\bar{d}.\bar{a}=24$, then $|\bar{d}|^2$ is equal to 
\hfill{\brak{April 2023}}
\begin{enumerate}
    \item 323
    \item 423
    \item 413
    \item 313
\end{enumerate}
\item A coin is biased so that the head is 3 times as likely to occur as tail. This coin ias tossed until a head or three tails occur. If X denotes the number of tosses of the coin, then the mean of X is 
\hfill{\brak{April 2023}}
\begin{enumerate}
    \item $\frac{21}{16}$
    \item $\frac{15}{16}$
    \item $\frac{81}{64}$
    \item $\frac{37}{16}$
\end{enumerate}
\item $max_{0 \leq x\leq \pi}{x-2sinxcosx+ \frac{1}{3}sin3x}=$
\hfill{\brak{April 2023}}
\begin{enumerate}
    \item 0
    \item $\pi$
    \item $\frac{5\pi+2+3\sqrt{3}}{6}$
    \item $\frac{\pi+2-3\sqrt{3}}{6}$
\end{enumerate}
\item The set of all $a \in R$ for which the equation $x|x-1| + |x+2| = a =0$ has exactly one real root, is
\hfill{\brak{April 2023}}
\begin{enumerate}
    \item $\brak{-\infty,-3}$
    \item $\brak{-\infty,\infty}$
    \item $\brak{-6,\infty}$
    \item $\brak{-6,-3}$
\end{enumerate}
\item Let PQ be a focal chord of the parabola $y^2 = 36x$ of length 100, making an acute with the position x-axis. Let the ordinate of P be positive and M be the point on the line segment PQ such that $PM:MQ=3:1$. Then wich of the following points does NOT lie on the line passing through M and perpendicular to the line PQ? 
\hfill{\brak{April 2023}}
\begin{enumerate}
    \item $\brak{3,33}$
    \item $\brak{6,29}$
    \item $\brak{-6,45}$
    \item $\brak{-3,43}$
\end{enumerate}
\item For the system of linear equations\\
$2x + 4y + 2az = b$\\
$x +  2y + 3z = 4$\\
$2x - 5y +2z = 8$\\
which of the following is NOT correct?
\hfill{\brak{April 2023}}
\begin{enumerate}
    \item It has infinitely many solutions if $a=3,b=8$
    \item It has unique solution if $a=b=8$
    \item It has unique solution if $a=b=6$
    \item It has infinitely many solutions if $a=3,b=6$
\end{enumerate}
\item Let $s_1,s_2,s_3,....s_{10}$ respectively be the sum of 12 terms of 10 A.P.s whose first terms are 1, 2,v 3, ..., 10 and the common differencce arev 1, 3, 5,....., 19 respectively. Then $\sum_{i=1}^{10} s_i$ is equal to 
\hfill{\brak{April 2023}}
\begin{enumerate}
    \item 7260
    \item 7380
    \item 7220
    \item 7360
\end{enumerate}
\item For the differentiable function $f:R-{0}\rightarrow R$, let $3f\brak{x}+2f\brak{\frac{1}{x}}=\frac{1}{x}-10$, then $|f\brak{3}+f^{\prime}\brak{\frac{1}{4}}|$ is equal to
\hfill{\brak{April 2023}}
\begin{enumerate}
    \item 13
    \item $\frac{29}{5}$
    \item $\frac{33}{5}$
    \item 7
\end{enumerate}
\item The negation of the statement $\brak{\brak{A \land \brak{B \lor C}} \Rightarrow \brak{A \lor B}} \Rightarrow A $ is
\hfill{\brak{April 2023}}
\begin{enumerate}
    \item equivalent to $ B \lor \sim C $
    \item a fallacy
    \item equivalent to $ \sim C $
    \item equivalent to $ sim A $
\end{enumerate}
\item Let the tangent and normal at the point $\brak{3\sqrt{3},1}$ on the ellipse $\frac{x^2}{36}+\frac{y^2}{4}=1$ meet the y-axis at the points A and B respectively. Let the circle C be drawn taking AB as a diameter and the line $x=2\sqrt{5}$ intersect C at the points P and Q. If the tangents at the points P and Q on the circle intersect at the point $\brak{\alpha,\beta}$, then $\alpha^2 -\beta^2$ is equal to 
\hfill{\brak{April 2023}}
\begin{enumerate}
    \item $\frac{304}{4}$
    \item 60
    \item $\frac{314}{5}$
    \item 61
\end{enumerate}
\item The distance of the points \brak{-1,2,3} from the plane $\bar{r}.\brak{\hat{i}-2\hat{j}+3\hat{k}}=10$ parallel to the line of the shortest distance between the lines $\bar{r}=\brak{\hat{i}-\hat{j}}+\lambda \brak{2\hat{i}+\hat{k}}$ and $\bar{r}=\brak{2\hat{i}-\hat{j}+\mu \brak{\hat{i}-\hat{j}+\hat{k}}}$ is
\hfill{\brak{April 2023}}
\begin{enumerate}
    \item $2\sqrt{5}$
    \item $3\sqrt{5}$
    \item $3\sqrt{6}$
    \item $2\sqrt{6}$
\end{enumerate}
\item Let $B = 
\begin{bmatrix}
1 & 3 & \alpha \\
1 & 2 & 3 \\
\alpha & \alpha & 4
\end{bmatrix}$, $\alpha>2$ be the adjoint of matrix A and $|A|=2$, then $\begin{bmatrix}
    \alpha & -2\alpha & \alpha 
    \end{bmatrix}$ B 
    $\begin{bmatrix}
        \alpha\\
        -2\alpha\\
        \alpha
    \end{bmatrix}$ is equal to
\hfill{\brak{April 2023}}
\begin{enumerate}
    \item 16
    \item 32
    \item 0
    \item -16
\end{enumerate}\
