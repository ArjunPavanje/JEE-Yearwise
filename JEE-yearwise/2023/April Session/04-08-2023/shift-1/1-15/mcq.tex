\iffalse
\title{2023}
\author{EE24BTECH11006}
\section{mcq-single}
\fi

\item{
The area of the region $\{\brak{x,y}:x^2\leq y \leq 8-x^2,y \leq 7\}$
\hfill{$\brak{\text{Apr-2023}}$}
\begin{multicols}{4}
\begin{enumerate}
\item$24$
\item$21$
\item$20$
\item$18$
\end{enumerate}
\end{multicols}
}
\item{
Let $P=\myvec{\frac{\sqrt{3}}{2}&\frac{1}{2}\\\frac{-1}{2}&\frac{\sqrt{3}}{2}}$ , $A=\myvec{1&1\\0&1}$ and $Q=PAP^T$. If $P^TQ^{2007}P=\myvec{a&b\\c&d}$ , then $2a+b-3c-4d$ equal to
\hfill{$\brak{\text{Apr-2023}}$}
\begin{multicols}{4}
\begin{enumerate}
\item $2004$
\item $2007$
\item $2005$
\item $2006$
\end{enumerate}
\end{multicols}
}
\item{
Negation of $\brak{p\rightarrow q}\rightarrow\brak{q\rightarrow q}$ is 
\hfill{$\brak{\text{Apr-2023}}$}
\begin{multicols}{4}
\begin{enumerate}
\item $\brak{\neg q}\wedge p$
\item $p\vee\brak{\neg q}$
\item $\brak{\neg p}\vee q$
\item $q\wedge\brak{\neg p}$
\end{enumerate}
\end{multicols}
}
\item{
Let $C\brak{\alpha,\beta}$ be the circumcenter of the triangle formed by the lines\\
\hfill{$\brak{\text{Apr-2023}}$}
\begin{align*}
4x+3y&=69\\
4y-3x&=17\\
x+7y&=61
\end{align*}
Then $\brak{\alpha-\beta}^2+\alpha+\beta$ is equal to
\begin{multicols}{4}
\begin{enumerate}
\item $18$
\item $15$
\item $16$
\item $17$
\end{enumerate}
\end{multicols}
}
\item{
Let $\alpha,\beta,\gamma$, be the three roots of the equation $x^3+bx+c=0$. If $\beta y=1=-\alpha$, then $b^3+2c^3-3\alpha^3-6\beta^3-8\gamma^3$ is equal to
\hfill{$\brak{\text{Apr-2023}}$}
\begin{multicols}{4}
\begin{enumerate}
\item $\frac{155}{8}$
\item $21$
\item $19$
\item $\frac{169}{8}$
\end{enumerate}
\end{multicols}
}
\item{
Let the number of elements in set $A$ and $B$ be five and two respectively. Then the number of subsets of $A\times B$ each having at least $3$ and at most $6$ elements is:
\hfill{$\brak{\text{Apr-2023}}$}
\begin{multicols}{4}
\begin{enumerate}
\item $752$
\item $772$
\item $782$
\item $792$
\end{enumerate}
\end{multicols}
}
\item{
If the coefficients of three consecutive terms in the expansion of $\brak{1+x}^n$ are in the ratio $1:5:20$, then the coefficient of the fourth term is 
\hfill{$\brak{\text{Apr-2023}}$}
\begin{multicols}{4}
\begin{enumerate}
\item $5481$
\item $3654$
\item $2436$
\item $1817$
\end{enumerate}
\end{multicols}

}
\item{
Let $R$ be the focus of the parabola $y^2=20x$ and the line $y=mx+c$ intersect the parabola at two points $P$ and $Q$. Let the point $G\brak{10,10}$ be the centroid of the triangle $PQR$. If $c-m=6$ then $\brak{PQ}^2$ is
\hfill{$\brak{\text{Apr-2023}}$}
\begin{multicols}{4}
\begin{enumerate}
\item $325$
\item $346$
\item $296$
\item $317$
\end{enumerate}
\end{multicols}
}
\item{
Let $S_K=\frac{1+2+_{...}+K}{K}$ and $\sum\limits_{j=1}^n S_j^2=\frac{n}{A}\brak{Bn^2+Cn+D}$ where $A,B,C,D\in N$ and $A$ has least value. Then
\hfill{$\brak{\text{Apr-2023}}$}
\begin{multicols}{2}
\begin{enumerate}
\item $A+B$ is divisible by $D$
\item $A+B=5\brak{D-C}$
\item $A+C+D$ is not divisible by $B$
\item $A+B+D$ is divisible by $5$
\end{enumerate}
\end{multicols}
}
\item{
The shortest distance between the lines $\frac{x-4}{4}=\frac{y+2}{5}=\frac{z+3}{3}$ and $\frac{x-1}{3}=\frac{y-3}{4}=\frac{z-4}{2}$
\hfill{$\brak{\text{Apr-2023}}$}
\begin{multicols}{4}
\begin{enumerate}
\item $2\sqrt{6}$
\item $3\sqrt{6}$
\item $6\sqrt{3}$
\item $6\sqrt{2}$
\end{enumerate}
\end{multicols}
}
\item{
The number of arrangements of the letters of the word $"INDEPENDENCE"$ in which all the vowels always occur together is
\hfill{$\brak{\text{Apr-2023}}$}
\begin{multicols}{4}
\begin{enumerate}
\item $16800$
\item $14800$
\item $18000$
\item $33600$
\end{enumerate}
\end{multicols}
}
\item{
If the points with position vectors $\alpha\hat{i}+10\hat{j}+13\hat{k}$, $6\hat{i}+11\hat{j}+11\hat{k}$, $\frac{9}{2}\hat{i}+\beta\hat{j}-8\hat{k}$ are collinear then, $\brak{19\alpha-6\beta}^2$ is equal to
\hfill{$\brak{\text{Apr-2023}}$}
\begin{multicols}{4}
\begin{enumerate}
\item $49$
\item $36$
\item $25$
\item $16$
\end{enumerate}
\end{multicols}
}
\item{
In a bolt factory, machines $A$,$B$, and $C$ manufacture respectively $20\%, 30\%,$ and $50\%$ of the total bolts. Of their output $3$,$4$, and $2$ percent are respectively defective bolts. A bolt is drawn at random from the product. If the bolt drawn is found to be defective, then the probabilty that it is manufactured by the machine $C$.
\hfill{$\brak{\text{Apr-2023}}$}
\begin{multicols}{4}
\begin{enumerate}
\item $\frac{5}{14}$
\item $\frac{3}{7}$
\item $\frac{9}{28}$
\item $\frac{2}{7}$
\end{enumerate}
\end{multicols}
}
\item{
If for $z=\alpha+\imath\beta$  ,$\abs{z+2}=z+\brak{4+\imath}$, then $\alpha+\beta$ and $\alpha\beta$ are the roots of the equation
\hfill{$\brak{\text{Apr-2023}}$}
\begin{multicols}{4}
\begin{enumerate}
\item $x^2+3x-4$
\item $x^2+7x+12$
\item $x^2+x-12$
\item $x^2+2x-3$
\end{enumerate}
\end{multicols}
}
\item{
$\lim_{x \to 0}\brak{\brak{\frac{\brak{1-\cos^2\brak{3x}}}{\cos^3\brak{4x}}}\brak{\frac{sin^3\brak{4x}}{\log_e\brak{2x+1}^5}}}$ is equal to
\hfill{$\brak{\text{Apr-2023}}$}
\begin{multicols}{4}
\begin{enumerate}
\item $24$
\item $19$
\item $18$
\item $15$
\end{enumerate}
\end{multicols}
}

