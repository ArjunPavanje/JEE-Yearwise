\iffalse
  \title{Assignment}
  \author{EE24BTECH11026}
  \section{mcq-single}
\fi

%   \begin{enumerate}
    \item 
        Let $A= \cbrak{\theta \in \brak{0,2\pi}:\frac{1+2\iota\sin{\theta}}{1-\iota\sin{\theta}} \text{ is purely imaginary }}$.Then sum of elements in A is. 
        \hfill{\brak{\textnormal{2023-April}}}
               \begin{enumerate}
               \begin{multicols}{4}
                \item $\pi$
                \item $3\pi$                 
                \item $4\pi$               
                \item $2\pi$
                \end{multicols}
            \end{enumerate}
            
	\item Let P be the plane passing through the line $\frac{x-1}{1}= \frac{y-2}{-3}= \frac{z+5}{7}$ and the point $\vec (2,4,3)$.If the image of the point $\vec(-1,3,4)$ in the plane P is $\vec(\alpha,\beta,\gamma)$then $\alpha+\beta+\gamma$is equal to
        \hfill{\brak{\textnormal{2023-April}}}
                \begin{enumerate}
                \begin{multicols}{4}
                \item $12$
                \item $9$
                \item $10$ 
                \item $11$
                \end{multicols}
            \end{enumerate} 
            
	\item If $A=\myvec{1 & 5 \\ \lambda & 10}$, $A^{-1}=\alpha A+\beta I$ and $\alpha+\beta=-2$,then $4\alpha^2+\beta^2+\lambda^2$ is equal to
        \hfill{\brak{\textnormal{2023-April}}}
        \begin{enumerate}    
        \begin{multicols}{4}
                \item $14$
                \item $12$
                \item $19$
                \item $10$
                \end{multicols}
       \end{enumerate}
       
	\item 
	The area of the quadrilateral $ABCD$ with the vertices $\vec{A}\brak{2, 1, 1}$, $\vec{B}\brak{1, 2, 5}$, $\vec{C}\brak{-2, -3, 5}$ and $\vec{D}\brak{1, -6, -7}$ is equal to 
	\hfill{\brak{\textnormal{2023-April}}}
	\begin{enumerate}    
        \begin{multicols}{4}
                \item $54$
                \item $9\sqrt{38}$
                \item $48$
                \item $8\sqrt{38}$
                \end{multicols}
       \end{enumerate}
       
       
    \item $25^{190}-19^{180}-8^{190}+2^{190}$is divisble by
        \hfill{\brak{\textnormal{2023-April}}}
        \begin{multicols}{4}
            \begin{enumerate}
                \item $34$ but not by 14
                \item $14$ but not by 34
                \item Both $14$ and $34$
                \item Neither $14$ nor $34$
            \end{enumerate}
        \end{multicols}
        
        \item Let O be the origin and $OP$ and $OQ$ be the tangents to the circle $x^2+y^2-6x+4y+8=0$ at the points $P$ and $Q$ on it.If the circumcircle of the triangle $OPQ$ passes through the point $\brak{\alpha,1/2}$  \hfill{\brak{\textnormal{2023-April}}}\\
       \begin{enumerate}    
        \begin{multicols}{4}
                \item $-1/2$
                \item $5/2$
                \item $1$
                \item $3/2$
                \end{multicols}
              \end{enumerate}   
       \item Let $a_n$ be the $n^{th}$ term of the series $5+8+14+23+35+50+.....$ and the $S_n =\sum a_k$.Then $S_{30}-a_{40}$ is equal to
        \hfill{\brak{\textnormal{2023-April}}}\\
      \begin{enumerate}    
        \begin{multicols}{4}
                \item $11260$
                \item $11280$
                \item $11290$
                \item $11310$
                \end{multicols}
              \end{enumerate}  
               \item If $\alpha>\beta>0$ are the roots of the equation $ax^2 + bx + 1 = 0$, and 
    \begin{align*}
        \lim_{x \to \frac{1}{\alpha}} \brak{\frac{1-\cos{\brak{x^2 + bx + a}}}{2\brak{1 - \alpha x}^2}}^{\frac{1}{2}} = \frac{1}{k}\brak{\frac{1}{\beta} - \frac{1}{\alpha}}
    \end{align*}
    then $k$ is equal to
        \hfill{\brak{\textnormal{2023-April}}}\\
        \begin{enumerate}    
        \begin{multicols}{4}
                \item $\beta$
                \item $2\alpha$
                \item $2\beta$
                \item $\alpha$
                \end{multicols}
              \end{enumerate}     
              \item  If the number of words, with or without meaning,which can be made using all the letters of the word MATHEMATICS in which C and S do not come together, is $(6!)k$,is equal to      
        \hfill{\brak{\textnormal{2023-April}}}\\
       \begin{enumerate}    
        \begin{multicols}{4}
                \item $1890$
                \item $945$
                \item $2835$
                \item $5670$
                \end{multicols}
              \end{enumerate}    

    \item Let $S$ be the set of all values of $\theta \in \sbrak{-\pi,\pi}$ for which the system of linear equations \begin{align}
        x+y+\sqrt{7}z=0 \\
         -x+\brak{\tan{\theta}}y+\sqrt{3}z=0 \\
          x+y+\brak{\tan{\theta}}z=0 
    \end{align}
    has non-trivial solution.Then $\frac{120}{\pi}\sum \theta$ is equal to
        \hfill{\brak{\textnormal{2023-April}}}\\
       \begin{enumerate}    
        \begin{multicols}{4}
                \item $20$
                \item $40$
                \item $30$
                \item $10$
                \end{multicols}
              \end{enumerate}    

    \item For $a,b \in Z$ and $\abs{a-b} \leq 10$,let the angle between the plane P:$ax+y-z=b$ and the line1:$x-1=a-y=z+1$ be $\cos^{-1}\brak{\frac{1}{3}}$.If the distance of the point $\vec(6,-6,4)$ from the plane P is $3\sqrt{6}$, the $a^4+b^2$ is equal to
        \hfill{\brak{\textnormal{2023-April}}}\\
       \begin{enumerate}    
        \begin{multicols}{4}
                \item $85$
                \item $48$
                \item $25$
                \item $32$
                \end{multicols}
              \end{enumerate}    

    \item Let the vectors $\overline{u_1}=\hat{i}+\hat{j}+a\hat{k}$,$\overline{u_2}=\hat{i}+b\hat{j}+\hat{k}$ and $\overline{u_3}=c\hat{i}+\hat{j}+\hat{k}$ be coplanar. If the vectors $\overline{v_1}=\brak{a+b}\hat{i}+c\hat{j}+c\hat{k}$,$\overline{v_2}=a\hat{i}+\brak{b+c}\hat{j}+a\hat{k}$ and $\overline{v_3}=b\hat{i}+b\hat{j}+\brak{c+a}\hat{k}$ are also coplanar , then $6\brak{a+b+c}$ is 
        \hfill{\brak{\textnormal{2023-April}}}\\
        \begin{enumerate}    
        \begin{multicols}{4}
                \item $4$
                \item $12$
                \item $6$
                \item $0$
                \end{multicols}
              \end{enumerate}    
    \item The absolute  difference of the coefficients of $x^{10}$ and $x^7$ in the expansion of $\brak{2x^2+\frac{1}{2x}}^{11}$
        \hfill{\brak{\textnormal{2023-April}}}\\
       \begin{enumerate}    
        \begin{multicols}{4}
                \item  $10^3-10$
                \item  $11^3-11$
                \item  $12^3-12$
                \item  $13^3-13$
                \end{multicols}
              \end{enumerate}    


    \item Let A=\cbrak{1,2,3,4,5,6,7}.Then the relation $R=\cbrak{\brak{x,y} \in A\times A :x+y=7}$ is 
        \hfill{\brak{\textnormal{2023-April}}}\\
        \begin{enumerate} 
                \item Symmetric but neither reflexive nor transitive
                \item Transitive but neither symmetric nor reflexive
                \item An equivalence relation
                \item Reflexive but neither symmetric nor transitive
              \end{enumerate}    

    \item If the probability that the random variable X takes values x is given by $P\brak{X=x}=k(x+1)3^{-x}$,$x=0,1,2,3,.....$, where $k$ is a constant, then $P\brak{X\geq 2}$ is equal to
    \hfill{\brak{\textnormal{2023-April}}}\\
        \begin{enumerate}    
        \begin{multicols}{4}
                \item  $\frac{7}{27}$
                \item  $\frac{11}{18}$
                \item  $\frac{7}{18}$
                \item  $\frac{20}{27}$
        \end{multicols}
        \end{enumerate} 
% \end{enumerate}





