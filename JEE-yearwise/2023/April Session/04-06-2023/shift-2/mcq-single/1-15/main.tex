\iffalse
\title{Assignment}
\author{AI24BTECH11020}
\section{mcq-single}
\fi

%begin{enumerate}
	\item If gcd $\brak{m, n} = 1$ and $1^2 - 2^2 + 3^2 - 4^2 + \ldots + \brak{2021}^2 - \brak{2022}^2 + \brak{2023}^2 = 1012 m^2n$ then $m^2 - n^2$ is equal to: \hfill \brak{Apr 2023}
    \begin{enumerate}
        \item $180$
        \item $220$
        \item $200$
        \item $240$
    \end{enumerate}
\item The area bounded by the curves $y = \abs{ x - 1 } + \abs{ x - 2} $ and $y = 3$ is equal to: \hfill\brak{Apr 2023}
    \begin{enumerate}
        \item $5$
        \item $4$
        \item $6$
        \item $3$
    \end{enumerate}                                   
\item For the system of equations \\                  
          $x+y+z=6$\\ $x+2y+ \alpha z=10 $ \\ $x+3y+5z=\beta $, which one of the following is NOT true : \hfill\brak{Apr 2023}                                  
    \begin{enumerate}                                                            
        \item System has a unique solution for $\alpha =3, \beta \neq 14.$                                   \item System has a unique solution for $\alpha = -3, \beta =14.$ 
        \item System has no solution for $\alpha =3, \beta =24.$ 
        \item System has infinitely many solutions for $\alpha =3, \beta =14.$                       
    \end{enumerate}                       
\item Among the statements : \\      
	\brak{S1}: $\brak{p \Rightarrow q} \vee \brak{\brak{\sim p}\wedge q}$ is a tautology\\
		\brak{S2}: $\brak{q \Rightarrow p} \Rightarrow \brak{\brak{\sim p}\wedge q}$ is a contradiction  \hfill\brak{Apr 2023}
    \begin{enumerate}
	    \item only \brak{S2} is True
	    \item only \brak{S1} is True
	    \item neither \brak{S1} and \brak{S2} is True
	    \item both \brak{S1} and \brak{S2} are True
    \end{enumerate}
    \item $\displaystyle \lim_{n \to \infty} \cbrak{ \brak{ 2^{\frac{1}{2}} -2^{\frac{1}{3}}} \brak{2^{\frac{1}{2}} - 2^{\frac{1}{5}} } \cdots \brak{2^{\frac{1}{2}} - 2^{\frac{1}{2n+1}}}}$ is equal to: \hfill\brak{Apr 2023}
    \begin{enumerate}
        \item $\frac{1}{\sqrt{2}}$
        \item $\sqrt{2}$
        \item $1$
        \item $0$
    \end{enumerate}
    \item Let P be a square matrix such that $P^2 = I-P$.For $\alpha , \beta, \gamma, \delta \in \mathbb{N}$, if  $P^{\alpha}+P^{\beta}=\gamma I-29 P$ and $P^{\alpha}-P^{\beta}=\delta I-13P$, then $\alpha +\beta+ \gamma-\delta$ is equal to: \hfill\brak{Apr 2023}
	    \begin{enumerate}
        \item $40$    
 \item $22$    
        \item $24$     
        \item $18$     
    \end{enumerate}    
   \item A plane $P$ contains the line of intersection of the plane $ \overrightarrow{r} ( \hat {i} + \hat{j} + \hat{k}) = 6$ and $\overrightarrow{r} (2\hat {i} + 3\hat{j} + 4\hat{k}) = -5 $. If  $P$ passes through the point  $\brak{0, 2, -2}$, then the square of distance of the point $\brak{12, 12, 18}$ from the plane $P$ is : \hfill\brak{Apr 2023}
\begin{enumerate}
        \item $620$                   
        \item $1240$          
        \item $310$     
        \item $155$     
    \end{enumerate}
\item Let $f\brak{x}$ be a function satisfying $f\brak{x}+f\brak{\pi -x}=\pi ^2,\forall x \in \mathbb{R}.$ Then $\int_0^{\pi} f(x) \sin x dx$ is equal to :   \hfill\brak{Apr 2023}
        \begin{enumerate}
        \item $\frac{\pi ^2}{2}$
        \item $\pi ^2$
        \item $2\pi ^2$
        \item $\frac{\pi ^2}{4}$
    \end{enumerate}
\item If the coefficients of $x^7$ in $\brak{ ax^2 + \frac{1}{2bx}}^{11}$ and $x^{-7}$ in $\brak{ax - \frac{1}{3bx^2}}^{11}$ are equal, then :    \hfill\brak{Apr 2023}
\begin{enumerate}
        \item $64 ab = 243$
        \item $32 ab = 729$
        \item $729 ab = 32$
        \item $243 ab = 64$
    \end{enumerate}
\item If the tangents at the points $\vec{P}$ and $\vec{Q}$ are the circle $ x^2 + y^2 -2x + y = 5 $ meet at the point $\vec{R} \brak{ \frac{9}{4},2} ,$ then the area of the triangle $PQR$ is : \hfill\brak{Apr 2023}
\begin{enumerate}                  
        \item $\frac{5}{4}$    
        \item $\frac{13}{4}$
        \item $\frac{5}{8}$
        \item $\frac{13}{8}$
\end{enumerate}
\item Three dice are rolled. If the probability of getting different numbers on the three dice is
$\frac{p}{q}$ ,where $p$ and $q$ are co-prime, then $q-p$ is equal to \hfill\brak{Apr 2023}
        \begin{enumerate}
        \item $1$ 
        \item $2$
        \item $4$ 
        \item $3$
    \end{enumerate}
\item In a group of 100 persons 75 speak English and 40 speak Hindi. Each person speaks at least one of the two languages. If the number of persons, who speak only English is $\alpha$ and the number of persons who speak only Hindi is $\beta $ then the eccentricity of the ellipse $25 \brak{\beta ^2 x^2 +\alpha ^2 y^2} = \alpha ^2 \beta ^2$ is : \hfill\brak{Apr 2023}
\begin{enumerate}
        \item $\frac{\sqrt{129}}{12}$ 
        \item $\frac{\sqrt{117}}{12}$
 \item $\frac{\sqrt{119}}{12}$ 
        \item $\frac{3\sqrt{15}}{12}$
\end{enumerate}
\item If the solution curve $f\brak{x, y} = 0$ of the differential equation $\brak{1+ \log_e x}\frac{dx}{dy} - x \log_e x =e^y, x>0$, passes through the points $\brak{1, 0}$ and $\brak{\alpha,2}$, then $\alpha ^{\alpha}$ is equal to : \hfill\brak{Apr 2023}
        \begin{enumerate}
                \item $e^{\sqrt{2}e^2}$
                \item $e^{e^2}$
                \item $e^{2e^{\sqrt{2}}}$ 
        \item $e^{2e^2}$    
\end{enumerate}
\item Let the sets $A$ and $B$ denote the domain and range respectively of the function $f(x)=\frac{1}{\sqrt{\sbrak{x}-x}}$, where $\sbrak{x}$ denotes the smallest integer greater than or equal to $x$. Then among the statements :\hfill\brak{Apr 2023} \\
	(S1) : $A \cap B = \brak{1,\infty} - \mathbb{N}$ and \\
(S2) : $A \cup B = \brak{1,\infty}$
 \begin{enumerate}
	 \item only \brak{S1} is true
	 \item neither \brak{S1} nor \brak{S2} is true
	 \item only \brak{S2} is true
	 \item both \brak{S1} and \brak{S2} are true
\end{enumerate}
\item Let $a \neq b$ be two-zero real numbers. Then the numbers of elements in the set \\ $X$ = $\{$$z \in \mathbb{C}:Re(az^2+bz)=a$  and $Re(bz^2+az)=b$$\}$ is equal to: \hfill\brak{Apr 2023}
\begin{enumerate}
                \item 0
                \item 2
                \item 1
                \item 3
\end{enumerate}
% \end{enumerate}

