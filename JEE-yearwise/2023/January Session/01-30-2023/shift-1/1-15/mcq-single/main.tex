\iffalse
\title{2023}
\author{EE24BTECH11008}
\section{mcq-single}
\fi
%\begin{enumerate}
    \item Let $\textbf{A}=\myvec{m&n\\p&q},d=\abs{A} \neq 0 \abs{A-D\brak{Adj A}}=0$.Then 
	    \hfill{\brak{2023-Jan}}\\
    \begin{enumerate}
        \item $\brak{1+d}^2=\brak{m+q}^2$
        \item $1+d^2=\brak{m+q}^2$
        \item $\brak{1+d}^2=m^2+q^2$
        \item $1+d^2=m^2+q^2$
    \end{enumerate}
    \item The line $l_1$ passes through the point $\brak{2,6,2}$ and is perpendicular to the plane $2x+y-2z=10.$ Then the shortest distance between the line $l_1$ and the line $\frac{x+1}{2}+\frac{y+4}{-3}+\frac{z}{2}$ is $:$\hfill{\brak{2023-Jan}}\\
    \begin{enumerate}
        \item $7$
        \item $\frac{19}{3}$
        \item $\frac{19}{2}$
        \item $9$
    \end{enumerate}
    \item If an unbiased die marked with $-2,-1,0,1,2,3$ on its faces, is through five times, then the probability that the product of the outcomes is positive is $\dots$\hfill{\brak{2023-Jan}}\\
    \begin{enumerate}
        \item $\frac{881}{2592}$
        \item $\frac{521}{2592}$
        \item $\frac{440}{2592}$
        \item $\frac{27}{288}$
    \end{enumerate}
    \item Let the system of linear equations $$x+y+kz=2$$ $$2x+3y-z=1$$ $$3x+4y+2z=k$$ have infinetely many solutions.Then the system $$\brak{k+1}x+\brak{2k-1}y=7$$ $$\brak{2k+1}x+\brak{k+5}y=10 \text{ has}: $$\hfill{\brak{2023-Jan}}\\
    \begin{enumerate}
        \item infenitely many solutions
        \item unique solution satisfying $x-y=1$
        \item no solution
        \item unique solution satisfying $x+y=1$
    \end{enumerate}
    \item If $\tan 15\degree +\frac{1}{\tan 75\degree}+\frac{1}{\tan 105\degree}+\tan 195\degree =2a,$ then the value of $\brak{a+\frac{1}{a}}$ is $:$
	    \hfill{\brak{2023-Jan}}\\
    \begin{enumerate}
        \item $4$
        \item $4-2\sqrt{3}$
        \item $2$
        \item $5-\frac{3}{2}\sqrt{3}$
    \end{enumerate}
    \item Suppose $f :\textbf{R} \to \brak{0,\infty}$ be a differentiable function such that $$5f\brak{x+y}=f\brak{x}\cdot f\brak{y},\forall x,y \in \textbf{R}$$ If $f\brak{3}=320,$ then $\sum_{n=0}^5f\brak{n}$ is equal to $:$
	  \hfill{\brak{2023-Jan}}  \\
    \begin{enumerate}
        \item $6875$
        \item $6575$
        \item $6825$
        \item $6528$
    \end{enumerate}
    \item If $a_n=\frac{-2}{4n^2-16n+15},$ then $a_1+a_2+\dots +a_25$ is equal to $:$
	    \hfill{\brak{2023-Jan}}\\
    \begin{enumerate}
        \item $\frac{51}{144}$
        \item $\frac{49}{138}$
        \item $\frac{50}{141}$
        \item $\frac{52}{147}$
    \end{enumerate}
    \item If the coefficient of $x^{15}$ in the expansion of $\brak{ax^2+\frac{1}{bx^{\frac{1}{3}}}}^{15}$ is equal to the coefficient of $x^{-15}$ in the expansion of $\brak{ax^{\frac{1}{3}}-\frac{1}{bx^3}},$ where $a$ and $b$ are positive real numbers, then for each such ordered pair $\brak{a,b}$ $:$
	   \hfill{\brak{2023-Jan}} \\
    \begin{enumerate}
        \item $a=b$
        \item $ab=1$
        \item $a=3b$
        \item $ab=3$
    \end{enumerate}
    \item if $\overline{a},\overline{b},\overline{c}$ are three non-zero vectors and $\hat{n}$ is a vector perpendicular to $\overline{c}$ such that $\overline{a}=\alpha\overline{b}-\hat{n},\brak{\alpha\neq 0}$ and $\overline{b}\cdot\overline{c}=12,$ then $\abs{\overline{c}X\brak{\overline{a}X\overline{b}}}$ is equal to $:$
	  \hfill{\brak{2023-Jan}}  \\
    \begin{enumerate}
        \item $15$
        \item $9$
        \item $12$
        \item $6$
    \end{enumerate}
    \item The number of points on the curve $y=54x^5-135x^4-70x^3+180x^2+210x$ at which the normal lines are parallel to $x+90y+2=0$ is $:$
	  \hfill{\brak{2023-Jan}}  \\
    \begin{enumerate}
        \item $2$
        \item $3$
        \item $4$
        \item $0$
    \end{enumerate}
    \item Let $y=x+2,4y=3x+6$ and $3y=4x+1$ be three tangent lines to the circle $\brak{x-h}^2+\brak{y-k}^2=r^2.$ Then $h+k$ is equal to $:$
	   \hfill{\brak{2023-Jan}} \\
    \begin{enumerate}
        \item $5$
        \item $5\brak{1+\sqrt{2}}$
        \item $6$
        \item $5\sqrt{2}$
    \end{enumerate}
    \item Let the solution curve $y=y\brak{x}$ of the differential equation $$\frac{dy}{dx}-\frac{3x^5\tan^{-1}x^3}{\brak{1+x^6}^{\frac{3}{2}}}y=2x\cdot \exp{\frac{x^3-\tan^{-1}x^3}{\sqrt{\brak{1+x^6}}}}$$ pass through the origin. Then $y\brak{1}$ is equal to $:$
	   \hfill{\brak{2023-Jan}} \\
    \begin{enumerate}
        \item $\exp\brak{{\frac{4-\pi}{4\sqrt{2}}}}$
        \item $\exp\brak{{\frac{\pi-4}{4\sqrt{2}}}}$
        \item $\exp\brak{{\frac{1-\pi}{4\sqrt{2}}}}$
        \item $\exp\brak{{\frac{4+\pi}{4\sqrt{2}}}}$
    \end{enumerate}
    \item Let a unit vector $\hat{\textbf{OP}}$ make angle $\alpha,\beta,\gamma$ with the positive directions of the co-ordinate axes $\textbf{OX,OY,OZ}$ respectively, where $\beta \in \brak{0,\frac{\pi}{2}}\hat{\textbf{OP}}$ is perpendicular to the plane through points $\brak{2,3,4}$ and $\brak{1,5,7},$ then which of the following is true $:$
	  \hfill{\brak{2023-Jan}}  \\
    \begin{enumerate}
        \item $\alpha\in\brak{\frac{\pi}{2},\pi}$ and $\gamma\in\brak{\frac{\pi}{2},\pi}$
        \item $\alpha\in\brak{0,\frac{\pi}{2}}$ and $\gamma\in\brak{0,\frac{\pi}{2}}$
        \item $\alpha\in\brak{\frac{\pi}{2},\pi}$ and $\gamma\in\brak{0,\frac{\pi}{2}}$
        \item $\alpha\in\brak{0,\frac{\pi}{2}}$ and $\gamma\in\brak{\frac{\pi}{2},\pi}$
    \end{enumerate}
    \item If $\sbrak{t}$ denotes the greatest integer $\le t,$ then the value of $$\frac{3\brak{e-1}^2}{e}\int_1^2x^2e^{\sbrak{x}+\sbrak{x^3}}dx \text{ is :}$$
	  \hfill{\brak{2023-Jan}}  \\
    \begin{enumerate}
        \item $e^9-e$
        \item $e^8-e$
        \item $e^7-e$
        \item $e^8-1$
    \end{enumerate}
    \item If $\textbf{P}\brak{h,k}$ be point on the parabola $x=4y^2,$ which is nearest to the point $\textbf{Q}\brak{0,33},$ then the distance of $\textbf{P}$ from the directrix of the parabola $y^2=4\brak{x+y}$ is equal to $:$
	   \hfill{\brak{2023-Jan}} \\
    \begin{enumerate}
        \item $2$
        \item $4$
        \item $8$
        \item $6$
    \end{enumerate}
%\end{enumerate}

