
\iffalse
  \title{2023}
  \author{EE24BTECH11032}
  \section{mcq-single}
\fi

%   \begin{enumerate}
        \item Let the function $f\brak{x}=2x^3+\brak{2p-7}x^2+3\brak{2p-9}x-6$ have a maxima for some value of $x<0$ and a minima for some value of $x>0$. Then, the set of all values of p is \hfill{\sbrak{Jan 2023}}
    \begin{enumerate}
        \item $\brak{\frac{9}{2},\infty}$
        \item $\brak{0,\frac{9}{2}}$
        \item $\brak{-\infty,\frac{9}{2}}$
        \item $\brak{-\frac{9}{2},\frac{9}{2}}$
    \end{enumerate}
    \item Let $z$ be a complex number such that $\abs{\frac{z-2i}{z+i}}=2,z\neq-i$. Then $z$ lies on the circle of radius 2 and centre \hfill{\sbrak{Jan 2023}}
    \begin{enumerate}
        \item $\brak{2,0}$
        \item $\brak{0,0}$
        \item $\brak{0,2}$
        \item $\brak{0,-2}$
    \end{enumerate}
    \item If the function \\
    $f\brak{x}=\left\{ \begin{array}{ll} \brak{1+\abs{\cos{x}}}\frac{\lambda}{\abs{\cos{x}}} \quad ,0<x < \frac{\pi}{2} \\ \mu \quad, x=\frac{\pi}{2} \\ e^{\frac{\cot6x}{\cot4x}} \quad ,\frac{\pi}{2}<x<\pi \end{array} \right. $\\
    is continuous at
    $x=\frac{\pi}{2}$, then $9\lambda+6log_e\mu+\mu^6-e^{6\lambda}$ is equal to \hfill{\sbrak{Jan 2023}}
    \begin{enumerate}
        \item $11$
        \item $8$
        \item $2e^4+8$
        \item $10$
    \end{enumerate}
    \item Let $f\brak{x}=2x^n+\lambda,\lambda \in \vec{R},n \in \vec{N}$, and $f\brak{4}=133, f\brak{5}=255$. Then the sum of all the positive integer divisors of $\brak{f\brak{3}-f\brak{2}}$ is \hfill{\sbrak{Jan 2023}}
    \begin{enumerate}
        \item $61$
        \item $60$
        \item $58$
        \item $59$
    \end{enumerate}
    \item If the four points, whose position vectors are $3\hat{i}-4\hat{j}+2\hat{k},\hat{i}+2\hat{j}-\hat{k}$ and $5\hat{i}-2\alpha\hat{j}+4\hat{k}$ are coplanar, then $\alpha$ is equal to \hfill{\sbrak{Jan 2023}}
    \begin{enumerate}
        \item $\frac{73}{17}$
        \item $\frac{-107}{17}$
        \item $\frac{-73}{17}$
        \item $\frac{107}{17}$
    \end{enumerate}
    \item Let $A=\myvec{\frac{1}{\sqrt{10}} & \frac{3}{\sqrt{10}} \\ \frac{-3}{\sqrt{10}} & \frac{1}{\sqrt{10}}}$ and $B=\myvec{1 & -i\\ 0 &1}$, where $i=\sqrt{-1}$.If $M=A^\top BA$, then the inverse of the matrix $AM^{2023}A^\top$ is \hfill{\sbrak{Jan 2023}}
    \begin{enumerate}
        \item $\myvec{1 & -2023i \\ 0 & 1}$
        \item $\myvec{1 & 0 \\ -2023i & 1}$
        \item $\myvec{1 & 0 \\ 2023i & 1}$
        \item $\myvec{1 & 2023i \\ 0 & 1}$
    \end{enumerate}
    \item Let $\Delta,\nabla \in \cbrak{\wedge,\vee}$ be such that $\brak{p \to q}\Delta\brak{p\nabla q}$ is a tautology. Then \hfill{\sbrak{Jan 2023}}
    \begin{enumerate}
        \item $\Delta=\wedge,\nabla=\vee$
        \item $\Delta=\vee,\nabla=\wedge$
        \item $\Delta=\vee,\nabla=\vee$
        \item $\Delta=\wedge,\nabla=\wedge$
    \end{enumerate}
    \item The number of numbers, strictly between $5000$ and $10000$ can be formed using the digits $1,3,5,7,9$ without repetition, is \hfill{\sbrak{Jan 2023}}
    \begin{enumerate}
        \item $6$
        \item $12$
        \item $120$
        \item $72$
    \end{enumerate}
    \item The number of functions f:$\cbrak{1,2,3,4} \to \cbrak{a \in \vec{Z}:\abs{a}\leq8}$ satisfying $f\brak{n}+\frac{1}{n}f\brak{n+1}=1,\forall n \in \cbrak{1,2,3}$ is \hfill{\sbrak{Jan 2023}}
    \begin{enumerate}
        \item $3$
        \item $4$
        \item $1$
        \item $2$
    \end{enumerate}
    \item The equations of two sides of a variable trianlge are $x=0$ and $y=3$, and its third side is a tangent to the parabola $y^2=6x$. The locus of its circumcentre is : \hfill{\sbrak{Jan 2023}}
    \begin{enumerate}
        \item $4y^2-18y-3x-18=0$
        \item $4y^2+18y+3x+18=0$
        \item $4y^2-18y+3x+18=0$
        \item $4y^2-18y-3x+18=0$
    \end{enumerate}
    \item Let f:$\vec{R} \to \vec{R}$ be a function defined by $f\brak{x}=log_{\sqrt{m}}\cbrak{\sqrt{2}\brak{\sin x-\cos x}+m-2}$, for some m, such that the range of f is $\sbrak{0,2}$. Then the value of m is \hfill{\sbrak{Jan 2023}}
    \begin{enumerate}
        \item $5$
        \item $3$
        \item $2$
        \item $4$
    \end{enumerate}
    \item Let A,B,C be $3 \times 3$ matrices such that A is symmetric and B and C are skew-symmetric. Consider the statements\\
    $\brak{S1}A^{13}B^{26}-B^{26}A^{13}$ is symmetric \\
    $\brak{S2}A^{26}C^{13}-C^{13}A^{26}$ is symmetric\\
    Then, \hfill{\sbrak{Jan 2023}}
    \begin{enumerate}
        \item Only S2 is true
        \item Only S1 is true
        \item Both S1 and S2 are false
        \item Both S1 and S2 are true
    \end{enumerate}
    \item Let $y=y\brak{t}$ be a solution of the differential equation $\frac{dy}{dt}+\alpha y=\gamma e^{-\beta t}$\\
    Where, $\alpha >0, \beta >0$ and $\gamma >0$. Then $\lim_{t \to \infty}y\brak{t}$ \hfill{\sbrak{Jan 2023}}
    \begin{enumerate}
        \item is $0$
        \item does not exist
        \item is $1$
        \item is $-1$
    \end{enumerate}
    \item $\sum_{k=0}^6 \comb{51-k}{3}$ is equal to \hfill{\sbrak{Jan 2023}}
    \begin{enumerate}
        \item $\comb{51}{4}-\comb{45}{4}$
        \item $\comb{51}{3}-\comb{45}{3}$
        \item $\comb{52}{4}-\comb{45}{4}$
        \item $\comb{52}{3}-\comb{45}{3}$
    \end{enumerate}
    \item The shortest distance between the lines $x+1=2y=-12z$ and $x=y+2=6z-6$ \\ is 
\hfill{\sbrak{Jan 2023}}
    \begin{enumerate}
        \item $2$
        \item $3$
        \item $\frac{5}{2}$
        \item $\frac{3}{2}$
    \end{enumerate}

% \end{enumerate}
