\iffalse
\title{Assignment 3}
\author{AI24BTECH11018}
\section{integer}
\fi
\item Let $a1 = 8, a_2, a_3,\cdots  a_n$ be an A.P. If the sum of its 
first four terms is $50$ and the sum of its last four 
terms is $170$, then the product of its middle two 
terms is
\hfill{\brak{\text{Jan 2023}}}
\item A\brak{2, 6, 2}, B\brak{-4, 0, \lambda}, C\brak{2, 3, -1} and D\brak{4, 5, 0}, 
$\abs \lambda \leq 5$ are the vertices of a quadrilateral $ABCD$. If 
its area is $18$ square units, then $5-6\lambda$ is equal to
\hfill{\brak{\text{Jan 2023}}}
\item The number of 
$3-digit$ numbers, that are divisible 
by either $2$ or $3$ but not divisible by $7$ is
\hfill{\brak{\text{Jan 2023}}}
\item The remainder when $19^{200}$ + $23^{200}$ is divided by $49$
is 
\hfill{\brak{\text{Jan 2023}}}
\item if$\int_{0}^{1}\brak{x^{21}+x^{14}+x^7}\brak{2x^{14}+3x^{7}+6}^\frac{1}{6}$ where $l,m,n \in m$ and $n$ are co primes then l+m+n is equal to
\hfill{\brak{\text{Jan 2023}}}
\item if $f\brak{x}=x^2+g^{'}\brak{1}x+g^{''}\brak{2}$ and $g\brak{x}=f\brak{1}x^2+xf^{'}\brak{x}+f^{''}\brak{x}$
\hfill{\brak{\text{Jan 2023}}}
\item $\overrightarrow{v}=\alpha \hat{i}+2j-3k$,$\overrightarrow{w}=2\alpha \hat{i}+j-k$, and $\overrightarrow{u}$ be a vector such that $\abs {\overrightarrow{u}}=\alpha  \textgreater 0$. If the minimum value of the scaler triple product $[\overrightarrow{u} \overrightarrow{v} \overrightarrow{w}]$ is $-\alpha \sqrt{3401}$ and $\abs {\overrightarrow{u}\cdot \hat{i}}^2=\frac{m}{n}$ where $m$ and $n$ are coprime natural numbers then m+n is equal to
\hfill{\brak{\text{Jan 2023}}}
\item The number of words, with or without meaning, 
that can be formed using all the letters of the word 
ASSASSINATION so that the vowels occur 
together, is
\hfill{\brak{\text{Jan 2023}}}
\item Let $A$ be the area bounded by the curve $y=x\abs {x-3}$, the X-axis and the ordinates $x=-1$ and $x=2$. Then $12A$ is equal to 
\hfill{[jan 2023]}
\item Let $f:\mathbb{R} \to \mathbb{R}$ be a differentiable function such that $f^{'}\brak{x}+f\brak{x}=\int_{0}^{2}f\brak{t}dt$. if $f\brak{0}=e^-2$, then $2f\brak{0}-f\brak{2}$ is equal to
\hfill{\brak{\text{Jan 2023}}}
%\end{enumerate}

%\end{document}
