
    \iffalse
    \title{Assignment}
    \author{ee24btech11059}
    \section{mcq-single}
    \fi
    \item{
          	The set of all values of $a$ for which $\lim_{x\to a} \brak{\sbrak{x-5} - \sbrak{2x+2}}= 0$, where $[\cdot]$ denotes the greatest integer less than or equal to $\cdot$ , is equal to \text{  } \hfill
                {[Jan 2023]}
                \begin{multicols}{4}
					\begin{enumerate}
						\item $\lsbrak{}–7.5, –6.5\rbrak{} $
						\item $\lbrak{}–7.5, –6.5\rsbrak{} $
						\item $\sbrak{–7.5, –6.5} $
						\item $\brak{–7.5, –6.5} $
					\end{enumerate}
				\end{multicols}
            }
    %code by ysiddhanth 
    \item{
           	Let $p$ and $q$ be two statements. Then $\sim (p \land (p \Rightarrow \sim q)$ is equivalent to \\ \text{ }
           	\hfill
           	{[Jan 2023]}
                \begin{multicols}{4}
                	\begin{enumerate}
                		\item $(\sim p) \lor q$
                		\item $p \lor ((\sim p) \land q)$
                		\item $p \lor (p \land q)$
                		\item $p \lor (p \land(\sim q))$
                	\end{enumerate}
                \end{multicols}
        }
\item{
        	
        	The locus of the mid points of the chords of the circle $C1: (x - 4)^2 + (y - 5)^2 = 4$ which subtend an angle $\theta_i$ at the centre of the circle $C_1$, is a circle of radius $r_i$. If $\theta_1 = \frac{\pi}{3}, \theta_3 = \frac{2\pi}{3}$ and $r_1^2 = r_2^2+r_3^2$, then $\theta_2$ is equal to
        	\hfill
        	{[Jan 2023]}
        	\begin{multicols}{4}
        		\begin{enumerate}
        			\item  $\frac{3\pi}{4}$
        			\item  $\frac{\pi}{4}$
        			\item  $\frac{\pi}{6}$
        			\item  $\frac{\pi}{2}$
        		\end{enumerate}
        	\end{multicols}
        	
        }
    \item{
     		If $f\brak{x} = \frac{2^{2x}}{2^{2x}+2}, x\in \mathbb{R}$, then $f\brak{\frac{1}{2023}}+ f\brak{\frac{2}{2023}}+........+ f\brak{\frac{2022}{2023}}$ is equal to \\ \text{ }
            \hfill
            {[Jan 2023]}
            \begin{multicols}{4}
                \begin{enumerate}
                	\item 1010
                	\item 2011
                	\item 1011
                	\item 2010
                \end{enumerate}
            \end{multicols}
        
        }
    \item{
            If the system of equations
            \[x + 2y + 3z = 3\]
            \[4x + 3y - 4z = 4\]
            \[8x + 4y - \lambda z = 9 + \mu\]
            has infinitely many solutions, then the ordered pair
            \((\lambda, \mu)\) is equal to : \\ \text{ }
           	\hfill
                {[Jan 2023]}
            
           \begin{multicols}{4}
            	\begin{enumerate}
            		\item $\brak{-\frac{72}{5},\frac{21}{5}}$
            		\item $\brak{\frac{72}{5},-\frac{21}{5}}$
            		\item $\brak{\frac{72}{5},\frac{21}{5}}$
            		\item $\brak{-\frac{72}{5},-\frac{21}{5}}$
            	\end{enumerate}
            \end{multicols}
        
        }
 	\item{
        	Let the plane containing the line of intersection of the planes $P_{1}: x + \brak{\lambda+4}y + z = 1$ and $P_{2}: 2x + y + z = 2$ pass through the points $(0, 1, 0)$ and $(1, 0, 1)$. Then the distance of the point $(2, \lambda,- \lambda)$ from the plane $P_{2}$ is
        	\hfill
        	{[Jan 2023]}
        	
        	\begin{multicols}{4}
        		\begin{enumerate}
        			\item $4\sqrt{6}$
        			\item $3\sqrt{6}$
        			\item $5\sqrt{6}$
        			\item $2\sqrt{6}$
        		\end{enumerate}
        	\end{multicols}
        	
        }
 	\item{
			If $\brak{\binom{30}{1}}^2 + 2\brak{\binom{30}{2}}^2+ 3\brak{\binom{30}{3}}^2 + ... + 30\brak{\binom{30}{30}}^2 = \frac{\alpha 60!}{\brak{30!}^2}$, then $\alpha$ is equal to \\
			\text{ }
			\hfill
			{[Jan 2023]}
			
			\begin{multicols}{4}
				\begin{enumerate}
					\item 60
					\item 30
					\item 15
					\item 10
				\end{enumerate}
			\end{multicols}
			
		}
 	\item{
			If the foot of the perpendicular drawn from $(1, 9, 7)$
			to the line passing through the point $(3, 2, 1)$ and
			parallel to the planes $x + 2y + z = 0$ and $3y - z = 3$
			is $(\alpha, \beta, \gamma)$, then $\alpha + \beta + \gamma$ is equal to\hfill
			{[Jan 2023]}
			
			\begin{multicols}{4}
				\begin{enumerate}
					\item -1
					\item 1
					\item 3
					\item 5
				\end{enumerate}
			\end{multicols}
			
		}
    \item{
          	Let $A$ be a $3 \times 3$ matrix such that $|\text{adj} (\text{adj} (\text{adj} A))| = 12^4$. Then $|A^{-1} \text{adj} A|$ is equal to
             \text{ }
             \hfill
                {[Jan 2023]}
            \begin{multicols}{4}
                \begin{enumerate}
                	\item $12$
                	\item $2\sqrt{3}$
                	\item $\sqrt{6}$
                	\item $1$
                \end{enumerate}
            \end{multicols}

        %code by ysiddhanth 
        
        }
    \item{
	        	The value of $\brak{\frac{1 + \sin{\frac{2\pi}{9}}+ i\cos{\frac{2\pi}{9}}}{1 + \sin{\frac{2\pi}{9}}- i\cos{\frac{2\pi}{9}}}}^3$ is 
            
             \hfill
                {[Jan 2023]}
            \begin{multicols}{2}
                \begin{enumerate}
                	\item $\frac{1}{2}\brak{\sqrt{3}+1}$
                	\item $-\frac{1}{2}\brak{1-i\sqrt{3}}$
                	\item $\frac{1}{2}\brak{1-\sqrt{3}}$
                	\item $-\frac{1}{2}\brak{\sqrt{3}-1}$
                \end{enumerate}
            \end{multicols}
        
        }
    \item{
    		
    		The number of square matrices of order 5 with entries from the set $\cbrak{0, 1}$, such that the sum of all the elements in each row is 1 and the sum of all the elements in each column is also 1, is 
             \hfill
                {[Jan 2023]}
			\begin{multicols}{2}
				\begin{enumerate}
					\item 120
					\item 225
					\item 150
					\item 125
				\end{enumerate}
			\end{multicols}
        
        }
    \item{
        	$\int_{\frac{3\sqrt{2}}{4}}^ \frac{3\sqrt{3}}{4} \frac{48}{\sqrt{9-4x^2}}dx$ is equal to \hfill
                {[Jan 2023]}
				\begin{multicols}{4}
	                \begin{enumerate}
	                	\item $\frac{\pi}{6}$
	                	\item $\frac{\pi}{2}$
	                	\item $\frac{\pi}{3}$
	                	\item $2\pi$
	                \end{enumerate}
				\end{multicols}
        
        }
 \item{
    	
	    	The number of real solutions of the equation $3\brak{x^2 + \frac{1}{x^2}} - 2\brak{x + \frac{1}{x}}+5 =0$ is\\
	    	\text{   }\hfill
	    	{[Jan 2023]}
	    	\begin{multicols}{4}
	    		\begin{enumerate}
	    			\item 3
	    			\item 0
	    			\item 2
	    			\item 4
	    		\end{enumerate}
	    	\end{multicols}
	    	
	    }
    \item{
	
		    Let
		    $\alpha = 4\hat{i} + 3\hat{j} + 5\hat{k}$
		    and
		    $\beta = 2\hat{i} + 4\hat{j}$.
		    Let
		    $\beta_1$
		    be
		    parallel to
		    $\alpha$
		    and
		    $\beta_2$
		    be perpendicular to
		    $\alpha$.
		    If
		    $\beta = \beta_1 + \beta_2$,
		    then the value of
		    $5\beta_2\cdot\brak{\hat{i} + \hat{j} + \hat{k}}$
		    is
			\text{   }\hfill
			{[Jan 2023]}
			\begin{multicols}{4}
				\begin{enumerate}
						\item 7
						\item 9
						\item 6
						\item 11
				\end{enumerate}
			\end{multicols}
			
		}

    \item{
        
            Let $f\brak{x}$ be a function such that $f\brak{x+y} = f\brak{x}\cdot f\brak{y}$ for all $x, y \in \mathbb{R}$. If $f\brak{1} = 3$ and $\sum_{k=1}^{n} f\brak{k} = 3279$, then the value of $n$ is.\text{ }
             \hfill
              {[Jan 2023]}
			\begin{multicols}{2}              
	              		\begin{enumerate}
	              			\item 8
	              			\item 9
	              			\item 6
	              			\item 7
	              	\end{enumerate}
  			\end{multicols}      
        }



