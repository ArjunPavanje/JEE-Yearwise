
\iffalse
\title{assignment}
\author{EE24BTECH11020}
\section{mcq-single}
\fi

% 1
\item The domain of $f\brak{x} = \frac{\log_{(x+1)}(x-2)}{e^{2 \log_e x} - (2x + 3)} $, $ x \in \mathbb{R} $, is: \hfill(Jan 2023)
\begin{multicols}{4}
    \begin{enumerate}
    \item $\mathbb{R} - \cbrak{1,-3}$ 
    \item $\brak{2,\infty} - \cbrak{3}$ 
     \item $\brak{1,\infty} - \cbrak{3}$ 
    \item $\mathbb{R} - \cbrak{-3}$ 
\end{enumerate}
\end{multicols}


%2
\item Let $f:\mathbb{R} \to \mathbb{R} $ be a function such that:
   \begin{align*}
        f(x) = \frac{x^2 + 2x + 1}{x^{2} + 1}
   \end{align*}
   Then: \hfill(Jan 2023)
   \begin{enumerate}
       \item $f\brak{x}$  is many-one in  $\brak{-\infty, -1}$
       \item $f\brak{x}$  is many-one in  $\brak{1,\infty}$
       \item $f\brak{x}$ \text{ is one-one in } $[1 ,\infty) $ \text{ but not in } $\brak{-\infty, \infty}$

       
       \item $ f\brak{x}$  is one-one in  $\brak{-\infty, \infty}$
   \end{enumerate}

%3
\item For two non-zero complex numbers  $z_1 $ and $z_2$, if:
\begin{align*}
   \text{Re}(z_1 z_2) = 0 \quad \text{and} \quad \text{Re}(z_1 + z_2) = 0,
\end{align*}
then which of the following are possible?
   \begin{enumerate}
       \item Im$\brak{z_1} > 0$ and Im$\brak{z_2} > 0$
       \item Im$\brak{z_1} < 0$ and Im$\brak{z_2} > 0$
      \item Im$\brak{z_1} > 0$ and Im$\brak{z_2} < 0$
      \item Im$\brak{z_1} < 0$ and Im$\brak{z_2} < 0$
   \end{enumerate}
   Choose the correct answer from the options given below: \hfill(Jan 2023)
\begin{multicols}{4}
   \begin{enumerate}
       \item $B$ and $D$
       \item $B$ and $C$
       \item $A$ and $B$
       \item $A$ and $C$
   \end{enumerate}
\end{multicols}
% 4
\item Let $\lambda \neq 0$ be a real number. Let $\alpha, \beta$ be the roots of the equation $ 14x^2 - 31x + 3\lambda = 0 $ and $\alpha, \gamma $ be the roots of the equation $ 35x^2 - 53x + 4\lambda = 0.$ Then 
$ \frac{3\alpha}{\beta} \quad \text{and} \quad \frac{4\alpha}{\gamma} $ 
are the roots of the equation:\hfill(Jan 2023)
\begin{multicols}{2}
\begin{enumerate}
    \item $ 7x^2 + 245x - 250 = 0$
    \item $ 7x^2 - 245x + 250 = 0$
    \item $ 49x^2 - 245x + 250 = 0$
    \item $ 49x^2 + 245x + 250 = 0 $
\end{enumerate}
\end{multicols}
% 5
\item Consider the system of equations:
   \begin{align*}
   \alpha x + 2y + z &= 1 \\
   2\alpha x + 3y + z &= 1 \\
   3x + \alpha y + 2z &= \beta
   \end{align*}
   For some \( \alpha, \beta \in \mathbb{R} \), which of the following is NOT correct?\hfill(Jan 2023)

   \begin{enumerate}
       \item No solution if $\alpha = -1$  and $\beta \neq 2$
       \item No solution for $\alpha = -1$ for all $\beta \in \mathbb{R}$
       \item No solution for $\alpha = 3$ and $\beta \neq 2$
       \item Solution for all $\alpha \neq -1$ and $\beta = 2$
   \end{enumerate}

% 6
\item Let $\alpha$ and $\beta$ be real numbers. Consider a $ 3 \times 3$ matrix $A$ such that: $A^2 = 3A + \alpha$I. If $A^4 = 21A + \beta$I   Then: \hfill(Jan 2023)
\begin{multicols}{4}
   \begin{enumerate}
       \item $\alpha = 1$
       \item $\alpha = 4$
       \item $\beta = 8$
       \item $\beta = -8$
   \end{enumerate}
\end{multicols}
%7
\item Let \( x = 2 \) be a root of the equation $x^2 + px + q = 0$ and 

\begin{align*}
f\brak{x} = 
\begin{cases} 
\frac{1 - \cos(x^2 - 4px + q^2 + 8q + 16)}{(x - 2p)^4} &, \quad x \neq 2p \\
0 &, \quad x = 2p 
\end{cases}
\end{align*}
Then  $\lim_{x \to 2p^+} \sbrak{f\brak{x}},$ where $\sbrak{.}$ denotes the greatest integer function, is\hfill(Jan 2023)
\begin{multicols}{4}
    \begin{enumerate}
    \item 2 
    \item 1
    \item 0 
    \item -1
\end{enumerate}
\end{multicols}

%8
\item 
Let 
$ f(x) = x + \frac{a}{\pi^2 - 4} \sin x + \frac{b}{\pi^2 - 4} \cos x, \quad x \in \mathbb{R} $ 
be a function which satisfies 
\begin{align*} 
f(x) = x + \int_0^{\pi/2} \sin(x + y) f(y) \, dy.  
\text{ Then }  \brak{a + b} \text{ is equal to: }
\end{align*}\hfill(Jan 2023)
\begin{multicols}{4}
\begin{enumerate}
   \item $-\pi(\pi + 2) $
   \item $-2\pi(\pi + 2) $
   \item $-2\pi(\pi - 2) $
   \item $-\pi(\pi - 2) $
\end{enumerate}
\end{multicols}
%9
\item Let \quad $$A = \left\{ (x, y) \in \mathbb{R}^2: y \geq 0, 2x \leq y \leq \sqrt{4 - (x-1)^2} \right\}$$ $$B = \cbrak{ (x, y) \in \mathbb{R} \times \mathbb{R}: 0 \leq y \leq \min \cbrak{ 2x, \sqrt{4 - (x-1)^2} }}$$

$\text{Then the ratio of the area of A to the area of B is}$
\hfill(Jan 2023)
\begin{multicols}{2}
   \begin{enumerate}
       \item $ \frac{\pi - 1}{\pi + 1}$\\
     \item $ \frac{\pi}{\pi - 1}$
    \item $ \frac{\pi}{\pi + 1}$\\
     \item $ \frac{\pi + 1}{\pi - 1}$
   \end{enumerate}
\end{multicols}

% 10
\item $\text{Let } \Delta \text{ be the area of the region}
\cbrak{ (x, y) \in \mathbb{R}^2: x^2 + y^2 \leq 21, y^2 \leq 4x, x \geq 1}. \text{ Then } \frac{1}{2} \brak{\Delta - 21 \sin^{-1} \frac{2}{\sqrt{7}}}$ is equal to \hfill(Jan 2023)
\begin{multicols}{2}
\begin{enumerate}
    \item $2\sqrt{3} - \frac{1}{3}$\\
    \item $\sqrt{3} - \frac{2}{3}$
    \item $2\sqrt{3} - \frac{2}{3}$\\
    \item $\sqrt{3} - \frac{4}{3}$
\end{enumerate}
\end{multicols}

% 11
\item A light ray emits from the origin at an angle of $30\degree$ with the positive x-axis. After geting reflected by the line $x + y = 1$, then intersects the x-axis at Q. Then the abscissa of Q is \hfill(Jan 2023)
\begin{multicols}{2}
\begin{enumerate}
    \item $\frac{2}{\brak{\sqrt{3} - 1}}$\\
     \item $\frac{2}{\brak{3 + \sqrt{3}}}$
     \item $\frac{2}{\brak{3 - \sqrt{3}}}$\\
     \item $\frac{2}{2\brak{\sqrt{3} + 1}}$
\end{enumerate}
\end{multicols}
% 12
\item Let $B$ and $C$ be the two points on the line $y + x = 0$
such that $B$ and $C$ are symmetric with respect to the origin. Suppose $A$ is a point on $y - 2x = 2$ such that $\triangle ABC$ is an equilateral triangle. Then, the area of the $\triangle ABC$ is\hfill(Jan 2023)
\begin{multicols}{4}
\begin{enumerate}
    \item $3\sqrt{3}$
   \item $2\sqrt{3}$
   \item $\frac{8}{\sqrt{3}}$
   \item $\frac{10}{\sqrt{3}}$
\end{enumerate}
\end{multicols}
% 13
\item The tangents at points  $A\brak{4, -11}$ and $B\brak{8, -5}$ on the circle: $x^2 + y^2 - 3x + 10y - 15 = 0$ intersect at $C$. Find the radius of the circle with center $C$ and line joining A and B as its
tangent, is equal to\hfill(Jan 2023)
\begin{multicols}{4}
\begin{enumerate}
    \item $3\sqrt{3}$
   \item $2\sqrt{3}$
   \item $\frac{8}{\sqrt{3}}$
   \item $\frac{10}{\sqrt{3}}$
\end{enumerate}
\end{multicols}
% 14
\item Let $\sbrak{x}$ denote the greatest integer $\leq x$. Consider the function $f(x) = \max \{x^2, 1 + \sbrak{x}\}$. Then the value of the integral $\int_0^2 f(x)\,dx$ is: \hfill (Jan 2023)
\begin{multicols}{2}
\begin{enumerate}
    \item $\frac{5 + 4\sqrt{2}}{3}$\\
    \item $\frac{8 + 4\sqrt{2}}{3}$
    \item $\frac{1 + 5\sqrt{2}}{3}$\\
    \item $\frac{4 + 5\sqrt{2}}{3}$\\
\end{enumerate}
\end{multicols}

%15
\item If the vectors $\overrightarrow{\vec{a}} = \lambda\vec{i} + \mu\vec{j} + 4\vec{k}$, $\overrightarrow{\vec{b}} = -2\vec{i} + 4\vec{j} -2\vec{k}$ and  $\overrightarrow{\vec{c}} = 2\mathbf{i} + 3\mathbf{j} + \mathbf{k}$ are coplanar and the projection\\ of  $\overrightarrow{\vec{a}}$ on $\overrightarrow{\vec{b}}$ is $\sqrt{54}$ units, then the sum of all possible values of $\lambda + \mu$ is equal to \hfill(Jan 2023)
\begin{multicols}{4}
    \begin{enumerate}
    \item 6
    \item 18
    \item 0 
    \item 24
\end{enumerate}
\end{multicols}
