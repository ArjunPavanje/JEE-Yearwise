\iffalse
\title{2024-January Session-01-30-2024-shift-1}
\author{EE24BTECH11008-ASLIN GARVASIS}
\section{mcq-single}
\fi
%\begin{enumerate}
    \item A line passing through the point $A\brak{9,0}$ makes an angle of $30\degree$ with the positive direction of x-axis. If this line is rotated about $A$ through an angle of $15\degree$ in the clockwise direction, then its equation in the new position is
	    \hfill{\brak{2024-Jan}} \\
    \begin{enumerate}
        \item $\frac{y}{\sqrt{3}-2}+x=9$
        \item $\frac{x}{\sqrt{3}-2}+y=9$
        \item $\frac{x}{\sqrt{3}+2}+y=9$
        \item $\frac{y}{\sqrt{3}+2}+x=9$
    \end{enumerate}
    \item Let $S_a$ denote the sum of first $n$ terms of an arithmetic progression. If $S_{20}=790$ and $S_{10}=145$, then $S_{15}-S_5$ is $:$
	   \hfill{\brak{2024-Jan}} \\
    \begin{enumerate}
        \item $395$
        \item $390$
        \item $405$
        \item $410$
    \end{enumerate}
    \item If $z = x + iy,$ $xy\neq 0,$ satisfies the equation $z^2+i\overline{z}=0,$ then $\abs{z^2}$ is equal to $:$
	   \hfill{\brak{2024-Jan}} \\
    \begin{enumerate}
        \item $9$
        \item $1$
        \item $4$
        \item $\frac{1}{4}$
    \end{enumerate}
    \item Let $\overrightarrow{a}=a_1\hat{i}+a_2\hat{j}+a_3\hat{k}$ and $\overrightarrow{b}=b_1\hat{i}+b_2\hat{j}+b_3\hat{k}$ be two vectors such that $\abs{\overrightarrow{a}=1};$ $\overrightarrow{a}\cdot\overrightarrow{b}=2$ and $\abs{b}=4.$ If $\overrightarrow{c}=2\brak{\overrightarrow{a}\times\overrightarrow{b}}-3\overrightarrow{b},$ then the angle between $\overrightarrow{b}$ and $\overrightarrow{c}$ is equal to $:$
	  \hfill{\brak{2024-Jan}}  \\
    \begin{enumerate}
        \item $\cos^{-1}\brak{\frac{2}{\sqrt{3}}}$
        \item $\cos^{-1}\brak{\frac{-1}{\sqrt{3}}}$
        \item $\cos^{-1}\brak{\frac{-\sqrt{3}}{2}}$
        \item $\cos^{-1}\brak{\frac{2}{3}}$
    \end{enumerate}
    \item The maximum area of a triangle whose one vertex is at $\brak{0,0}$ and the other two vertices lie on the curve $y=-2x^2+54$ at points $\brak{x,y}$ and $\brak{-x,y}$ where $y\textgreater 0$ is $:$
	  \hfill{\brak{2024-Jan}}  \\
    \begin{enumerate}
        \item $88$
        \item $122$
        \item $92$
        \item $108$
    \end{enumerate}
    \item The value of $\lim_{n \to \infty}\sum_{k=1}^n\frac{n^3}{\brak{n^2+k^2}\brak{n^2+3k^2}}$ is 
	   \hfill{\brak{2024-Jan}} \\
    \begin{enumerate}
        \item $\frac{\brak{2\sqrt{3}+3}\pi}{24}$
        \item $\frac{13\pi}{8\brak{4\sqrt{3}+3}}$
        \item $\frac{13\brak{2\sqrt{3}-3}\pi}{8}$
        \item $\frac{\pi}{8\brak{2\sqrt{3}+3}}$
    \end{enumerate}
    \item Let $g$ $:$ $\textbf{R}\to\textbf{R}$ be a non constant twice differential such that $g\prime\brak{\frac{1}{2}}=g\prime\brak{\frac{3}{2}}.$ If a real valued function $f$ is defined as $f\brak{x}=\frac{1}{2}\brak{g\brak{x}+g\brak{2-x}},$ then
	  \hfill{\brak{2024-Jan}}  \\
    \begin{enumerate}
        \item $f\prime\brak{x}=0$ for atleast two $x$ in $\brak{0,2}$
        \item $f\prime\brak{x}=0$ for exactly one $x$ in $\brak{0,1}$
        \item $f\prime\brak{x}=0$ for no $x$ in $\brak{0,1}$
        \item $f\prime\brak{\frac{3}{2}}+f\prime\brak{\frac{1}{2}}=1$
    \end{enumerate}
    \item The area $\brak{\text{in square units}}$ of the region bounded by the parabola $y^2=4\brak{x-2}$ and the line $y=2x-8$
	  \hfill{\brak{2024-Jan}}  \\
    \begin{enumerate}
        \item $8$
        \item $9$
        \item $6$
        \item $7$
    \end{enumerate}
    \item Let $y=y\brak{x}$ be the solution of the differential equation $$\sec x dy +\cbrak{2\brak{1-x}\tan x +x\brak{2-x}}dx=0$$ such that $y\brak{0}=2\cdot$ Then $y\brak{2}$ is equal to $:$
	   \hfill{\brak{2024-Jan}} \\
    \begin{enumerate}
        \item $2$
        \item $2\cbrak{1-\sin{\brak{2}}}$
        \item $2\cbrak{\sin{\brak{2}}+1}$
        \item $1$
    \end{enumerate}
    \item Let $\brak{\alpha,\beta,\gamma}$ be the foot of perpendicular from the point $\brak{1,2,3}$ on the line $\frac{x+3}{5}=\frac{y-1}{2}=\frac{z+4}{3},$ then $19\brak{\alpha+\beta+\gamma}$ is equal to $:$
	   \hfill{\brak{2024-Jan}} \\
    \begin{enumerate}
        \item $102$
        \item $101$
        \item $99$
        \item $100$
    \end{enumerate}
    \item Two integers $x$ and $y$ are chosen with replacement from the set $\cbrak{0,1,2,3,\dots,10}.$ Then the probability that $\abs{x-y}\textgreater 5$ is $:$
	   \hfill{\brak{2024-Jan}} \\
    \begin{enumerate}
        \item $\frac{30}{121}$
        \item $\frac{62}{121}$
        \item $\frac{60}{121}$
        \item $\frac{31}{121}$
    \end{enumerate}
    \item If the domain of the function $$f\brak{x}=\cos^{-1}\brak{\frac{2-\abs{x}}{4}}+\brak{\log_e\brak{3-x}}^{-1}$$ is $\sbrak{\alpha,\beta}-\cbrak{y,\beta},$ then $\alpha+\beta+\gamma$ is equal to $:$
	   \hfill{\brak{2024-Jan}} \\
    \begin{enumerate}
        \item $12$
        \item $9$
        \item $11$
        \item $8$
\end{enumerate}
\item Consider the system of linear equation $x+y+z=4\mu,x+2y+2\lambda z=10\mu,x+33y+4\lambda^2z=\mu^2+15$ where $\lambda,\mu\in \textbf{R}.$ Which one of the following statements is NOT correct $?$
\hfill{\brak{2024-Jan}}	\\
\begin{enumerate}
    \item The system has unique solution if $\lambda \neq \frac{1}{2}$ and $\mu\neq 1,15$
    \item The system is inconsistent if $\lambda=\frac{1}{2}$ and $\mu\neq 1$
    \item The system has infinite number of solutions if $\lambda=\frac{1}{2}$ and $\mu=15$
    \item The system is consistent if $\lambda\neq\frac{1}{2}$
\end{enumerate}
\item If the circles $\brak{x+1}^2+\brak{y+2}^2=r^2$ and $x^2+y^2-4x-4y+4=0$ intersect at exactly two distinct points, then
\hfill{\brak{2024-Jan}}	\\
\begin{enumerate}
    \item $5\textless r\textless 9$
    \item $0\textless r\textless 7$
    \item $3\textless r\textless 7$
    \item $\frac{1}{2}\textless r\textless 7$
\end{enumerate}
\item If the lenght of the minor axis of ellipse is equal to half of the distance between the foci, then the eccentricity of the ellipse is $:$
\hfill{\brak{2024-Jan}}	\\
\begin{enumerate}
    \item $\frac{\sqrt{5}}{3}$
    \item $\frac{\sqrt{3}}{2}$
    \item $\frac{1}{\sqrt{3}}$
    \item $\frac{2}{\sqrt{5}}$
\end{enumerate}
%\end{enumerate}
