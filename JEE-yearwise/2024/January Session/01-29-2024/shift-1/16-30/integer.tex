\iffalse
    \title{2024}
    \author{EE24BTECH11021}
    \section{integer}
\fi
  \item A line with direction ratio $2,1,2$ meets the lines $x=y+2=z$ and $x+2=2y=2z$ respectively at points $P$ and $Q$. If the length of the perpendicular from the point $\brak{1,2,12}$ to the line $PQ$ is $l,$ then $l^2$ is \dots
    \hfill{[Jan-2024]}
    \item The area \brak{in\,sq.\,units} of the part of the circle $x^2+y^2=169$ which is below the line $5x-y=13$ is $\frac{\pi\alpha}{2\beta}-\frac{65}{2}+\frac{\alpha}{\beta}\sin^{-1}\brak{\frac{12}{13}}$, where $\alpha,\beta$ are coprime numbers. Then $\alpha+\beta$ is equal to \dots
    \hfill{[Jan-2024]}
    \item If the solution curve $y=y\brak{x}$ to the differential equation $\brak{1+y^2}\brak{+\log_{e}x}dx+x dy=0,x\textgreater 0$ passes through the point $\brak{1,1}$ and $y\brak{e}=\frac{\alpha-\tan\brak{\frac{3}{2}}}{\beta+\tan\brak{\frac{3}{2}}}$, then $\alpha+2\beta$ is \dots
    \hfill{[Jan-2024]}
    \item If the mean and variance of the data $65, 68, 58, 44, 48, 45, 60, \alpha, \beta, 60 where \alpha > \beta$ are $56$ and $66.2$ respectively, then $\alpha^2+\beta^2$ is equal to\dots
    \hfill{[Jan-2024]}
    \item If $\frac{\binom{11}{1}}{2}+\frac{\binom{11}{2}}{3}+\dots+\frac{\binom{11}{9}}{10}=\frac{n}{m}$with $gcd\brak{m,n}=1,$ then $m+n$ is equal to \dots 
    \hfill{[Jan-2024]}
    \item If the points of intersection of two conics $x^2+y^2=4b$ and $\frac{x^2}{16}+\frac{y^2}{b^2}=1$ lie on the curve $y^2=3x^2$, then $3\sqrt{3}$ times the area of the rectangle formed by the intersection points is \dots
    \hfill{[Jan-2024]}
    \item Let $\alpha,\beta$ be the roots of the equation $x^2-x+2=0$ with $Im\brak{\alpha}\textgreater Im\brak{\beta}$. Then $\alpha^6+\alpha^4+\beta^4-5\alpha^2$ is equal to \dots
    \hfill{[Jan-2024]}
    \item Equations of two diameters of a circle are $2x-3y=5$ and $3x-4y=7$. The line joining the points $\brak{-\frac{22}{7},-4}$ and $\brak{-\frac{1}{7},3}$ intersects the circle at only one point $P\brak{\alpha,\beta}$. Then, $17\beta-\alpha$ is equal to \dots
    \hfill{[Jan-2024]}
    \item All the letters of the word $"GTWENTY"$ are written in all possible ways with or without meaning and these words are written as in a dictionary. The serial number of the word $"GTWENTY"$ is \dots
    \hfill{[Jan-2024]}
    \item Let $f\brak{x}=2^x-x^2,x\in R$. If $m$ and $n$ are respectively the number of points t which the curves $y=f\brak{x}$ and $y=f\prime\brak{x}$ intersect the $x-axis$ then the value of $m+n$ is \dots
    \hfill{[Jan-2024]}
