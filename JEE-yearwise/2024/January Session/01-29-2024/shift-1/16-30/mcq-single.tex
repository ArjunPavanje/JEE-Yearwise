 \iffalse
    \title{2024}
    \author{EE24BTECH11021}
    \section{mcq-single}
\fi
    \item Let $PQR$ be a triangle with $R\brak{-1,4,2}$. Suppose $M\brak{2,1,2}$ is the mid-point of $PQ$. The distance of the centroid of $\triangle PQR$ from the point of intersection of the lines $\frac{x-2}{0}=\frac{y}{2}=\frac{z+3}{-1}$ and $\frac{x-1}{1}=\frac{y+3}{-3}=\frac{z+1}{1}$ is
    \hfill{[Jan-2024]}
        \begin{enumerate}
            \item $\sqrt{99}$
            \item $9$
            \item $\sqrt{69}$
            \item $69$
        \end{enumerate}
    \item Let $\overrightarrow{a},\overrightarrow{b}$ and $\overrightarrow{c}$ be three non-zero vectors such that $\overrightarrow{b}$ and $\overrightarrow{c}$ are non-collinear. If $\overrightarrow{a}+5\overrightarrow{b}$ is collinear with $\overrightarrow{c},\overrightarrow{b}+6\overrightarrow{c}$ is collinear with $\overrightarrow{a}$ and $\overrightarrow{a}+\alpha\overrightarrow{b}+\beta\overrightarrow{c}=\overrightarrow{0},$ then $\alpha+\beta$ is equal to 
    \hfill{[Jan-2024]}
        \begin{enumerate}
            \item $-25$
            \item $35$
            \item $-30$
            \item $30$
        \end{enumerate}
    \item If $z=\frac{1}{2}-2i$ is such that $\abs{z+1}=\alpha z+\beta\brak{1+i},i=\sqrt{-1}$ and $\alpha,\beta\in R$, then $\alpha+\beta$ is equal to 
    \hfill{[Jan-2024]}
        \begin{enumerate}
            \item $-1$
            \item $-4$
            \item $2$
            \item $3$
        \end{enumerate}
    \item Let $O$ be the origin and the position vectors of $A$ and $B$ be $2\hat{i}+2\hat{j}+\hat{k}$ and $2\hat{i}+4\hat{j}+4\hat{k}$ respectively. If the internal bisector of $\angle AOB$ meets the line $AB$ at $C$, then the length of $OC$ is
    \hfill{[Jan-2024]}
        \begin{enumerate}
            \item $\frac{3}{2}\sqrt{34}$
            \item $\frac{3}{2}\sqrt{31}$
            \item $\frac{2}{3}\sqrt{34}$
            \item $\frac{2}{3}\sqrt{31}$
        \end{enumerate}
    \item If the value of the integral $\int_{-\frac{\pi}{2}}^{\frac{\pi}{2}}\brak{\frac{x^2\cos{x}}{1+\pi^x}+\frac{1+\sin^2{x}}{1+e^{\sin{x}^{2023}}}} \, dx=\frac{\pi}{4}\brak{\pi+a}-2$, then the value of $a$ is 
    \hfill{[Jan-2024]}
        \begin{enumerate}
            \item $2$
            \item $-\frac{3}{2}$
            \item $\frac{3}{2}$
            \item $3$
        \end{enumerate}
