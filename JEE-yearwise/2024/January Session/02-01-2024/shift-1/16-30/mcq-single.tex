 \iffalse
    \title{2024}
    \author{EE24BTECH11021}
    \section{mcq-single}
\fi
    \item Let $\frac{x^2}{a^2}+\frac{y^2}{b^2}=1,a\textgreater b$ be an ellipse, whose eccentricity is $\frac{1}{\sqrt{2}}$ and the length of latus rectum is $\sqrt{14}$. Then the square of the eccentricity of $\frac{x^2}{a^2}-\frac{y^2}{b^2}=1$ is $\colon$
    \hfill{[Feb-2024]}
        \begin{enumerate}
            \item $3$
            \item $\frac{7}{2}$
            \item $\frac{3}{2}$
            \item $\frac{5}{2}$
        \end{enumerate}
    \item Let $3,a,b,c$ be in $A.P.$ and $3,a-1,b+1,c+9$ be in $G.P.$ Then, the arithmetic mean of $a,b$ and $c$ is $\colon$
    \hfill{[Feb-2024]}
        \begin{enumerate}
            \item $-4$
            \item $-1$
            \item $13$
            \item $11$
        \end{enumerate}
    \item Let $C\colon x^2+y^2=4$ and $C\prime\colon x^2+y^2-4\lambda x+9=0$ be two circles.If the set of all values of $\lambda$ so that the circles $C$ and $C\prime$ intersect at two distinct points, is $R-\sbrak{a,b}$, then the point $\brak{8a+12,16b-20}$ lies on the curve $\colon$
    \hfill{[Feb-2024]}
        \begin{enumerate}
            \item $x^2+2y^2-5x+6y=3$
            \item $5x^2-y=-11$
            \item $x^2-4y^2=7$
            \item $6x^2+y^2=42$
        \end{enumerate}
    \item Let $5f\brak{x}+4f\brak{\frac{1}{x}}=x^2-2,\forall x\neq 0$ and $y=9x^2f\brak{x}$, then $y$ is strictly increasing in $\colon$
    \hfill{[Feb-2024]}
        \begin{enumerate}
            \item $\brak{0,\frac{1}{\sqrt{5}}}\cup \brak{\frac{1}{\sqrt{5}},\infty}$
            \item $\brak{-\frac{1}{\sqrt{5}},0}\cup \brak{\frac{1}{\sqrt{5}},\infty}$
            \item $\brak{-\frac{1}{\sqrt{5}},0}\cup \brak{0,\frac{1}{\sqrt{5}}}$
            \item $\brak{-\infty,-\frac{1}{\sqrt{5}}}\cup \brak{0,\frac{1}{\sqrt{5}}}$
        \end{enumerate}
    \item If the shortest distance between the lines $\frac{x-\lambda}{-2}=\frac{y-2}{1}=\frac{z-1}{1}$ and $\frac{x-\sqrt{3}}{1}=\frac{y-1}{-2}=\frac{z-2}{1}$ is $1$, then the sum of all possible values of $\lambda$ is $\colon$ 
    \hfill{[Feb-2024]}
        \begin{enumerate}
            \item $0$
            \item $2\sqrt{3}$
            \item $3\sqrt{3}$
            \item $-2\sqrt{3}$
        \end{enumerate}
