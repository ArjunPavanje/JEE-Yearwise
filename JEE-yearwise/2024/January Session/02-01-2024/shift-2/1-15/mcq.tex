\iffalse
\title{2024}
\author{EE24BTECH11006}
\section{mcq-single}
\fi
\item{
Let $f\brak{x}=\abs{2x^2+5\abs{x}-3},x\in R$. If $m$ and $n$ denote the number of points where $f$ is not continuous and not differentiable respectively, then $m+n$ is equal to:
\hfill{$\brak{\text{Jan 2024}}$}
\begin{multicols}{4}
\begin{enumerate}
\item$5$
\item$2$
\item$0$
\item$3$
\end{enumerate}
\end{multicols}
}
\item{
Let $\alpha$ and $\beta$ be the roots of the equation $px^2+qx-r=0$, where $p\neq 0$. If $p$,$q$, and $r$ be the consecutive terms of a non-constant G$\cdot$P and $\frac{1}{\alpha}+\frac{1}{\beta}=\frac{3}{4}$, then the value of $\brak{\alpha -\beta}^2$ is 
\hfill{$\brak{\text{Jan 2024}}$}
\begin{multicols}{4}
\begin{enumerate}
\item $\frac{80}{9}$
\item $9$
\item $\frac{20}{3}$
\item $8$
\end{enumerate}
\end{multicols}}
\item{
The number of solutions of the equation $4\sin^2x-4\cos^3x+9-4\cos x=0;x\in\sbrak{-2\pi,2\pi}$ is: 
\hfill{$\brak{\text{Jan 2024}}$}
\begin{multicols}{4}
\begin{enumerate}
\item $1$
\item $3$
\item $2$
\item $0$
\end{enumerate}
\end{multicols}
}
\item{
The value of $\int\limits_{0}^{1}\brak{2x^3-3x^2-x+1}^{1/3}\,dx$ is equal to
\hfill{$\brak{\text{Jan 2024}}$}
\begin{multicols}{4}
\begin{enumerate}
\item $0$
\item $1$
\item $2$
\item $-1$
\end{enumerate}
\end{multicols}
}
\item{
Let $P$ be a point on the ellipse $\frac{x^2}{9}+\frac{y^2}{4}=1$ . Let the line passing through $P$ and parallel to the $y$-axis meet the circle $x^2+y^2=9$ at point $Q$ such that $P$ and $Q$ are on the same side of the $x$-axis. Then, the eccentricity of the locus of the point $R$ on $PQ$ such that $PR:RQ=4:3$ as $P$ moves along the ellipse,is:
\hfill{$\brak{\text{Jan 2024}}$}
\begin{multicols}{4}
\begin{enumerate}
\item $\frac{11}{19}$
\item $\frac{13}{21}$
\item $\frac{\sqrt{139}}{23}$
\item $\frac{\sqrt{13}}{7}$
\end{enumerate}
\end{multicols}
}
\item{
Let $m$ and $n$ be the coefficient of seventh and thirteenth terms respectively in the expansion of $\brak{\frac{1}{3}x^{\frac{1}{3}}+\frac{1}{2x^\frac{2}{3}}}^{18}$ .Then $\brak{\frac{n}{m}}^\frac{1}{3}$ is:
\hfill{$\brak{\text{Jan 2024}}$}
\begin{multicols}{4}
\begin{enumerate}
\item $\frac{4}{9}$
\item $\frac{1}{9}$
\item $\frac{1}{4}$
\item $\frac{9}{4}$
\end{enumerate}
\end{multicols}
}
\item{
Let $\alpha$ be a non-zero real number. Suppose $f:R\to R$ is a differentiable function such that $f\brak{0}=2$ and $\lim_{x\to\infty}f\brak{x}=1$ . If $f\prime\brak{x}=\alpha f\brak{x}+3$, for all $x\in R$, then $f\brak{\log_e2}$ is equal to:
\hfill{$\brak{\text{Jan 2024}}$}
\begin{multicols}{4}
\begin{enumerate}
\item $3$
\item $5$
\item $9$
\item $7$
\end{enumerate}
\end{multicols}
}
\item{
Let $P$ and $Q$ be the points on the line $\frac{x+3}{8}=\frac{y-4}{8}=\frac{z+1}{2}$ which are at a distance of $6$ units from the point $R\brak{1,2,3}$. If the centroid of the triangle $PQR$ is $\brak{\alpha,\beta,\gamma}$, then $\alpha^2+\beta^2+\gamma^2$ is:
\hfill{$\brak{\text{Jan 2024}}$}
\begin{multicols}{4}
\begin{enumerate}
\item $26$
\item $36$
\item $18$
\item $24$
\end{enumerate}
\end{multicols}
}
\item{
Consider a $\triangle ABC$ where $A\brak{1,3,2}$, $B\brak{-2,8,0}$, and $\brak{3,6,7}$. If the angle bisector of $\angle BAC$ meets the line $BC$ at $D$, then the length of the projection of the vector $\overrightarrow{AD}$ on the vector $\overrightarrow{AC}$ is:
\hfill{$\brak{\text{Jan 2024}}$}
\begin{multicols}{4}
\begin{enumerate}
\item $\frac{37}{2\sqrt{38}}$
\item $\frac{\sqrt{38}}{2}$
\item $\frac{39}{2\sqrt{38}}$
\item $\sqrt{19}$
\end{enumerate}
\end{multicols}
}
\item{
Let $S_n$ denote the sum of the first $n$ terms of an arithmetic progression. If $S_n=390$ and the ratio of the tenth and the fifth terms is $15:7$, then $S_{15}-S_5$ is equal to:
\hfill{$\brak{\text{Jan 2024}}$}
\begin{multicols}{4}
\begin{enumerate}
\item $800$
\item $890$
\item $790$
\item $690$
\end{enumerate}
\end{multicols}
}
\item{
If $\int_{0}^{\frac{\pi}{3}}\cos^4x\,dx=a\pi+b\sqrt{3}$, where $a$ and $b$ are rational numbers, then $9a+8b$ is equal to:
\hfill{$\brak{\text{Jan 2024}}$}
\begin{multicols}{4}
\begin{enumerate}
\item $2$
\item $1$
\item $3$
\item $\frac{3}{2}$
\end{enumerate}
\end{multicols}
}
\item{
If $z$ is a complex number such that $\abs{z}\geq 1$, then the minimum value of $\abs{z+\frac{1}{2}\brak{3+4i}}$
\hfill{$\brak{\text{Jan 2024}}$}
\begin{multicols}{4}
\begin{enumerate}
\item $\frac{5}{2}$
\item $2$
\item $3$
\item $\brak{3}{2}$
\end{enumerate}
\end{multicols}
}
\item{
If the domain of the function $f\brak{x}=\frac{\sqrt{x^2-25}}{\brak{4-x^2}}+\log_{10}\brak{x^2+2x-15}$ is $\brak{-\infty,\alpha}\cup[\beta,\infty)$, then $\alpha ^2+\beta ^3$ is equal to:
\hfill{$\brak{\text{Jan 2024}}$}
\begin{multicols}{4}
\begin{enumerate}
\item $140$
\item $175$
\item $150$
\item $125$
\end{enumerate}
\end{multicols}
}
\item{
Consider the relations $R_1$ and $R_2$ defined as $aR_1b\iff a^2+b^2=1$ for all $a,b\in R$ and $\brak{a,b}R_2\brak{c,d}\iff a+d=b+c$ for all $\brak{a,b},\brak{c,d}\in N\times N$. Then
\hfill{$\brak{\text{Jan 2024}}$}
\begin{enumerate}
\item Only $R_1$ is an equivalence relation 
\item Only $R_2$ is an equivalence relation 
\item $R_1$ and $R_2$ both are equivalence relations
\item Neither $R_1$ nor $R_2$ is an equivalence relation
\end{enumerate}
}
\item{
If the mirror image of the point $P\brak{3,4,9}$ in the line $\frac{x-1}{3}=\frac{y+1}{2}=\frac{z-2}{1}$ is $\brak{\alpha,\beta,\gamma}$, then $14\brak{\alpha+\beta+\gamma}$ is:
\hfill{$\brak{\text{Jan 2024}}$}
\begin{multicols}{4}
\begin{enumerate}
\item $102$
\item $138$
\item $108$
\item $132$
\end{enumerate}
\end{multicols}
}
