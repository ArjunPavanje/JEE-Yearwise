\iffalse
\title{2024}
\author{EE24BTECH11036}
\section{mcq-single}
\fi

\item If f\brak{x} = $ \myvec{\cos{x} & -\sin{x} & 0 \\ \sin{x} & \cos{x} & 0 \\ 0 & 0 & 1 } $ then \\ Statement 1 : f\brak{-x} is inverse of f\brak{x} . \\ Statement 2 :f\brak{x + y} = f\brak{x}f\brak{y}. \hfill{[Jan 2024]}
\begin{enumerate}
\item Both are true 
\item Both are false 
\item Only statement 1 is true 
\item Only statement 2 is true 
\end{enumerate}
\item If $S=\cbrak{z\in\mathbb{C}:\abs{z-i}=\abs{z-1}=\abs{z+i}}$ , then the $n\brak{S}$ is : \hfill{[Jan 2024]}
\begin{enumerate}
\begin{multicols}{4}
\item $ 2 $
\item $ 3 $
\item $ 1 $
\item $ 0 $
\end{multicols}
\end{enumerate}
\item If a = $ \lim\limits_{x\to0} \frac{\brak{\sqrt{1+\sqrt{1+x^4}}}-\sqrt{2}}{x^4} $ and  b = $ \lim\limits_{x\to0} \frac{\sin^2{x}}{\sqrt{2}-\sqrt{1+\cos{x}}} $ , find $ ab^3 $\hfill{[Jan 2024]}
\begin{enumerate}
\begin{multicols}{4}
\item $ 36 $
\item $ 32 $
\item $ 25 $
\item $ 30 $
\end{multicols}
\end{enumerate}
\item Let $ \int_0^1 \frac{1}{\sqrt{x+1}+\sqrt{x+3}} \, dx $ = a+b$ \sqrt{2} $+$ c\sqrt{3} $ where $ a,b,c $ are rational numbers, then  $ 2a+3b-4c $ is equal to :\hfill{[Jan 2024]}
\begin{enumerate}
\begin{multicols}{4}
\item $ 10 $
\item $ 8 $
\item $ 4 $
\item $ 7 $
\end{multicols}
\end{enumerate}
\item The distance of the point \brak{7,-2,11} from the line $ \frac{x-6}{1}=\frac{y-4}{0}=\frac{z-8}{3} $ along the line $ \frac{x-5}{2}=\frac{y-4}{-3}=\frac{z-5}{6} $ , is :\hfill{[Jan 2024]}
\begin{enumerate}   
\begin{multicols}{4}
\item $14$
\item $21$
\item $12$
\item $18$                                                                        
\end{multicols}
\end{enumerate}
\item The length of chord of thw ellipse $ \frac{x^2}{25} + \frac{y^2}{16} = 1 $ , whose midpoint is $ \brak{1,\frac{2}{5}} $ , is equal to: \hfill{[Jan 2024]}
\begin{enumerate}
\begin{multicols}{4}
\item $ \frac{\sqrt{1691}}{5} $
\item $ \frac{\sqrt{2009}}{5} $
\item $ \frac{\sqrt{1741}}{5} $
\item $ \frac{\sqrt{1541}}{5} $
\end{multicols}
\end{enumerate}
\item Find number of common terms in the two given series ; \\ $ 4, 9, 14, 19,\cdot\cdot\cdot $ up to $ 25 $ terms and $ 3, 9, 15, 21,\cdot\cdot\cdot $ up to $ 37 $ terms \hfill{[Jan 2024]}
\begin{enumerate}
\begin{multicols}{4}
\item $9$
\item $8$
\item $5$
\item $7$
\end{multicols}
\end{enumerate}
\item If the shortest distance of the parabola $ y^2=4x $ from the centre of the circle $ x^2+y^2-4x-16y+64=0 $ is $d$ ,then $ d^2 $ is equal to :\hfill{[Jan 2024]}
\begin{enumerate}   
\begin{multicols}{4}
\item $36$
\item $20$
\item $16$
\item $24$
\end{multicols}	
\end{enumerate}
\item Let $S=\cbrak{1,2,3,...,10}$  . Suppose $ M $ is the set of all subsets of $ S $ , the relation $R=\cbrak{ (A,B):A \cap B \neq \phi ;A,B \in M }$ is : \hfill{[Jan 2024]}
\begin{enumerate}
\item symmetric and transitive only 
\item symmetric only 
\item symmetric and reflexive only 
\item reflexive only 
\end{enumerate}
\item Let $x=x\brak{t}$ and $y=y\brak{t}$ be the solutions of the diffrential equations $\frac{dx}{dt}+ax=0$ and $\frac{dy}{dt}+by=0$ respectively, $a,b \in \mathbb{R}$ . Given that $x\brak{0}=2$ , $y\brak{0}=1$ and $3y\brak{1}=2x\brak{1}$ , the value of $t$ , for which $x\brak{t}=y\brak{t}$ , is :  \hfill{[Jan 2024]}
\begin{enumerate}   
\begin{multicols}{2}
\item $ \log_3 4 $
\item $ \log_{\frac{4}{3}} 2 $
\item $ \log_4 3 $
\item $ \log_2 2 $
\end{multicols}
\end{enumerate}
\item If $\comb{n-1}{r} = \brak{k^2 - 8}\comb{n}{r+1}$ , then the range of $k$ is \hfill{[Jan 2024]}
\begin{enumerate}
\begin{multicols}{2}
\item $ \sqrt{2}<k \leq 3 $
\item $ 2\sqrt{2}<k<3 $
\item $ 2 \leq k<3 $
\item $ 2\sqrt{2}<k<8 $
\end{multicols}
\end{enumerate} \\
\item If the shortest distance between the lines $ \frac{x-4}{1}=\frac{y+1}{2}=\frac{z}{-3} $ and $ \frac{x-\lambda}{2}=\frac{y+1}{4}=\frac{z-2}{-5} $ , is $\frac{6}{\sqrt{5}}$ , then the sum of all possible values of $\lambda$ is : \hfill{[Jan 2024]}
\begin{enumerate}   
\begin{multicols}{4}
\item $10$
\item $5$
\item $8$
\item $7$                                                                        
\end{multicols}
\end{enumerate}
\item Let $a=\hat{i}+2\hat{j}+\hat{k}$ , $b=3(\hat{i}-\hat{j}+\hat{k})$ $\cdot$ Let $c$ be the vector such that $a \times c=b$ and $a$$\cdot$$c=3$ . Let $a$$\cdot$$\brak{\brak{b \times c}-b-c}$ is equal to : \hfill{[Jan 2024]}
\begin{enumerate}   
\begin{multicols}{4}
\item $24$
\item $36$ 
\item $32$ 
\item $20$                                                                  
\end{multicols}
\end{enumerate}
\item If $A$ denotes the sum of all the coeeficients in the expansion of $\brak{1-3x+10x^2}^{n}$ and $B$ enotes the sum of all the coefficients in the expansion of $\brak{1+x^2}^{n}$ , then :\hfill{[Jan 2024]}
\begin{enumerate}
\begin{multicols}{4}
\item $ A=B^3 $
\item $ A=3B $
\item $ B=A^3 $
\item $ 3A=B $
\end{multicols}
\end{enumerate}
\item Consider the line $L:4x+5y=20$ . Let two other lines are $L_1$ and $L_2$ which trisect the line L and pass through origin, then tangent of angle between lines $L_1$ and $L_2$ is \hfill{[Jan 2024]}
\begin{enumerate}
\begin{multicols}{4}
\item $ \frac{25}{41} $
\item $ \frac{30}{41} $
\item $ \frac{2}{5} $
\item $ \frac{3}{5} $
\end{multicols}
\end{enumerate}
\end{enumerate}
\end{document}
