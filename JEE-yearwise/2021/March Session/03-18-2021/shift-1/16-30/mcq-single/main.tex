\iffalse


\title{assignment}
\author{EE24BTECH11020}
\section{mcq-single}
\fi




%\begin{enumerate}



\item If $\lim\limits_{x \to 0} \frac{\sin^{-1} x - \tan^{-1} x}{3x^3}$ is equal to $L$, then the value of $(6L + 1)$ is: \hfill (March 2021)
\begin{multicols}{4}
\begin{enumerate}
     \item $\frac{1}{2}$
    \item $2$
    \item $\frac{1}{6}$
    \item $6$
\end{enumerate}
\end{multicols}

   
\item For all four circles $M, N, O$, and $P$, the following four equations are given:
\begin{align*}
\text{Circle M}:  &  x^2 + y^2 = 1 \\
\text{Circle N}:  &  x^2 + y^2 - 2x = 0 \\
\text{Circle O}:  &  x^2 + y^2 - 2x - 2y + 1 = 0 \\
\text{Circle P}:  &  x^2 + y^2 - 2y = 0
\end{align*}

If the center of circle $M$ is joined with the center of the circle $N$, further, the center of circle $N$ is joined with the center of the circle $O$, the center of circle $O$ is joined with the center of circle $P$, and lastly, the center of circle $P$ is joined with the center of circle $M$, then these lines form the sides of a:\hfill (March 2021)
\begin{multicols}{2}
\begin{enumerate}
    \item Rectangle
    \item Square
    \item Parallelogram
    \item Rhombus
\end{enumerate}
    
\end{multicols}


\item Let $(1 + x + 2x^2)^{20} = a_0 + a_1x + a_2x^2 + \cdots + a_{40}x^{40}$. Then, $a_1 + a_3 + a_5 + \cdots + a_{37}$ is equal to:\hfill (March 2021)

\begin{enumerate}
    \begin{multicols}{2}
    \item $2^{20}(2^{20} + 21)$
    \item $2^{19}(2^{20} + 21)$ 
    \item $2^{20}(2^{20} - 21)$
    \item $2^{19}(2^{20} - 21)$
    \end{multicols}
    
\end{enumerate}
   

\item Let $
A + 2B = \myvec{1 & 2 & 0 \\
6 & -3 & 3 \\
-5 & 3 & 1 }$
 and 
  $2A - B = \myvec{2 & -1 & 5 \\
2 & -1 & 6 \\
0 & 1 & 2}
$ If $\text{Tr}\brak{A}$ denotes the sum of all diagonal elements of the matrix $A$, then $\text{Tr}\brak{A} - \text{Tr}\brak{B}$ has value equal to:\hfill (March 2021)

\begin{enumerate}
  \begin{multicols}{4}
    \item 0
    \item 1
    \item 3
    \item 2
  \end{multicols}
  
\end{enumerate}
\item The equations of one of the straight lines which pass through the point \brak{1, 3} and make an angle $\tan^{-1} \sqrt{2}$ with the straight line $y + 1 = 3\sqrt{2}x$ is:\hfill (March 2021)

\begin{enumerate}
   \begin{multicols}{2}
    \item $5\sqrt{2}x + 4y - 15 + 4\sqrt{2} = 0$
    \item $4\sqrt{2}x - 5y - 5 + 4\sqrt{2} = 0$
    \item $4\sqrt{2}x + 5y - 4\sqrt{2} = 0$
    \item $4\sqrt{2}x + 5y - (15 + 4\sqrt{2}) = 0$
   \end{multicols}
    
\end{enumerate}

  




%\end{enumerate}



