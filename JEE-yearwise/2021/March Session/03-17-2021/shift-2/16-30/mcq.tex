\iffalse
\title{March 2021, shift 2}
\author{EE24BTECH11062}
\section{mcq-single}
\fi

%\begin{enumerate}  
\item If the sides $AB, BC$, and $CA$ of a triangle $ABC$ have, 3, 5 and 6 interior points respectively, then the total number of triangles that can be constructed using these points as vertices is equal to: \hfill{[March 2021]}
\begin{multicols}{4}
    a) 360\\
    b) 240\\
    c) 333\\
    d) 364
\end{multicols}
 \item The value of $\frac{\lim_{r \to \infty}\sbrak{r}+\sbrak{2r}+\dots\sbrak{nr}}{\sbrak{n^2}}$, where r is a non-zero number and \sbrak{r} denotes the greatest integer less than or equal to $r$, is equal to : \hfill{[March 2021]}
 \begin{multicols}{4}
     a) 0\\
     b) $r$\\
     c) $r/2$\\
     d) $2r$
 \end{multicols}
 
 \item The value of $\sum_{r=0}^{6}\brak{^6C_r.^6C_{6-r}}$ is equal to : \hfill{[March 2021]}
 \begin{multicols}{4}
    a) 1124\\
    b) 924\\
    c) 1324\\
    d) 1024
 \end{multicols}
 
\item Two tangents are drawn from a point $\vec{P}$ to the circle $x^2 + y^2 -2x-4y+4=0$, such that the angle between these tangents is  $\tan^{-1}\brak{\frac{12}{5}}$, where $\tan^{-1}\brak{\frac{12}{5}}\in \brak{0,\pi}$. If the centre of the circle is denoted by $\vec{C}$ and these tangents touch the circle at points $\vec{A}$ and $\vec{B}$, then the ratio of the areas of $\triangle PAB$ and $\triangle CAB$ is: \hfill{[March 2021]}
\begin{multicols}{4}
    a) 11:4\\
    b) 9:4\\
    c) 2:1\\
    d) 3:1
\end{multicols}

\item The number of solutions of the equation $\sin^{-1}\sbrak{x^2+\brak{\frac{1}{3}}}+\cos^{-1}\sbrak{x^2-\brak{\frac{2}{3}}}=x^2$, for $x \in \sbrak{-1,1}$, and \sbrak{x} denotes the greatest integer less than or equal to $x$, is: \hfill{[March 2021]}
\begin{multicols}{4}
    a) 0\\
    b) 2\\
    c) 4\\
    d) infinite
\end{multicols}
