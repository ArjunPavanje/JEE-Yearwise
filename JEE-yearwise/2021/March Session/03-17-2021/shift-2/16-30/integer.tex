\iffalse
\title{March 2021, shift 2}
\author{EE24BTECH11062}
\section{integer}
\fi

%begin{enumerate}
\item Let the coefficients of third, fourth and fifth terms in the expansion of $\sbrak{x+\brak{\frac{a}{x^2}}}^n, x\neq 0$ be in the ratio $12:8:3$. Then the term independent of $x$ in the expansion is equal to  \hfill{[March 2021]}

\item Let $A=\begin{bmatrix}a & b \\ c& d\end{bmatrix}$ and $B=\begin{bmatrix}\alpha \\ \beta\end{bmatrix}\neq \begin{bmatrix}0\\0\end{bmatrix}$ such that $AB=B$ and $a+d=2021$, then the value of $ad-bc$ is equal to \hfill{[March 2021]}

\item Let $f: \sbrak{-1,1}\mapsto R$ be defined as $f\brak{x}=ax^2 +bx+c$ for all $x\in \sbrak{-1,1}$, where $a,b,c \in R$ such that $f\brak{-1}=2, f^{\prime}\brak{-1}=1$ and for $x\in \sbrak{-1,1}$ the maximum value of $f^{\prime\prime}\brak{x}$ is $\frac{1}{2}$. If $f\brak{x}\leq \alpha, x\in \sbrak{-1,1},$ then least value of $\alpha$ is equal to \hfill{[March 2021]}

\item Let $I_n = 
\int_{1}^{e} x^{19}\brak{log\abs{x}}^n \, dx$, where $n \in N$. If $20\brak{I_{10}}=\alpha I_9+\beta I_8$, for natural numbers $\alpha$ and $\beta$, then $\alpha-\beta$ is equal to \hfill{[March 2021]}

\item Let $f:\sbrak{-3,1}\mapsto R$ be given as $f\brak{x}=$ $\begin{cases}
min\sbrak{\brak{x+6,x^2}}, -3\leq x \leq 0\\
max\sbrak{\sqrt{x}, x^2}, 0\leq x \leq 1
\end{cases}$
If the area bounded by $y=f\brak{x}$ and x-axis is $A$, then the value of $6A$ is equal to \hfill{[March 2021]}
\item Let $\vec{x}$ be a vector in the plane containing vectors $a=2i-j+k$ and $b=i+2j-k$. If the vector $\vec{x}$ is perpendicular to \brak{3i+2j-k} and its projection on $a$ is $\frac{17\sqrt{6}}{2}$, then the value of $x^2$ is equal to \hfill{[March 2021]}
\item Consider a set of $3n$ numbers having variance 4. In this set, the mean of the first $2n$ numbers is 6 and the mean of the remaining $n$ numbers is 3. A new set is constructed by adding 1 into each of the first $2n$ numbers and subtracting 1 from each of the remaining $n$ numbers. If the variance of the new set is $k$, then $9k$ is equal to \hfill{[March 2021]}
\item If 1, $log_{10}\brak{4^x-2}$ and $log_{10}\brak{4^x+\brak{\frac{18}{5}}}$ are in arithmetic progression for a real number $x$, then the value of the determinant $\begin{bmatrix}2\sbrak{x-\brak{1/2}}& x-1 & x^2 \\ 1 & 0 & x \\ x & 1 & 0\end{bmatrix}$ is equal to \hfill{[March 2021]}
\item Let $\vec{P}$ be an arbitary point having the sum of squares of the distances from the planes $x+y+z=0, lx-nz=0$ and $x-2y+z=0$, equal to 9. If the locus of the point $\vec{P}$ is $x^2+y^2+z^2=9$ then the value of $l-n$ is equal to \hfill{[March 2021]}
\item Let $\tan \alpha, \tan \beta$ and $\tan \gamma;\alpha,\beta,\gamma\neq\sbrak{2n-1}\frac{\pi}{2},n\in N$ be the slopes of three-line segment $OA,OB$ and $OC$, respectively, where $\vec{O}$ is origin. If the circumcentre of triangle $ABC$ coincides with the origin and its orthocentre lies on y-axis, then the value of $\sbrak{\brak{\cos 3\alpha +\cos 3\beta +\cos 3\gamma}/\brak{\cos \alpha *\cos \beta *\cos \gamma}}^2$ is equal to \hfill{[March 2021]}
%end{enumerate}
%end{document}
