\iffalse
 	\title{2021}
 	\author{EE24BTECH11008}
 	\section{mcq-single}
\fi
% \begin{enumerate}
    \item Let $\sbrak{x}$ denote greatest integer less than or equal to $x.$ If for $n \in \textbf{N},$ $\brak{1-x+x^3}^n=$ $\sum_{j=0}^{3n}a_jx^j$, then $\sum_{j=0}^{\sbrak{\frac{3n}{2}}}a_{2j}$$+4$$\sum_{j=0}^{\sbrak{\frac{3n-1}{2}}}a_{2j}+1$ is equal to $:$
	    \hfill{\brak{2021-Mar}} \\
    \begin{enumerate}
        \item $2$
        \item $2^{n-1}$
        \item $1$
        \item $n$
    \end{enumerate}
    \item If $y=y\brak{x}$ is the solution of the differential equation, $\frac{dy}{dx}+2y\tan x=\sin x,y\brak{\frac{\pi}{3}}=0,$ then the maximum value of the function $y\brak{x}$ over $R$ is equal to $:$
	    \hfill{\brak{2021-Mar}} \\
    \begin{enumerate}
        \item $8$
        \item $\frac{1}{2}$
        \item $-\frac{15}{4}$
        \item $\frac{1}{8}$
    \end{enumerate}
    \item The locus of the midpoints of the chord of the circle, $x^2+y^2=25$ which is tangent to the hyperbola, $\frac{x^2}{9}-\frac{y^2}{16}=1$ is $:$
	    \hfill{\brak{2021-Mar}} \\
    \begin{enumerate}
        \item $\brak{x^2+y^2}^2-16x^2+9y^2=0$
        \item $\brak{x^2+y^2}^2-9x^2+144y^2=0$
        \item $\brak{x^2+y^2}^2-9x^2-16y^2=0$
        \item $\brak{x^2+y^2}^2-9x^2+16y^2=0$
    \end{enumerate}
    \item The number of roots of the equation, $\brak{81}^{\sin^2x} +\brak{81}^{\cos^2x} =30$ in the interval $\sbrak{0, \pi}$ is equal to $:$
	    \hfill{\brak{2021-Mar}}  \\

    \begin{enumerate}
        \item $3$
        \item $4$
        \item $8$
        \item $2$
    \end{enumerate}
    \item Let $\textbf{S}_k=\sum_{r=1}^k\tan^{-1}\brak{\frac{6^r}{2^{2r+1}+3^{2r+1}}}.$ Then $\lim_{k \to \infty}\textbf{S}_k$ is equal to $:$
	    \hfill{\brak{2021-Mar}}   \\
    \begin{enumerate}
        \item $\tan^{-1}\brak{\frac{3}{2}}$
        \item $\frac{\pi}{2}$
        \item $\cot^{-1}\brak{\frac{3}{2}}$
        \item $\tan^{-1}\brak{3}$
    \end{enumerate}
% \end{enumerate}

