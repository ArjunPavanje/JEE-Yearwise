 \iffalse
    \title{2021}
    \author{EE24BTECH11021}
    \section{mcq-single}
\fi
    \item The logical equivalence of the Boolean expression $\brak{p\wedge\sim q}\implies \brak{q\vee \sim p}$ is equivalent to $\colon$
    \hfill{[Jul-2021]}
        \begin{enumerate}
            \item $q\implies p$
            \item $p\implies q$
            \item $\sim q\implies p$
            \item $p\implies \sim q$
        \end{enumerate}
    \item Let $a$ be a positive real number such that $\int_{0}^{a}e^{x-\sbrak{x}} \, dx=10e-9$ where $\sbrak{x}$ is the greatest integer less than or equal to $x$. Then $a$ is equal to $\colon$
    \hfill{[Jul-2021]}
        \begin{enumerate}
            \item $10-\log_{e}\brak{1+e}$
            \item $10+\log_{e}2$
            \item $10+\log_{e}3$
            \item $10+\log_{e}\brak{1+e}$
        \end{enumerate}
    \item The mean of $6$ numbers is $6.5$ and its variance is $10.25$. If $4$ numbers are $2,4,5\, and\, 7$, then find the other two.
    \hfill{[Jul-2021]}
        \begin{enumerate}
            \item $10,11$
            \item $3,18$
            \item $8,13$
            \item $1,20$
        \end{enumerate}
    \item The value of the integral $\int_{-1}^{1}\log_{e}\brak{\sqrt{1-x}+\sqrt{1+x}} \, dx$ is equal to $\colon$
    \hfill{[Jul-2021]}
        \begin{enumerate}
            \item $\frac{1}{2}\log_{e}2+\frac{\pi}{4}-\frac{3}{2}$
            \item $2\log_{e}2+\frac{\pi}{4}-1$
            \item $\log_{e}2+\frac{\pi}{2}-1$
            \item $2\log_{e}2+\frac{\pi}{4}-\frac{1}{2}$
        \end{enumerate}
    \item If the roots of the quadratic equation $x^2+3^{\frac{1}{4}}x+3^{\frac{1}{2}}=0$ are $\alpha$ and $\beta$, then the value of $\alpha^{96}\brak{\alpha^{12}-1}+\beta^{96}\brak{\beta^{12}-1}$
    \hfill{[Jul-2021]}
        \begin{enumerate}
            \item $50\cdot 3^{24}$
            \item $51\cdot 3^{24}$
            \item $52\cdot 3^{24}$
            \item $104\cdot 3^{24}$
        \end{enumerate}
    \item Let $A=\myvec{2&3\\a&0},a\in R$ can be wriiten as $P+Q$ where $P$ is a symmetric matrix and $Q$ is a skew symmetric matrix. If $det\brak{Q}=9$, then the modulus of the sum of all possible values of determinant of $P$ is equal to $\colon$
    \hfill{[Jul-2021]}
        \begin{enumerate}
            \item $36$
            \item $24$
            \item $45$
            \item $18$
        \end{enumerate}
    \item If $z$ and $\omega$ are two complex numbers such that $\abs{z\omega}=1$ and $\arg{z}-\arg{\omega}=\frac{3\pi}{2}$, then $\arg{\brak{\frac{1-2\Bar{z}\omega}{1+3\Bar{z}\omega}}}$ is $\colon$\\
    \brak{Here \,arg\brak{z}\,denotes\,the\,principal\,argument\,of\,complex\,number\,z}
    \hfill{[Jul-2021]}
        \begin{enumerate}
            \item $\frac{\pi}{4}$
            \item $-\frac{3\pi}{4}$
            \item $-\frac{\pi}{4}$
            \item $\frac{3\pi}{4}$
        \end{enumerate}
    \item In $\triangle ABC$, if $AB=5,\angle B=\cos^{-1}\brak{\frac{3}{5}}$ and the radius of circumcircle of triangle is $5$ units, then the area $\brak{in \, sq. \, units}$ of $\triangle ABC$ is $\colon$
    \hfill{[Jul-2021]}
        \begin{enumerate}
            \item $10+6\sqrt{2}$
            \item $8+2\sqrt{2}$
            \item $6+8\sqrt{3}$
            \item $4+2\sqrt{3}$
        \end{enumerate}
    \item Let $\sbrak{x}$ denote the greatest integer $\leq x$, where $x\in R$. If the domain of the real valued function $f\brak{x}=\sqrt{\frac{\abs{\sbrak{x}}-2}{\abs{\sbrak{x}}-3}}$ is $\brak{-\infty,a}\cup \left[b,c \right)\cup \left[4,\infty \right), a\textless b\textless c$, then the value of $a+b+c$ is $\colon$
    \hfill{[Jul-2021]}
        \begin{enumerate}
            \item $8$
            \item $1$
            \item $-2$
            \item $-3$
        \end{enumerate}
    \item Let $y=y\brak{x}$ be the solution of differential equation $x\tan\brak{\frac{y}{x}}=\brak{y\tan\brak{\frac{y}{x}}-x}dx, -1\leq x\leq 1,y\brak{\frac{1}{2}}=\frac{\pi}{6}$. Then the area of the region bounded by the curves $x=0,x=\frac{1}{\sqrt{2}}$ and $y=y\brak{x}$ in the upper half plane is $\colon$
    \hfill{[Jul-2021]}
        \begin{enumerate}
            \item $\frac{1}{8}\brak{\pi-1}$
            \item $\frac{1}{12}\brak{\pi-3}$
            \item $\frac{1}{4}\brak{\pi-2}$
            \item $\frac{1}{6}\brak{\pi-1}$
        \end{enumerate}
    \item Find the coefficient of $x^{256}$ in $\brak{1-x}^{101}\brak{x^2+x+1}^{100}$ is$\colon$
    \hfill{[Jul-2021]}
        \begin{enumerate}
            \item $\binom{100}{16}$
            \item $\binom{100}{15}$
            \item $-\binom{100}{16}$
            \item $-\binom{100}{15}$
        \end{enumerate}
    \item Let $A=\sbrak{a_{ij}}$ be a $3\times 3$ matrix, where 
    $a_{ij}
        \begin{cases}
            1 & \text{, if } i=j\\
            -x & \text{, if} \abs{i-j}=1\\
            2x+1 & \text{, otherwise}
        \end{cases}
    $\\
    Let a function $f\colon R\rightarrow R$ be defined as $f\brak{x}=det\brak{A}.$ Then the sum of maximum and minimum values of $f$ on $R$ is equal to $\colon$
    \hfill{[Jul-2021]}
        \begin{enumerate}
            \item $-\frac{20}{27}$
            \item $\frac{88}{27}$
            \item $\frac{20}{27}$
            \item $-\frac{88}{27}$
        \end{enumerate}
    \item Let $\overrightarrow{a}=2\hat{i}+\hat{j}-2\hat{k}$ and $\overrightarrow{b}=\hat{i}+\hat{j}$. If$\overrightarrow{c}$ is a vector such that $\overrightarrow{a}\cdot\overrightarrow{c}=\abs{\overrightarrow{c}},\abs{\overrightarrow{c}-\overrightarrow{a}}=2\sqrt{2}$ and the angle between $\brak{\overrightarrow{a}\times\overrightarrow{b}}$ and $\overrightarrow{c}$ is $\frac{\pi}{6}$, then the value of $\abs{\brak{\overrightarrow{a}\times\overrightarrow{b}}\times\overrightarrow{c}}$ is $\colon$
    \hfill{[Jul-2021]}
        \begin{enumerate}
            \item $\frac{2}{3}$
            \item $4$
            \item $3$
            \item $\frac{3}{2}$
        \end{enumerate}
    \item The number of solutions of $\tan^{-1}{\sqrt{x\brak{x+1}}}+\sin^{-1}{\sqrt{x^2+x+1}}=\frac{\pi}{4}$ is
    \hfill{[Jul-2021]}
        \begin{enumerate}
            \item $1$
            \item $2$
            \item $4$
            \item $0$
        \end{enumerate}
    \item Let $y=y\brak{x}$ be the solution of the differential equation $e^x\sqrt{1-y^2}dx+\brak{\frac{y}{x}}dy=0,y\brak{1}=-1$. Then the value of $\brak{y\brak{3}}^2$ is equal to $\colon$
    \hfill{[Jul-2021]}
        \begin{enumerate}
            \item $1-4e^3$
            \item $1-4e^6$
            \item $1+4e^3$
            \item $1+4e^6$
        \end{enumerate}
