
   \iffalse
   \title{Assignment}
   \author{ee24btech11059}
   \section{mcq-single}
   \fi
    \item{
          	Let $L$ be the line of intersection of planes $r \cdot \brak{i - j + 2k} = 2$ and $r \cdot \brak{2i + j - k} = 2$. If $P\brak{\alpha, \beta, \gamma}$ is the foot of perpendicular on $L$ from the point $\brak{1, 2, 0}$, then the value of $35\brak{\alpha + \beta + \gamma}$ is equal to \text{  }\hfill
                {[Jul 2021]}
                \begin{multicols}{4}
					\begin{enumerate}
						\item 101
						\item 119
						\item 143
						\item 134
					\end{enumerate}
				\end{multicols}
            }
    %code by ysiddhanth 
    \item{
           	Let $S_n$ denote the sum of the first n terms of an arithmetic progression. If $S_{10} = 530$, $S_5 = 140$, then 
           	$S_{20} - S_6$ is equal to:
           	\hfill
           	{[Jul 2021]}
                \begin{multicols}{4}
                	\begin{enumerate}
                		\item 1862
                		\item 1842
                		\item 1852
                		\item 1872
                	\end{enumerate}
                \end{multicols}
        }
\item{
        	
        	Let $f : \mathbb{R} \rightarrow \mathbb{R}$ be defined as
        	
        	\[ f\brak{x} = \begin{cases} 
        		-\frac{4}{3}x^3 + 2x^2 + 3 & \text{if } x > 0 \\
        		3xe^x & \text{if } x \leq 0 
        	\end{cases} \]
        	
        	Then $f$ is an increasing function in the interval.
        	\hfill
        	{[Jul 2021]}
        	\begin{multicols}{4}
        		\begin{enumerate}
        			\item $\brak{-\frac{1}{2}, 2}$
        			\item $\brak{0,2}$
        			\item $\brak{-1,\frac{3}{2}}$
        			\item $\brak{-3,-1}$
        		\end{enumerate}
        	\end{multicols}
        	
        }
    \item{
     
            Let $y=y\brak{x}$ be the solution of the differential equation $\csc^2x dy + 2dx = \brak{1+y \cos 2x} \csc^2x dx$, with $y\brak{\pi/4} = 0$. Then, the value of $\brak{y(0)+1}^2$ is equal to: \\ \text{ }
            \hfill
            {[Jul 2021]}
            \begin{multicols}{4}
                \begin{enumerate}
                	\item $e^\frac{1}{2}$
                	\item $e^{-\frac{1}{2}}$
                	\item $e^{-1}$
                	\item $e$
                \end{enumerate}
            \end{multicols}
        
        }
    \item{
            Four dice are thrown simultaneously and the numbers shown on these dice are recorded in $2\times 2$
            matrices. The probability that such formed matrices have all different entries and are non-singular, is :
           	\hfill
                {[Jul 2021]}
            
            \begin{multicols}{4}
				\begin{enumerate}
					\item $\frac{45}{162}$
					\item $\frac{23}{81}$
					\item $\frac{22}{81}$
					\item $\frac{43}{162}$
				\end{enumerate}
			\end{multicols}
        
        }
 	\item{
        	Let a vector $\overrightarrow{a}$ be coplanar with vectors $\overrightarrow{b} = 2\hat{i} + \hat{j}  + \hat{k} $ and $\overrightarrow{c} = \hat{i}  - \hat{j}  + \hat{k} $. If $\overrightarrow{a}$ is perpendicular to $\overrightarrow{d} = 3\hat{i} + 2\hat{j} + 6\hat{k}$, and $|\textbf{a}| = 10$. Then a possible value of $\sbrak{\overrightarrow{a}\text{ }\overrightarrow{b}\text{ }\overrightarrow{c}}+\sbrak{\overrightarrow{a}\text{ }\overrightarrow{b}\text{ }\overrightarrow{d}}+\sbrak{\overrightarrow{a}\text{ }\overrightarrow{c}\text{ }\overrightarrow{d}} $ is equal to :\text{ }
        	\hfill
        	{[Jul 2021]}
        	
        	\begin{multicols}{4}
        		\begin{enumerate}
					\item -42
					
					\item -40
					
					\item -29
					\item -38
        		\end{enumerate}
        	\end{multicols}
        	
        }
 	\item{
			If  $$\int_0^{100\pi} \frac{\sin^2 x}{e^{\frac{x}{\pi} - \sbrak{\frac{x}{\pi}}}} dx = \frac{\alpha \pi^3}{1+ 4\pi^2}$$
			
			where $[x]$ is the greatest integer less than or equal to $x$, then the value of $\alpha$ is:
			\\ \text{ }
			\hfill
			{[Jul 2021]}
			
			\begin{multicols}{4}
				\begin{enumerate}
					\item $200(1 - e ^{-1})$
					\item $100(1 - e)$
					\item $50(e - 1)$
					\item $150(e^{-1} - 1)$
				\end{enumerate}
			\end{multicols}
			
		}
 	\item{
			Let three vectors $\overrightarrow{a},\overrightarrow{b}$ and $\overrightarrow{c}$ be such that $\overrightarrow{a} \times \overrightarrow{b} = \overrightarrow{c}$, $\overrightarrow{b} \times \overrightarrow{c} =\overrightarrow{a}$ and $|\overrightarrow{a}| = 2$. Then which one of the following is not true?\text{ }
			\hfill
			{[Jul 2021]}
			
			\begin{multicols}{2}
				\begin{enumerate}
					\item $\overrightarrow{a}\times\brak{\brak{\overrightarrow{b}+\overrightarrow{c}}\times\brak{\overrightarrow{b}-\overrightarrow{c}}} = 0$
					\item Projection of $\overrightarrow{a}$ on $\brak{\overrightarrow{b}\times\overrightarrow{c}}$ is 2
					\item $\sbrak{\overrightarrow{a}\text{ }\overrightarrow{b}\text{ }\overrightarrow{c}}+\sbrak{\overrightarrow{c}\text{ }\overrightarrow{a}\text{ }\overrightarrow{b}} =8$
					\item $|3\overrightarrow{a}+\overrightarrow{b}  -2\overrightarrow{c}|^2 = 51$
				\end{enumerate}
			\end{multicols}
			
		}
    \item{
            The values of $\lambda$ and $\mu$ such that the system of equations 
            \begin{align*}
            	x+y+z &= 6, \\
            	3x+5y+5z &= 26, \\
            	x + 2y + \lambda z &= \mu
            \end{align*}
            has no solution, are : \text{ }
             \hfill
                {[Jul 2021]}
            \begin{multicols}{4}
                \begin{enumerate}
                	\item $\lambda = 3, \mu = 5$
                    \item $\lambda = 3, \mu \neq 10$
                    \item $\lambda \neq 2, \mu = 10$
                    \item $\lambda = 2, \mu \neq 10$
                \end{enumerate}
            \end{multicols}

        %code by ysiddhanth 
        
        }
    \item{
            If the shortest distance between the straight lines $3\brak{x-1}=6\brak{y-2}=2\brak{z-1}$ and $4\brak{x-2}=2\brak{y-\lambda}=\brak{z-3}$, $\lambda \in \mathbb{R}$ is $\frac{1}{\sqrt{38}}$then the integral value of $\lambda$ is equal to:
            
             \hfill
                {[Jul 2021]}
            \begin{multicols}{4}
                \begin{enumerate}
                	\item 3
                	\item 2
                	\item 5
                	\item -1
                \end{enumerate}
            \end{multicols}
        
        }
    \item{
            Which of the following Boolean expressions is not a tautology ?
             \hfill
                {[Jul 2021]}
			\begin{multicols}{2}
				\begin{enumerate}
					\item $\brak{p \Rightarrow q} \lor \brak{\neg q \Rightarrow p}$
					\item $\brak{q \Rightarrow p} \lor \brak{\neg q \Rightarrow p}$
					\item $\brak{p \Rightarrow \neg q} \lor \brak{\neg q \Rightarrow p}$
					\item $\brak{\neg p \Rightarrow q} \lor \brak{\neg q \Rightarrow p}$
				\end{enumerate}
			\end{multicols}
        
        }
    \item{
        	Let $A=\sbrak{a_{ij}}$ be a real matrix of order $3 \times 3$, such that $a_{i1}+a_{i2}+a_{i3}=1$, for $i=1,2,3$. Then, the sum of all the entries of the matrix $A^3$ is equal to:
        	
             \text{   }\hfill
                {[Jul 2021]}
				\begin{multicols}{4}
	                \begin{enumerate}
	                   	\item 2
	                   	\item 1
	                   	\item 3
	                   	\item 9
	                \end{enumerate}
				\end{multicols}
        
        }
 \item{
    	
	    	Let $\sbrak{x}$ denote the greatest integer less than or equal to $x$. Then, the values of $x \in \mathbb{R}$ satisfying the equation $[e^x] + 2 + [e^{x}+1] - 3 = 0$ lie in the interval:\\
	    	\text{   }\hfill
	    	{[Jul 2021]}
	    	\begin{multicols}{4}
	    		\begin{enumerate}
	    			\item $\lsbrak{}0, \frac{1}{e}\rbrak{}$
	    			\item $\lsbrak{} \log_e{2}, \log_e{3}\rbrak{}$
	    			\item $\lsbrak{}1, e\rbrak{}$
	    			\item $\lsbrak{}0, \log_e{2}\rbrak{}$
	    		\end{enumerate}
	    	\end{multicols}
	    	
	    }
    \item{
	
		    Let the circle $S: 36x^2 + 36y^2 - 108x + 120y + C = 0$ be such that it neither intersects nor touches the
		    co-ordinate axes. If the point of intersection of the lines, $x - 2y = 4$ and $2x - y = 5$ lies inside the
		    circle $S$, then :
			\text{   }\hfill
			{[Jul 2021]}
			\begin{multicols}{4}
				\begin{enumerate}
						\item $\frac{25}{9} < C < \frac{13}{3}$
						\item $100 < C < 165$
						\item $ 81 < C < 156$
						\item $100 < C < 156$
				\end{enumerate}
			\end{multicols}
			
		}

    \item{
        
            Let $n$ denote the number of solutions of the equation $z^2 + 3z = 0$, where $z$ is a complex number. Then the value of $\sum_{k=0}^{\infty} \frac{1}{n^k}$ is equal to:\\ \text{ }
             \hfill
              {[Jul 2021]}
			\begin{multicols}{4}              
	              	\begin{enumerate}
	              		\item $1$
	              		\item $\frac{4}{3}$
	              		\item $\frac{3}{2}$
	              		\item $2$
	              	\end{enumerate}
  			\end{multicols}      
        }



