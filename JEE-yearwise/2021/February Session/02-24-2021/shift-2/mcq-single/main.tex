\iffalse
\title{2021}
\author{EE24BTECH11012}
\section{mcq-single}
\fi

%\begin{enumerate}
	\item Let $\vec{A}$ and $\vec{B}$ be $3 \times 3$ real matrices such that $\vec{A}$ is symmetric matrix and $\vec{B}$ is skew-symmetric matrix. Then the sytem of linear equations $\brak{\vec{A^2B^2 - B^2A^2}}\vec{X} = \vec{O}$, where $\vec{X}$ is a $3 \times 1$ column matrix of unknown variables and $\vec{O}$ is a $3 \times 1$ null matrix, has: \hfill{[Feb 2021]} 
		\begin{enumerate}
			\item a unique solution
			\item exactly two solutions
			\item infinitely many solutions
			\item no solution
		\end{enumerate}
	\item If $ n\geq2 $ is a positive integer, then the sum of the series $\comb{n+1}{2} + 2\brak{\comb{2}{2} + \comb{3}{2} + \comb{4}{2} + \dots + \comb{n}{2}}$ is \hfill{[Feb 2021]}
		\begin{enumerate}
				\begin{multicols}{4}
				\item $ \frac{n\brak{n+1}^2\brak{n+2}}{12}$
				\item $ \frac{n\brak{n-1}\brak{2n+1}}{6} $
				\item $ \frac{n\brak{n+1}\brak{2n+1}}{6} $
				\item $ \frac{n\brak{2n+1}\brak{3n+1}}{6} $
				\end{multicols}
		\end{enumerate}
	\item If a curve $ y = f\brak{x} $ passes through the point $\myvec{1,2}$ and satisfies $ x\frac{dy}{dx} + y = \emph{b}x^4 $, then for what value of \emph{b}, $ \int_{1}^{2} f\brak{x} dx = \frac{62}{5} $ holds good ?\hfill{[Feb 2021]}
		\begin{enumerate}
				\begin{multicols}{4}
			\item 5
			\item $\frac{62}{5}$
			\item $\frac{31}{5}$
			\item 10
				\end{multicols}
		\end{enumerate}
	\item The area of the region: $ \textbf{R}\cbrak{\myvec{x,y} : 5x^2 \leq y \leq 2x^2 + 9 } $ is :\hfill{[Feb 2021]}
		\begin{enumerate}
				\begin{multicols}{4}
				\item $ 9\sqrt3 $
				\item $ 12\sqrt3 $
				\item $ 11\sqrt3 $
				\item $ 6\sqrt3 $
				\end{multicols}
		\end{enumerate}
	\item Let f\brak{x} be a differentiable function defined on \sbrak{0,2} such that $ f^{\prime}\brak{x} = f^{\prime}\brak{2-x} $ for all x $\in$ \brak{0,2}, $ f\brak{0} = 1 $ and $ f\brak{2} = e^2 $ . Then the value of $ \int_{0}^{2} f\brak{x} dx $ is: \hfill{[Feb 2021]}
		\begin{enumerate}
				\begin{multicols}{4}
				\item $ 1+e^2 $
				\item $ 1-e^2 $
				\item $ 2\brak{1-e^2} $
				\item $ 2\brak{1+e^2} $
				\end{multicols}
		\end{enumerate}
%\end{enumerate}
