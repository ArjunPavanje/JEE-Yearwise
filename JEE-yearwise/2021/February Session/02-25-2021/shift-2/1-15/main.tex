\iffalse
\title{Assignment}
\author{ee24btech11059}
\section{mcq-single}
\fi

    \item{
          	A plane passes through the points $A\brak{1,2,3}$, $B\brak{2,3,1}$ and $C\brak{2,4,2}$. If $O$ is the origin and $P$ is $\brak{2,-1,1}$, then the projection of vector $\overrightarrow{OP}$ on this plane is of length:\\ \text{  }\hfill
                {[Feb 2021]}
            \begin{multicols}{4}
				\begin{enumerate}
					\item $\sqrt{\frac{2}{5}}$
					
					\item $\sqrt{\frac{2}{3}}$
					
					\item $\sqrt{\frac{2}{11}}$
					
					\item $\sqrt{\frac{2}{7}}$
				\end{enumerate}
			\end{multicols}
            }
    %code by ysiddhanth 
    \item{
           	The contrapositive of the statement “If you will work, you will earn money” is: \\ \text{ }\hfill
                {[Feb 2021]}
                \begin{enumerate}
                   	\item If you will not earn money, you will not work
                   	\item You will earn money, if you will not work
                   	\item If you will earn money, you will work
                   	\item To earn money, you need to work
                \end{enumerate}
        }
\item{
        	
        	If $\alpha, \beta \in \mathbb{R}$ are such that $1 - 2i$ (here $i^2 = -1$) is a root of $z^2 + \alpha z + \beta = 0$, then $\brak{\alpha - \beta}$ is equal to:
        	\hfill
        	{[Feb 2021]}
        	\begin{multicols}{4}
        		\begin{enumerate}
        			\item 7
        			\item -3
        			\item -7
        			\item 3
        		\end{enumerate}
        	\end{multicols}
        	
        }
    \item{
     
            If $\int_{\pi/4}^{\pi/2}cot^nx dx$, then:\hfill
                {[Feb 2021]}
            \begin{multicols}{2}
                \begin{enumerate}
                    \item $\frac{1}{I_2 + I_4}$, $\frac{1}{I_3 + I_5}$, $\frac{1}{I_4 + I_6}$ are in G.P.
                    
                    \item $\frac{1}{I_2 + I_4}$, $\frac{1}{I_3 + I_5}$, $\frac{1}{I_4 + I_6}$ are in A.P.
                    
                    \item $I_2 + I_4$, $I_3 + I_5$, $I_4 + I_6$ are in A.P.
                    
                    \item $I_2 + I_4$, $I_3 + I_5$, $I_4 + I_6$ are in G.P.
                \end{enumerate}
            \end{multicols}
        
        }
    \item{
            $A = \myvec{ 1 & -\alpha \\ \alpha & \beta}$, $ AA^\top = I_2, \text{ then the value of } \alpha^4 + \beta^4 \text{ is:} $
           	\hfill
                {[Feb 2021]}
            
            \begin{multicols}{4}
				\begin{enumerate}
					\item 1
					\item 2
					\item 3
					\item 4
				\end{enumerate}
			\end{multicols}
        
        }
 	\item{
        	Let $x$ denote the total number of one-one functions from a set $A$ with 3 elements to a set $B$ with 5 elements and $y$ denote the total number of one-one functions from the set $A$ to the set $A \times B$. Then:\\ \text{ }
        	\hfill
        	{[Feb 2021]}
        	
        	\begin{multicols}{4}
        		\begin{enumerate}
					\item $y = 273x$
					
					\item $2y = 91x$
					
					\item $y = 91x$
					
					\item $2y = 273x$
        		\end{enumerate}
        	\end{multicols}
        	
        }
 	\item{
			If the curve $x^2 + 2y^2 = 2$ intersects the line $x + y = 1$ at two points P and Q, then the angle subtended by the line segment PQ at the origin is:
			\hfill
			{[Feb 2021]}
			
			\begin{multicols}{4}
				\begin{enumerate}
					\item $\frac{\pi}{2} + \tan^{-1}\brak{\frac{1}{4}}$
					\item $\frac{\pi}{2} - \tan^{-1}\brak{\frac{1}{4}}$
					\item $\frac{\pi}{2} + \tan^{-1}\brak{\frac{1}{3}}$
					\item $\frac{\pi}{2} - \tan^{-1}\brak{\frac{1}{3}}$
				\end{enumerate}
			\end{multicols}
			
		}
 	\item{
			The integral $\int\frac{e^{3log_{e}2x} + 5e^{2log_{e}2x}}{e^{4log_{e}x}+ 5e^{3log_{e}x} - 7e^{2log_{e}x}}, x>0$ is equal to:\\ \text{ }
			\hfill
			{[Feb 2021]}
			
			\begin{multicols}{2}
				\begin{enumerate}
					\item $\log_e |x^2 + 5x - 7| + c$
					
					\item $\frac{1}{4} \log_e |x^2 + 5x - 7| + c$
					
					\item $4 \log_e |x^2 + 5x - 7| + c$
					
					\item $\log_e \sqrt{x^2 + 5x - 7} + c$
				\end{enumerate}
			\end{multicols}
			
		}
    \item{
            A hyperbola passes through the foci of the ellipse $\frac{x^2}{25} + \frac{y^2}{16} = 1$ and its transverse and conjugate axes coincide with major and minor axes of the ellipse, respectively. If the product of their eccentricities is one, then the equation of the hyperbola is:\\ \text{ }
             \hfill
                {[Feb 2021]}
            \begin{multicols}{2}
                \begin{enumerate}
                    \item $\frac{x^2}{9} - \frac{y^2}{4} = 1$
                    
                    \item $\frac{x^2}{9} - \frac{y^2}{16} = 1$
                    
                    \item $x^2 - y^2 = 9$
                    
                    \item $\frac{x^2}{9} - \frac{y^2}{25} = 1$
                \end{enumerate}
            \end{multicols}

        %code by ysiddhanth 
        
        }
    \item{
            $\lim_{n \rightarrow \infty}\sbrak{\frac{1}{n} + \frac{n}{\brak{n+1}^2} + \frac{n}{\brak{n+2}^2} +……+ \frac{n}{\brak{2n-1}^2}}$ is equal to: 
             \hfill
                {[Feb 2021]}
            \begin{multicols}{4}
                \begin{enumerate}
                	\item 1
                	\item $\frac{1}{3}$
                	\item $\frac{1}{2}$
                	\item $\frac{1}{4}$
                \end{enumerate}
            \end{multicols}
        
        }
    \item{
            In a group of 400 people, 160 are smokers and non-vegetarian; 100 are smokers and vegetarian and the remaining 140 are non-smokers and vegetarian. Their chances of getting a particular chest disorder are 35\%, 20\% and 10\% respectively. A person is chosen from the group at random and is found to be suffering from chest disorder. The probability that the selected person is a smoker and non-vegetarian is: \\ \text{ }
             \hfill
                {[Feb 2021]}
			\begin{multicols}{4}
				\begin{enumerate}
					\item $\frac{7}{45}$  
					\item $\frac{8}{45}$  
					\item $\frac{14}{45}$  
					\item $\frac{28}{45}$
				\end{enumerate}
			\end{multicols}
        
        }
    \item{
        	The following system of equations,
            $
            	2x + 3y + 2z = 9 ,
            	3x + 2y + 2z = 9 ,
            	x - y + 4z = 8
            $
             \text{   }\hfill
                {[Feb 2021]}

                \begin{enumerate}
                   	\item does not have any solution 
                   	\item has a unique solution 
                   	\item has a solution $\brak{\alpha, \beta, \gamma}$ satisfying $\alpha + \beta^2 + \gamma^3 = 12$ 
                   	\item has infinitely many solutions
                \end{enumerate}

        
        }
    \item{
	
			The minimum value of
			$ f\brak{x} = a^{a^{x}} + a^{1-a^{x}}$
			where $a, x \in \mathbb{R}$ and $a > 0$, is equal to:\\
			\text{   }\hfill
			{[Feb 2021]}
			\begin{multicols}{4}
				\begin{enumerate}
						\item $a + \frac{1}{a}$
						\item $a + 1$
						\item $2a$
						\item $2\sqrt{a}$
				\end{enumerate}
			\end{multicols}
			
		}
    \item{
	
			The function \( f\brak{x} \) is given by \( f\brak{x} = \frac{5x}{5x+5} \), then the sum of the series \( f\brak{\frac{1}{20}} + f\brak{\frac{2}{20}} + f\brak{\frac{3}{20}} + \ldots + f\brak{\frac{39}{20}} \) is equal to:
			\text{   }\hfill
			{[Feb 2021]}
			\begin{multicols}{4}
				\begin{enumerate}
					\item $\frac{19}{2}$
					\item $\frac{49}{2}$
					\item $\frac{39}{2}$
					\item $\frac{29}{2}$
				\end{enumerate}
			\end{multicols}
			
		}
    \item{
        
            Let $\alpha$ and $\beta$ be the roots of $x^2 - 6x - 2 = 0$. If $a_n = \alpha^n - \beta^n$ for $n \geq 1$, then the value of $\frac{a_{10} - 2a_8}{3a_9}$ is:
             \hfill
              {[Feb 2021]}
              \begin{multicols}{4}
	              	\begin{enumerate}
	              		\item 4
	              		\item 1
	              		\item 2
	              		\item 3
	              	\end{enumerate}
              \end{multicols}
        
        }


