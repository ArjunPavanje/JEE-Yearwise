\iffalse
\title{2021}   
\author{AI24Btech11024}
\section{mcq-single}              
\fi
\item The coefficients a,b and c of the quadratic equation,$ax^{2}+bx+c=0$ are obtained by throwing a dice three times. The probability that this equation has equal roots is$\colon$ \hfill{\brak{\text{Feb 2021}}}
\begin{enumerate}
    \item $\frac{1}{54}$
    \item $\frac{1}{72}$
    \item $\frac{1}{36}$
    \item $\frac{5}{216}$
\end{enumerate}
\item Let $\alpha$ be the angle between the lines whose direction cosines satisfy the equations $l+m-n=0$ and $l^{2}+m^{2}-n^{2}=0$.Then the value of $\sin^{4}\alpha+\cos^{4}\alpha$ is$\colon$ \hfill{\brak{\text{Feb 2021}}}
\begin{enumerate}
    \item $\frac{3}{4}$
    \item $\frac{1}{2}$
    \item $\frac{5}{8}$
    \item $\frac{3}{8}$
\end{enumerate}
\item The value of the integral\\
$\int \frac{\sin \theta \cdot \sin 2\theta \brak{ \sin^{6} \theta - \sin^{4} \theta + \sin^{2} \theta } \sqrt{2 \sin^{4} \theta + 3 \sin^{2} \theta + 6}}{1 - \cos 2\theta} \, d\theta$\\is \hfill{\brak{\text{Feb 2021}}}
\begin{enumerate}
    \item $\frac{1}{18\sbrak{9-2\sin^{6}\theta-2\sin^{6}\theta-3\sin^{4}\theta-6\sin^{2}\theta}^{\brak{\frac{3}{2}}}}+c$
    \item $\frac{1}{18\sbrak{11-18\sin^{2}\theta+9\sin^{4}\theta-2\sin^{6}\theta}^{\brak{\frac{3}{2}}}}+c$
    \item $\frac{1}{18\sbrak{11-18\cos^{2}\theta+9\cos^{4}\theta-2\cos^{6}\theta}^{\brak{\frac{3}{2}}}}+c$
    \item $\frac{1}{18\sbrak{9-2\cos^{6}\theta-2\cos^{6}\theta-3\cos^{4}\theta-6\cos^{2}\theta}^{\brak{\frac{3}{2}}}}+c$
\end{enumerate}
\item A man is observing, from the top of a tower, a boat speeding point A, with uniform speed. At that point, the angle of depression of the boat with the man's eye is $30\degree$ (Ignore man's height).After sailing for 20 seconds towards the base of the tower(which is at the level of water),the boat has reached point B, where the angle of depression is $45\degree$.Then the time taken (in seconds) by the boat from B to reach the base of the tower is$\colon$ \hfill{\brak{\text{Feb 2021}}}
\begin{enumerate}
    \item $10\brak{\sqrt{3}-1}$
    \item $10\sqrt{3}$
    \item 10
    \item $10\brak{\sqrt{3}+1}$
\end{enumerate}
\item If $0\!<\theta, \, \phi < \frac{\pi}{2}, \,x=\sum_{n=0}^{\infty} \cos^{2n} \theta, \, y = \sum_{n=0}^{\infty} \sin^{2n} \phi $ and $z = \sum_{n=0}^{\infty} \cos^{2n} \theta \cdot \sin^{2n} \phi$ then$\colon$ \hfill{\brak{\text{Feb 2021}}}
\begin{enumerate}
    \item $xyz=4$
    \item $xy-z=\brak{x+y}z$
    \item $xy+yz+zx=z$
    \item $xy+z=\brak{x+y}z$
\end{enumerate}
\item The equation of the line through the point $\brak{0,1,2}$ and perpendicular to the line\\$\frac{x-1}{2}=\frac{y+1}{3}=\frac{z-1}{-2}$\\is$\colon$ \hfill{\brak{\text{Feb 2021}}}
\begin{enumerate}
    \item $\frac{x}{-3}=\frac{y-1}{4}=\frac{z-2}{3}$
    \item $\frac{x}{3}=\frac{y-1}{4}=\frac{z-2}{3}$
    \item $\frac{x}{3}=\frac{y-1}{-4}=\frac{z-2}{3}$
    \item $\frac{x}{3}=\frac{y-1}{4}=\frac{z-2}{-3}$
\end{enumerate}
\item The statement $A\to\brak{B\to A}$ is equivalent to$\colon$ \hfill{\brak{\text{Feb 2021}}}
\begin{enumerate}
    \item $A\to\brak{A\land B}$
    \item $A\to\brak{A\lor B}$
    \item $A\to\brak{A\to B}$
    \item $A\to\brak{A\leftrightarrow B}$
\end{enumerate}
\item The integer k, for which the inequality $x^{2}-2\brak{3k-1}x+8k^{2}-7>0$ is valid for every x in R is$\colon$ \hfill{\brak{\text{Feb 2021}}}
\begin{enumerate}
    \item 3
    \item 2
    \item 4
    \item 0
\end{enumerate}
\item A tangent is drawn to the parabola $y^{2}=6x$ which is perpendicular to the line $2x+y=1$.which of the following points does NOT lie on it? \hfill{\brak{\text{Feb 2021}}}
\begin{enumerate}
    \item \brak{0,3}
    \item \brak{-6,0}
    \item \brak{4,5}
    \item \brak{5,4}
\end{enumerate}
\item Let $f,g\colon N\to N$ such that $f\brak{n+1}=f\brak{n}+f\brak{1}$ for all $n\in N$ and g be any arbitrary function. Which of the following statements is NOT true? \hfill{\brak{\text{Feb 2021}}}
\begin{enumerate}
    \item f is one-one
    \item If fog is one-one, then g is one-one
    \item If g is onto, then fog is one-one
    \item If f is onto, then $f\brak{n}=n$ for all $n \in N$
\end{enumerate}
\item Let the lines $\brak{2-i}z=\brak{2+i}\overline{z}$ and $\brak{2+i}z+\brak{i-2}\overline{z}-4i=0$, $\brak{here\; i^{2}=-1}$ be normal to this circle C, then its radius is$\colon$ \hfill{\brak{\text{Feb 2021}}}
\begin{enumerate}
    \item $\frac{3}{\sqrt{2}}$
    \item $3\sqrt{2}$
    \item $\frac{3}{2\sqrt{2}}$
    \item $\frac{1}{2\sqrt{2}}$
\end{enumerate}
\item All possible values of $\theta\in\sbrak{0,2\pi}$ for which $\sin2\theta+\tan2\theta>0$ lie in \hfill{\brak{\text{Feb 2021}}}
\begin{enumerate}
    \item $\brak{0,\frac{\pi}{2}}\cup\brak{\pi,\frac{3\pi}{2}}$
    \item $\brak{0,\frac{\pi}{4}}\cup\brak{\frac{\pi}{2},\frac{3\pi}{4}}\cup\brak{\pi,\frac{5\pi}{4}}\cup\brak{\frac{3\pi}{2},\frac{7\pi}{4}}$
    \item $\brak{0,\frac{\pi}{2}}\cup\brak{\frac{\pi}{2},\frac{3\pi}{4}}\cup\brak{\pi,\frac{7\pi}{6}}$
    \item $\brak{0,\frac{\pi}{4}}\cup\brak{\frac{\pi}{2},\frac{3\pi}{4}}\cup\brak{\frac{3\pi}{2},\frac{11\pi}{6}}$
\end{enumerate}
\item The image of the point $\brak{3,5}$ in the line $x-y+1=0$, lies on$\colon$ \hfill{\brak{\text{Feb 2021}}}
\begin{enumerate}
    \item $\brak{x-2}^{2}+\brak{y-4}^{2}=4$
    \item $\brak{x-4}^{2}+\brak{y+2}^{2}=16$
    \item $\brak{x-4}^{2}+\brak{y-4}^{2}=8$
    \item $\brak{x-2}^{2}+\brak{y-2}^{2}=12$
\end{enumerate}
\item If Rolle's theorem holds for the function $f\brak{x}=x^{3}-ax^{2}+bx-4, x \in \sbrak{1,2}$ with $f^{\prime}\brak{\frac{4}{3}}=0,$ then ordered pair $\brak{a,b}$ is equal to$\colon$ \hfill{\brak{\text{Feb 2021}}}
\begin{enumerate}
    \item $\brak{-5,8}$
    \item $\brak{5,8}$
    \item $\brak{5,-8}$
    \item $\brak{-5,-8}$
\end{enumerate}
\item If the curves $\frac{x^{2}}{a}+\frac{y^{2}}{b}$ and $\frac{x^{2}}{c}+\frac{y^{2}}{d}$ intersect each other at an angle of $90\degree$, then which of the following relations is true? \hfill{\brak{\text{Feb 2021}}}
\begin{enumerate}
    \item $a+b=c+d$
    \item $a-b=c-d$
    \item $ab=\frac{\brak{c+d}}{\brak{a+b}}$
    \item $a-c=b+d$
\end{enumerate}
